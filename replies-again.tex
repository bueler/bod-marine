\documentclass[11pt,reqno]{amsart}
%prepared in AMSLaTeX, under LaTeX2e
\addtolength{\oddsidemargin}{-.65in}
\addtolength{\evensidemargin}{-.65in}
\addtolength{\topmargin}{-.3in}
\addtolength{\textwidth}{1.5in}
\addtolength{\textheight}{.6in}

\renewcommand{\baselinestretch}{1.1}

\usepackage{verbatim} % for "comment" environment

\usepackage[pdftex, colorlinks=true, plainpages=false, linkcolor=blue, citecolor=red, urlcolor=blue]{hyperref}

\newtheorem*{thm}{Theorem}
\newtheorem*{defn}{Definition}
\newtheorem*{example}{Example}
\newtheorem*{problem}{Problem}
\newtheorem*{remark}{Remark}

\newcommand{\mtt}{\texttt}
\usepackage{alltt,xspace}
\usepackage[normalem]{ulem}
\newcommand{\mfile}[1]
{\medskip\begin{quote}\scriptsize \begin{alltt}\input{#1.m}\end{alltt} \normalsize\end{quote}\medskip}

\usepackage[final]{graphicx}
\newcommand{\mfigure}[1]{\includegraphics[height=2.5in,
width=3.5in]{#1.eps}}
\newcommand{\regfigure}[2]{\includegraphics[height=#2in,
keepaspectratio=true]{#1.eps}}
\newcommand{\widefigure}[3]{\includegraphics[height=#2in,
width=#3in]{#1.eps}}

% macros
\usepackage{amssymb}

\usepackage[T1, OT1]{fontenc}
\renewcommand{\dh}{\fontencoding{T1}\selectfont{\symbol{240}}}

\newcommand{\bod}{B\"o\dh varsson\xspace}
\newcommand{\bods}{B\"o\dh varsson's}
\newcommand{\citebod}{B\"o\dh varsson (1955)\xspace}
\newcommand{\citepbod}{(B\"o\dh varsson, 1955)\xspace}

\newcommand{\bA}{\mathbf{A}}
\newcommand{\bB}{\mathbf{B}}
\newcommand{\bE}{\mathbf{E}}
\newcommand{\bF}{\mathbf{F}}
\newcommand{\bJ}{\mathbf{J}}
\newcommand{\br}{\mathbf{r}}
\newcommand{\bx}{\mathbf{x}}
\newcommand{\hbi}{\mathbf{\hat i}}
\newcommand{\hbj}{\mathbf{\hat j}}
\newcommand{\hbk}{\mathbf{\hat k}}
\newcommand{\hbn}{\mathbf{\hat n}}
\newcommand{\hbr}{\mathbf{\hat r}}
\newcommand{\hbt}{\mathbf{\hat t}}
\newcommand{\hbx}{\mathbf{\hat x}}
\newcommand{\hby}{\mathbf{\hat y}}
\newcommand{\hbz}{\mathbf{\hat z}}
\newcommand{\hbphi}{\mathbf{\hat \phi}}
\newcommand{\hbtheta}{\mathbf{\hat \theta}}
\newcommand{\complex}{\mathbb{C}}
\newcommand{\ppr}[1]{\frac{\partial #1}{\partial r}}
\newcommand{\ppt}[1]{\frac{\partial #1}{\partial t}}
\newcommand{\ppx}[1]{\frac{\partial #1}{\partial x}}
\newcommand{\ppy}[1]{\frac{\partial #1}{\partial y}}
\newcommand{\ppz}[1]{\frac{\partial #1}{\partial z}}
\newcommand{\pptheta}[1]{\frac{\partial #1}{\partial \theta}}
\newcommand{\ppphi}[1]{\frac{\partial #1}{\partial \phi}}
\newcommand{\pp}[2]{\frac{\partial #1}{\partial #2}}
\newcommand{\ppp}[2]{\frac{\partial^2 #1}{\partial^2 #2}}
\newcommand{\pppp}[3]{\frac{\partial^2 #1}{\partial #2 \partial #3}}
\newcommand{\Div}{\ensuremath{\nabla\cdot}}
\newcommand{\Curl}{\ensuremath{\nabla\times}}
\newcommand{\curl}[3]{\ensuremath{\begin{vmatrix} \hbi & \hbj & \hbk \\ \partial_x & \partial_y & \partial_z \\ #1 & #2 & #3 \end{vmatrix}}}
\newcommand{\cross}[6]{\ensuremath{\begin{vmatrix} \hbi & \hbj & \hbk \\ #1 & #2 & #3 \\ #4 & #5 & #6 \end{vmatrix}}}
\newcommand{\eps}{\epsilon}
\newcommand{\grad}{\nabla}
\newcommand{\image}{\operatorname{im}}
\newcommand{\integers}{\mathbb{Z}}
\newcommand{\ip}[2]{\ensuremath{\left<#1,#2\right>}}
\newcommand{\lam}{\lambda}
\newcommand{\lap}{\triangle}
\newcommand{\Matlab}{\textsc{Matlab}\xspace}
\newcommand{\exers}[1]{\bigskip\noindent\textbf{Exercises} #1}
\newcommand{\fexer}[2]{\bigskip\noindent\textbf{Lesson #1, \##2}\quad }
\newcommand{\prob}[1]{\bigskip\noindent\textbf{#1} }
\newcommand{\pts}[1]{(\emph{#1 pts}) }
\newcommand{\epart}[1]{\medskip\noindent\textbf{(#1)}\quad }
\newcommand{\ppart}[1]{\,\textbf{(#1)}\quad }
\newcommand{\note}[1]{[\scriptsize #1 \normalsize]}
\newcommand{\MatIN}[1]{\mtt{>> #1}}
\newcommand{\onull}{\operatorname{null}}
\newcommand{\rank}{\operatorname{rank}}
\newcommand{\range}{\operatorname{range}}
\renewcommand{\P}{\mathcal{P}}
\newcommand{\real}{\mathbb{R}}
\newcommand{\trace}{\operatorname{tr}}
\renewcommand{\Re}{\operatorname{Re}}
\renewcommand{\Im}{\operatorname{Im}}
\newcommand{\Arg}{\operatorname{Arg}}

\newcommand{\comm}[2]{\item \emph{#1}:\, #2}

\renewcommand{\ln}[2]{\comm{line #1}{#2}}
\newcommand{\lnpage}[3]{\comm{line #1 \underline{on page #2}}{#3}}
\newcommand{\lns}[2]{\comm{lines #1}{#2}}
\newcommand{\lnspage}[3]{\comm{lines #1 \underline{on page #2}}{#3}}
\newcommand{\fg}[2]{\comm{Figure #1}{#2}}
\newcommand{\eqn}[2]{\comm{equation #1}{#2}}

\newcommand{\reply}[2]{
\medskip\medskip
\item  \begin{quote}
\emph{#1}
\end{quote}

\medskip
\noindent #2}


\title[Author's replies to extraordinary reviews of \emph{An exact solution \dots}]{Author's replies to extraordinary reviews of \\ \emph{An exact solution for a steady, flow-line marine ice sheet}}

\author{Ed Bueler}

\date{\today}

\begin{document}
\maketitle

\thispagestyle{empty}



\subsection*{Chief Editor's comments and instructions}  \begin{quote}
\emph{I have now received notes from Scientific Editor, Sergio Faria advising that the above paper should be accepted for publication in the Journal of Glaciology.  My intention is to do just that, i.e. accept for publication.  Before I do however, Sergio has also forwarded some further comments from reviewers and himself that I would like you to consider.  Could you please consider these, then submit a revised manuscript.  The comments follow.}
\end{quote}

\medskip
\noindent I will indeed address the extended issues raised at this stage, namely reviewer Gladstone's comments on what drives grounding line numerical difficulties (below; starts at the bottom of this page), and the multi-person discussion of ``mass balance'' (below; starts on page \pageref{mb}).

\medskip
\noindent These extended issues are, however, driven by off-topic obsessions on the parts of the reviewers, and they have minimal relation to the actual content of my paper.  My paper has to do with (1) history of Bodvardsson's lost solution, (2) construction of an exact solution across a grounding line, (3) measurement of numerical errors with this solution.  Neither of the reviewers have complained about this actual content.  Instead, the controversies are about: (a) which of the two ``bad'' things, which we agree occur at a grounding line, has generated the prior literature on certain numerical parameterization, but which I don't use or debate in the paper, and (b) my use of the label ``mass balance'' which we apparently all agree is in utterly common use (see below; page \pageref{mb}).

\subsection*{Replies to the reviewer's comments on my revised paper and my replies}  \begin{itemize}
\reply{From Kolumban Hutter, just one line: There is probably a misprint on p 7 line 89: The reference to (13) should be a reference to (12).}{I have corrected this.  Thanks!}

\label{gl}
\reply{From Rupert Gladstone, significantly more: \medskip \\
I am happy with all responses made by the author except the following. Note that the following issue is discussed in more detail in the authors response than is encapsulated in the revised paper. Hence the following discussion is purely academic and does not affect my recommendation. \medskip \\
I remain convinced that the step change in basal drag at the grounding line is a major obstacle to marine ice sheet models, at least from the perspective of grounding line movement.}
{I agree that the step change in basal drag is a major obstacle, and I did not claim otherwise.  I did assert, in essence, that the \emph{other} major obstacle is more major.  See the next item.}

\reply{\dots The author states in his response: ``In particular, if you take an already steady-state numerical MISMIP model, with a well-resolved grounding line, and if you `zero out' the basal drag from the grounding line up-stream a good distance (e.g. by simply setting the sliding coefficient to zero, and having a smooth transition to nonzero values start substantially upstream), then all the standard difficulties and standard techniques (with which Gladstone is familiar) are needed at the then-evolving grounding line, even though there is no drop in basal drag there at all.'' \medskip \\
I don't agree with this. Can you back this up somehow? Have you carried out simulations to demonstrate this?}
{Part of the scientific method, as I understand it, is to sufficiently-describe experiments so that they are reproducible.  With regard to numerical experiments I think I have sufficiently-described an experiment so that Gladstone can reproduce it.  Gladstone can carry it out if he wishes, or not.  If he does not agree with my interpretation then he should do the experiment.  If he finds it illuminating then he should publish.\medskip \\
I have done this experiment and I believe I have correctly interpreted the result.  No, I have not published it.  No, I do not know of anybody else who has thought: ``gee \dots is a grounding line with no drag on the grounded side already bad?''  Apparently I should have published my experiments in 2008, but I figured then that no one cared because I couldn't provide any helpful resolution of the difficulties I was having at the time. \medskip \\
But it is not an experiment which is material to the paper itself!  And I did not claim as much about it as Gladstone seems to be responding to.  In particular, before the quote he reproduces, the dialog was \\
\begin{quote}
Gladstone: \emph{\dots Current thinking is that the step change in basal drag, and the associated rapid variations in stress and other properties near the grounding line, are the cause of the problem.} \medskip \\
Me: \emph{\dots  I claim \dots that this ``current thinking'' is mostly wrong, though indeed jumps in basal drag confuse matters.}
\end{quote}
\medskip
Note Gladstone says that the step change in basal drag is ``the'' cause of the ``the'' problem.  With this I disagree.  And I did not say that basal drag discontinuities somehow don't matter.  Instead I disagreed with ``current thinking'' by saying that such discontinuities ``confuse matters'' and that the other step-change is more important, namely the change in the formula for driving stress at a grounding line.}
\reply{\dots Let me provide counter examples: \medskip \\
Pattyn (2006) showed that imposing a "transition zone" improved model capability to demonstrate reversibility (though I accept there were some flaws in his experimental design). \medskip \\
The recent TCD paper by Leguy showed improved model capability (or at least a reduction in the severity of resolution requirements) when using a basal drag law in which drag goes more smoothly to zero at the grounding line instead of a step change. \medskip \\
My 2012 paper in Annals Glac (which is not referenced actually, you may wish to consider this) shows more lenient resolution requirements when the step change in drag across the grounding line is reduced (through imposing parameterised buttressing or reducing basal drag coefficient). \medskip \\
With these published studies in mind, you would need to provide some evidence to back up your above statement.}{This is a very strange use of the phrase ``counter examples'' because all three cited works involve \emph{both} a step change in basal drag \emph{and} a step change in the coefficient of the thickness gradient in the driving stress formula (because of flotation).\medskip \\
My problem with Gladstone's argument here is like this: If I claimed in 1946 that the reason test pilots were dying in straight-wing propeller planes dives at transonic speeds was that ``the'' problem is that they had straight wings then I would be wrong (by omission at least).  My claim would not be disproven, however, by citing three accidents in which straight-wing propeller planes were involved.  Propellers are actually so bad that there has never been a Mach $>1$ propeller plane.   However, straight-wing planes can go supersonic (e.g.~the first level supersonic flight in 1947 in Bell X-1), even though swept wings, in use since the early 1940s, are generally better (i.e.~less transonic vibration).  \medskip \\
I agree that Gladstone's numerical parameterizations are useful for helping in the case where both step changes are present.  This is why all versions of my manuscript have cited Gladstone et al (2010), which covers the idea adequately in my opinion.  But in this context my manuscript says ``Grounding line parameterizations [like Gladstone's may] improve global solver behavior in this case, but such considerations go beyond our scope.  No regularization of grounding line discontinuities and slope discontinuities in the formulas for $\beta(x)$ and $h(x)$ (respectively) were applied in the current paper.'' \medskip \\
The reason I chose not to add these parameterizations is they complicate the analysis of the numerical errors.  My paper actually computes numerical errors for the first time in the literature---one must have an exact solution in hand to compute numerical errors.  Readers can make use of this new exact solution exactly by measuring the numerical errors made by their schemes.  I hope that some reader demonstrates that their numerical errors are significantly smaller than mine, as that would represent objective progress. \medskip \\
I have said that the driving-stress step change is more important than the basal-drag step change in the numerical difficulties at the grounding line.  A ``counter example'' to this statement would be a published experiment where there was no step change in basal resistance, and there remained a step change in the driving stress term, and yet it was discovered that reversibility was no longer a problem.  I know of no such literature and Gladstone does not offer any.  Gladstone should, clearly, write that paper! \medskip \\
However, I can easily point to literature which opposes the idea that basal drag step changes \emph{alone} are sufficient to generate grounding line numerical difficulties.  Just read the first paper that modeled ice flow over subglacial lakes: Pattyn (2003)\footnote{F.~Pattyn, 2003. A new three-dimensional higher-order thermomechanical ice sheet model: Basic sensitivity, ice stream development, and ice flow across subglacial lakes. J.~Geophys.~Res.~108 (B8), doi:10.1029/2002JB002329.} modeled these as zero-drag patches where the driving stress still has its grounded form.  The basal drag step change is absolutely present.  Yet there is no evidence that he, or anyone else, has ever had grounding-line-like difficulties with flow over subglacial lakes, and neither do we in our models (e.g.~PISM).  But of course Gladstone could also do experiments to demonstrate this one way or another.}

\reply{I am currently using a Stokes flow model (Elmer/Ice) to carry out some idealised marine ice sheet simulations. In these simulations (and likely in the future Stokes flow will become the standard for ice sheet models) the ONLY difference between grounded and floating ice is the bedrock. This impacts on the ice sheet through basal friction and through the modification to elevation. If I understand right, conventional opinion is that the change in basal friction causes most of the problems experienced by the current generation of marine ice sheet models, whereas Bueler is claiming that the impact of bedrock on elevation is actually more important.}{Sounds good.  It would appear that these experiments are not far enough advanced to reveal a definitive answer one way or another.}

\reply{Can you provide a clear argument linking the stiffness to this switch from $h = H+b$ to $h = wH$ (where $w$ is density ratio) across the grounding line?  Maybe I've missed it somehow, but I didn't see such an argument.}
{Now \emph{this} is a good question for the paper itself! \medskip \\
In the case of my paper, I can provide a weak argument in this direction.  Namely, if the ``$k$'' in equation (19) is set to zero, so that there is no influence of the basal drag step change on the matrix ``$A(x)$'' in equation (21), then Figure 8 showing a jump in stiffness is unaltered. \medskip \\
This is a weak argument, however, because the exact solution has the basal drag step change ``baked in''; it is an exact solution of the equations I am stating.  Changing the linearization to remove the basal drag step change is not the same as linearizing around the solution without the basal drag step change. Because it is weak in this sense, I have already made the choice not to include it in the paper, as too subtle to explain. \medskip \\
The key idea here is that there is a switch from $h_x = H_x$ to $h_x = w H_x$ in the $b=$(constant) flat bed case of the driving stress formula.  Because $w \approx 0.1$, this is an order-of-magnitude jump in the coefficient of $H_x$ in the driving stress term.  And it feeds back to the mass continuity equation because near-grounding-line changes in $H$ from the mass continuity equation affect the derivative $H_x$ in the driving stress term in the stress balance.}

\reply{\dots I've seen your comment that ``It is possible that models of modified effective pressure near the grounding line (Leguy and others, 2014), in sliding laws that are parameterized by effective pressure (Schoof, 2005), can reduce this stiffness contrast.''  But this seems to link stiffness to the basal drag change rather than the geometry change.}
{Yes it does link stiffness to the basal drag change.  I don't know how much of the stiffness change is the step change in basal drag and how much is the step change in the driving stress term.  I say success in this direction is ``possible'' because I don't think it addresses the main problem, which is the latter step change.  But I don't want to claim that the idea is unfounded.  (In the terms of the aviation analogy earlier, yes I would invest in swept wings for the Concorde.)}

\reply{and from Sergio Faria: The manuscript has improved much in this revised version, and I congratulate the Author for his effort!
\medskip \\
Reviewer 2 (Gladstone) is satisfied and has just a final question about the step change in basal drag at the grounding line. The Author may wish to address briefly this question in the final version of the manuscript and/or in extended form in the reply to the Reviewer.}
{I have addressed this content above.}

\label{mb}
\reply{Reviewer 1 (Hutter) is also satisfied with the new manuscript. He just points out a potential misprint and explains further his criticism to the usage of the term ``mass balance.''
\medskip \\
As usual, disputes over nomenclature can be very difficult to settle: they often involve personal taste, yet they can be very important for the clarity of the text.  I recall talking about nomenclature conflicts with a colleague many years ago, and he remarked: ``I don't mind if somebody uses the word 'cat' in reference to a dog -- as long as this person makes the meaning clear beforehand.''}
{I agree with the sentiments in the last paragraph.  Frank Pattyn suggested ``cauliflowers'' to replace ``surface mass balance'' because he was frustrated with this debate.}

\reply{I discussed this issue (``mass balance'' nomenclature) with Reviewer 2 and the Chief Editor (Jo Jacka). In a personal note, Reviewer 2 remarked that he was ``glad to hear the term `mass balance' criticised,'' although he agreed with the Author that ``unfortunately it has become standard use and will require a more concerted effort to eradicate than can be achieved through individual paper reviews.''  Also the Chief Editor expressed some discomfort with the ambiguity of the term.  Personally, I feel at unease too.  I agree with the Author that the term ``surface mass balance'' is widespread in the glaciological literature.  In contrast, the expression ``basal mass balance'' (used in reference to basal melting rate) seems much less usual to me.  Thus, the amalgamation of both expressions into the even more ambiguous term ``mass balance'' may not only cause discomfort to many readers, but it may even confuse a skimming reader that overlooks the explanation on Lines 38-40.
\medskip \\
The Author already declared his sympathy for the term ``mass balance'' as used in the manuscript, and I am not going to force the Author to go against his predilections. Notwithstanding, maybe the Author decides reconsidering the use of this term, after reading the above opinions. Otherwise, I propose to include at least a note near the term "mass balance" in Table 1, explaining that the quantity M combines the rates of surface accumulation/ablation and basal melting.}
{In fact I don't have ``sympathy'' for the term mass balance, nor is it a ``predilection.''  I would be perfectly happy to write down the equation \emph{without attaching names to the terms at all}, which would be my natural behavior as a mathematician, but that would assure that no glaciologist would read what I write.
\medskip \\
Very specifically I have been told that our paper (Bueler and others, 2005) that used the name ``rate of surface accumulation, negative for ablation,'' for the same term under discussion, was using an obscure name and that it should be called ``surface mass balance'' to be acceptable to glaciologists.
\medskip \\
I do read the literature and I respect the efforts of glaciologists to use clear language.  There was a recent attempt by a significant group of authors under UNESCO auspices to stop endlessly repeats of this debate.  They published a glossary, namely Cogley and others (2011).  Here is what they say about the relevant term in the relevant equation: \\
\begin{quote}
\emph{4.2 Climatic mass balance and climatic-basal mass balance}
\smallskip \\
In studies of glacier dynamics, the term $\dot b$ in [the mass continuity equation $\dot h = \dot b - \nabla \cdot \mathbf{q}$] is often called the ``mass balance,'' or more appropriately the ``mass-balance rate.''  In this interpretation, ``mass balance'' excludes mass changes due to ice flow, \dots  To resolve this ambiguity, we introduce \emph{climatic-basal mass balance} as an appropriate new name for the $\dot b$ that appears in the continuity equation \dots  This terminology makes it clear that $\dot b$ represents mass changes at and near the surface, which are driven primarily by climate, and those at the bed, but not those due to flow dynamics.
\smallskip \\
Sometimes, with the aim of emphasizing this distinction, $\dot b$ is called the ``surface mass balance.''  The \emph{surface mass balance} is the sum of \emph{surface accumulation} and \emph{surface ablation}, so this usage is accurate if the internal and basal terms \dots are negligible.  \dots
\end{quote} \medskip
Based on the above advice I have enhanced the name of the term in question to ``climatic-basal mass balance rate''.  Note the basal component of this quantity is often dominant for a floating ice shelf, so I would actually be making a model error if I ignored the ``-basal'' qualification.  When referring only to the upper surface part, e.g.~in referring to Bodvardsson's construction where the term is clearly only for the upper ice surface, I continue to call it the ``surface mass balance rate.'' \medskip \\
Please don't make me go through another round of this.}

\end{itemize}


%\bigskip
%\small
%\bibliography{ice-bib}
%\bibliographystyle{igs}

\end{document}
