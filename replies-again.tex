\documentclass[11pt,reqno]{amsart}
%prepared in AMSLaTeX, under LaTeX2e
\addtolength{\oddsidemargin}{-.65in}
\addtolength{\evensidemargin}{-.65in}
\addtolength{\topmargin}{-.3in}
\addtolength{\textwidth}{1.5in}
\addtolength{\textheight}{.6in}

\renewcommand{\baselinestretch}{1.1}

\usepackage{verbatim} % for "comment" environment

\usepackage[pdftex, colorlinks=true, plainpages=false, linkcolor=blue, citecolor=red, urlcolor=blue]{hyperref}

\newtheorem*{thm}{Theorem}
\newtheorem*{defn}{Definition}
\newtheorem*{example}{Example}
\newtheorem*{problem}{Problem}
\newtheorem*{remark}{Remark}

\newcommand{\mtt}{\texttt}
\usepackage{alltt,xspace}
\usepackage[normalem]{ulem}
\newcommand{\mfile}[1]
{\medskip\begin{quote}\scriptsize \begin{alltt}\input{#1.m}\end{alltt} \normalsize\end{quote}\medskip}

\usepackage[final]{graphicx}
\newcommand{\mfigure}[1]{\includegraphics[height=2.5in,
width=3.5in]{#1.eps}}
\newcommand{\regfigure}[2]{\includegraphics[height=#2in,
keepaspectratio=true]{#1.eps}}
\newcommand{\widefigure}[3]{\includegraphics[height=#2in,
width=#3in]{#1.eps}}

% macros
\usepackage{amssymb}

\usepackage[T1, OT1]{fontenc}
\renewcommand{\dh}{\fontencoding{T1}\selectfont{\symbol{240}}}

\newcommand{\bod}{B\"o\dh varsson\xspace}
\newcommand{\bods}{B\"o\dh varsson's}
\newcommand{\citebod}{B\"o\dh varsson (1955)\xspace}
\newcommand{\citepbod}{(B\"o\dh varsson, 1955)\xspace}

\newcommand{\bA}{\mathbf{A}}
\newcommand{\bB}{\mathbf{B}}
\newcommand{\bE}{\mathbf{E}}
\newcommand{\bF}{\mathbf{F}}
\newcommand{\bJ}{\mathbf{J}}
\newcommand{\br}{\mathbf{r}}
\newcommand{\bx}{\mathbf{x}}
\newcommand{\hbi}{\mathbf{\hat i}}
\newcommand{\hbj}{\mathbf{\hat j}}
\newcommand{\hbk}{\mathbf{\hat k}}
\newcommand{\hbn}{\mathbf{\hat n}}
\newcommand{\hbr}{\mathbf{\hat r}}
\newcommand{\hbt}{\mathbf{\hat t}}
\newcommand{\hbx}{\mathbf{\hat x}}
\newcommand{\hby}{\mathbf{\hat y}}
\newcommand{\hbz}{\mathbf{\hat z}}
\newcommand{\hbphi}{\mathbf{\hat \phi}}
\newcommand{\hbtheta}{\mathbf{\hat \theta}}
\newcommand{\complex}{\mathbb{C}}
\newcommand{\ppr}[1]{\frac{\partial #1}{\partial r}}
\newcommand{\ppt}[1]{\frac{\partial #1}{\partial t}}
\newcommand{\ppx}[1]{\frac{\partial #1}{\partial x}}
\newcommand{\ppy}[1]{\frac{\partial #1}{\partial y}}
\newcommand{\ppz}[1]{\frac{\partial #1}{\partial z}}
\newcommand{\pptheta}[1]{\frac{\partial #1}{\partial \theta}}
\newcommand{\ppphi}[1]{\frac{\partial #1}{\partial \phi}}
\newcommand{\pp}[2]{\frac{\partial #1}{\partial #2}}
\newcommand{\ppp}[2]{\frac{\partial^2 #1}{\partial^2 #2}}
\newcommand{\pppp}[3]{\frac{\partial^2 #1}{\partial #2 \partial #3}}
\newcommand{\Div}{\ensuremath{\nabla\cdot}}
\newcommand{\Curl}{\ensuremath{\nabla\times}}
\newcommand{\curl}[3]{\ensuremath{\begin{vmatrix} \hbi & \hbj & \hbk \\ \partial_x & \partial_y & \partial_z \\ #1 & #2 & #3 \end{vmatrix}}}
\newcommand{\cross}[6]{\ensuremath{\begin{vmatrix} \hbi & \hbj & \hbk \\ #1 & #2 & #3 \\ #4 & #5 & #6 \end{vmatrix}}}
\newcommand{\eps}{\epsilon}
\newcommand{\grad}{\nabla}
\newcommand{\image}{\operatorname{im}}
\newcommand{\integers}{\mathbb{Z}}
\newcommand{\ip}[2]{\ensuremath{\left<#1,#2\right>}}
\newcommand{\lam}{\lambda}
\newcommand{\lap}{\triangle}
\newcommand{\Matlab}{\textsc{Matlab}\xspace}
\newcommand{\exers}[1]{\bigskip\noindent\textbf{Exercises} #1}
\newcommand{\fexer}[2]{\bigskip\noindent\textbf{Lesson #1, \##2}\quad }
\newcommand{\prob}[1]{\bigskip\noindent\textbf{#1} }
\newcommand{\pts}[1]{(\emph{#1 pts}) }
\newcommand{\epart}[1]{\medskip\noindent\textbf{(#1)}\quad }
\newcommand{\ppart}[1]{\,\textbf{(#1)}\quad }
\newcommand{\note}[1]{[\scriptsize #1 \normalsize]}
\newcommand{\MatIN}[1]{\mtt{>> #1}}
\newcommand{\onull}{\operatorname{null}}
\newcommand{\rank}{\operatorname{rank}}
\newcommand{\range}{\operatorname{range}}
\renewcommand{\P}{\mathcal{P}}
\newcommand{\real}{\mathbb{R}}
\newcommand{\trace}{\operatorname{tr}}
\renewcommand{\Re}{\operatorname{Re}}
\renewcommand{\Im}{\operatorname{Im}}
\newcommand{\Arg}{\operatorname{Arg}}

\newcommand{\comm}[2]{\item \emph{#1}:\, #2}

\renewcommand{\ln}[2]{\comm{line #1}{#2}}
\newcommand{\lnpage}[3]{\comm{line #1 \underline{on page #2}}{#3}}
\newcommand{\lns}[2]{\comm{lines #1}{#2}}
\newcommand{\lnspage}[3]{\comm{lines #1 \underline{on page #2}}{#3}}
\newcommand{\fg}[2]{\comm{Figure #1}{#2}}
\newcommand{\eqn}[2]{\comm{equation #1}{#2}}

\newcommand{\reply}[2]{
\medskip\medskip
\item  \begin{quote}
\emph{#1}
\end{quote}

\medskip
\noindent #2}


\title[Author's replies to extraordinary reviews of \emph{An exact solution \dots}]{Author's replies to extraordinary reviews of \\ \emph{An exact solution for a steady, flow-line marine ice sheet}}

\author{Ed Bueler}

\date{\today}

\begin{document}
\maketitle

\thispagestyle{empty}



\subsection*{Chief Editor's comments and instructions}  \begin{quote}
\emph{I have now received notes from Scientific Editor, Sergio Faria advising that the above paper should be accepted for publication in the Journal of Glaciology.  My intention is to do just that, i.e. accept for publication.  Before I do however, Sergio has also forwarded some further comments from reviewers and himself that I would like you to consider.  Could you please consider these, then submit a revised manuscript.  The comments follow.}
\end{quote}

\medskip
\noindent I will indeed address these points.

\medskip
\noindent But the extended issues raised at this stage, namely reviewer Gladstone's comments on what drives grounding line numerical difficulties(below; page \pageref{gl}), and the multi-person discussion of ``mass balance'' (below; page \pageref{mb}), are driven by off-topic obsessions on the parts of the reviewers, compared to the actual content of my paper.

\medskip
\noindent Surely my paper has to do with (1) history of Bodvardsson's lost solution, (2) construction of an exact solution across a grounding line, (3) measurement of numerical errors with this solution.  Neither of the reviewers have complained about this actual content.  Instead, the points below are about: (a) which of the two ``bad'' things which we agree occur at a grounding line has generated the prior literature on certain numerical treatments, which I don't use or debate in the paper, and (b) my use of the label ``mass balance'' which in utterly common use and even UNESCO has addressed at great length (see below).  Is this a good use of everyone's time?

\subsection*{Replies}  \begin{itemize}
\reply{From Kolumban Hutter, just one line: There is probably a misprint on p 7 line 89: The reference to (13) should be a reference to (12).}{I have corrected this.  Thanks!}

\label{gl}
\reply{From Rupert Gladstone, significantly more: \medskip \\
I am happy with all responses made by the author except the following. Note that the following issue is discussed in more detail in the authors response than is encapsulated in the revised paper. Hence the following discussion is purely academic and does not affect my recommendation.}{}

\reply{}{}

\reply{}{}

\reply{}{}

\reply{and from Sergio Faria \dots \medskip \\
The manuscript has improved much in this revised version, and I congratulate the Author for his effort!
\medskip \\
Reviewer 2 (Gladstone) is satisfied and has just a final question about the step change in basal drag at the grounding line. The Author may wish to address briefly this question in the final version of the manuscript and/or in extended form in the reply to the Reviewer.}
{I have addressed this content above.}

\label{mb}
\reply{Reviewer 1 (Hutter) is also satisfied with the new manuscript. He just points out a potential misprint and explains further his criticism to the usage of the term ``mass balance.''
\medskip \\
As usual, disputes over nomenclature can be very difficult to settle: they often involve personal taste, yet they can be very important for the clarity of the text.  I recall talking about nomenclature conflicts with a colleague many years ago, and he remarked: ``I don't mind if somebody uses the word 'cat' in reference to a dog -- as long as this person makes the meaning clear beforehand.''}
{I agree with the sentiments in the last paragraph.}

\reply{I discussed this issue (``mass balance'' nomenclature) with Reviewer 2 and the Chief Editor (Jo Jacka). In a personal note, Reviewer 2 remarked that he was ``glad to hear the term `mass balance' criticised,'' although he agreed with the Author that ``unfortunately it has become standard use and will require a more concerted effort to eradicate than can be achieved through individual paper reviews.''  Also the Chief Editor expressed some discomfort with the ambiguity of the term.  Personally, I feel at unease too.  I agree with the Author that the term ``surface mass balance'' is widespread in the glaciological literature.  In contrast, the expression ``basal mass balance'' (used in reference to basal melting rate) seems much less usual to me.  Thus, the amalgamation of both expressions into the even more ambiguous term ``mass balance'' may not only cause discomfort to many readers, but it may even confuse a skimming reader that overlooks the explanation on Lines 38-40.
\medskip \\
The Author already declared his sympathy for the term ``mass balance'' as used in the manuscript, and I am not going to force the Author to go against his predilections. Notwithstanding, maybe the Author decides reconsidering the use of this term, after reading the above opinions. Otherwise, I propose to include at least a note near the term "mass balance" in Table 1, explaining that the quantity M combines the rates of surface accumulation/ablation and basal melting.}
{Frank Pattyn suggested ``cauliflowers,'' for exactly this term---and exactly because he was frustrated with this debate---and used it to good effect for an entire talk.  Should I follow his example?
\medskip \\
In fact I don't have ``sympathy'' for the term mass balance, nor is it a ``predilection.''  I would be perfectly happy to write down the equation \emph{without attaching names to the terms at all}, which would be my natural behavior as a mathematician, but that would assure that no glaciologist would read what I write.
\medskip \\
Very specifically I have been told that our paper (Bueler and others, 2005) that used the name ``rate of surface accumulation, negative for ablation,'' for the same term under discussion, was using an obscure name and that it should be called ``surface mass balance'' to be acceptable to glaciologists.
\medskip \\
I do read the literature and I respect the efforts of glaciologists to use clear language.  There was a recent attempt by 11 authors under UNESCO auspices to stop endlessly repeating this debate.  They published a glossary, namely Cogley and others (2011).  Here is what they say about the relevant term in the relevant equation: \\
\begin{quote}
\emph{4.2 Climatic mass balance and climatic-basal mass balance}
\smallskip \\
In studies of glacier dynamics, the term $\dot b$ in [the mass continuity equation $\dot h = \dot b - \nabla \cdot \mathbf{q}$] is often called the ``mass balance,'' or more appropriately the ``mass-balance rate.''  In this interpretation, ``mass balance'' excludes mass changes due to ice flow, \dots  To resolve this ambiguity, we introduce \emph{climatic-basal mass balance} as an appropriate new name for the $\dot b$ that appears in the continuity equation \dots  This terminology makes it clear that $\dot b$ represents mass changes at and near the surface, which are driven primarily by climate, and those at the bed, but not those due to flow dynamics.
\smallskip \\
Sometimes, with the aim of emphasizing this distinction, $\dot b$ is called the ``surface mass balance.''  The \emph{surface mass balance} is the sum of \emph{surface accumulation} and \emph{surface ablation}, so this usage is accurate if the internal and basal terms \dots are negligible.  \dots
\end{quote} \medskip
Based on the above advice I have enhanced the name of the term in question to ``climatic-basal mass balance rate''.  Note the basal component of this quantity is often dominant for a floating ice shelf, so I would actually be making a model error to ignor the ``-basal'' qualification.  When referring only to the upper surface part, e.g.~in referring to Bodvardsson's construction where the term is clearly only for the upper ice surface, I call it the ``surface mass balance rate.'' \medskip \\
Please don't make me go through another round of this.}

\end{itemize}


%\bigskip
%\small
%\bibliography{ice-bib}
%\bibliographystyle{igs}

\end{document}
