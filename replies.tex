\documentclass[11pt,reqno]{amsart}
%prepared in AMSLaTeX, under LaTeX2e
\addtolength{\oddsidemargin}{-.65in}
\addtolength{\evensidemargin}{-.65in}
\addtolength{\topmargin}{-.3in}
\addtolength{\textwidth}{1.5in}
\addtolength{\textheight}{.6in}

\renewcommand{\baselinestretch}{1.1}

\usepackage{verbatim} % for "comment" environment

\usepackage[pdftex, colorlinks=true, plainpages=false, linkcolor=blue, citecolor=red, urlcolor=blue]{hyperref}

\newtheorem*{thm}{Theorem}
\newtheorem*{defn}{Definition}
\newtheorem*{example}{Example}
\newtheorem*{problem}{Problem}
\newtheorem*{remark}{Remark}

\newcommand{\mtt}{\texttt}
\usepackage{alltt,xspace}
\usepackage[normalem]{ulem}
\newcommand{\mfile}[1]
{\medskip\begin{quote}\scriptsize \begin{alltt}\input{#1.m}\end{alltt} \normalsize\end{quote}\medskip}

\usepackage[final]{graphicx}
\newcommand{\mfigure}[1]{\includegraphics[height=2.5in,
width=3.5in]{#1.eps}}
\newcommand{\regfigure}[2]{\includegraphics[height=#2in,
keepaspectratio=true]{#1.eps}}
\newcommand{\widefigure}[3]{\includegraphics[height=#2in,
width=#3in]{#1.eps}}

% macros
\usepackage{amssymb}

\usepackage[T1, OT1]{fontenc}
\renewcommand{\dh}{\fontencoding{T1}\selectfont{\symbol{240}}}

\newcommand{\bod}{B\"o\dh varsson\xspace}
\newcommand{\bods}{B\"o\dh varsson's}
\newcommand{\citebod}{B\"o\dh varsson (1955)\nocite{Bodvardsson}\xspace}
\newcommand{\citepbod}{(B\"o\dh varsson, 1955)\nocite{Bodvardsson}\xspace}

\newcommand{\bA}{\mathbf{A}}
\newcommand{\bB}{\mathbf{B}}
\newcommand{\bE}{\mathbf{E}}
\newcommand{\bF}{\mathbf{F}}
\newcommand{\bJ}{\mathbf{J}}
\newcommand{\br}{\mathbf{r}}
\newcommand{\bx}{\mathbf{x}}
\newcommand{\hbi}{\mathbf{\hat i}}
\newcommand{\hbj}{\mathbf{\hat j}}
\newcommand{\hbk}{\mathbf{\hat k}}
\newcommand{\hbn}{\mathbf{\hat n}}
\newcommand{\hbr}{\mathbf{\hat r}}
\newcommand{\hbt}{\mathbf{\hat t}}
\newcommand{\hbx}{\mathbf{\hat x}}
\newcommand{\hby}{\mathbf{\hat y}}
\newcommand{\hbz}{\mathbf{\hat z}}
\newcommand{\hbphi}{\mathbf{\hat \phi}}
\newcommand{\hbtheta}{\mathbf{\hat \theta}}
\newcommand{\complex}{\mathbb{C}}
\newcommand{\ppr}[1]{\frac{\partial #1}{\partial r}}
\newcommand{\ppt}[1]{\frac{\partial #1}{\partial t}}
\newcommand{\ppx}[1]{\frac{\partial #1}{\partial x}}
\newcommand{\ppy}[1]{\frac{\partial #1}{\partial y}}
\newcommand{\ppz}[1]{\frac{\partial #1}{\partial z}}
\newcommand{\pptheta}[1]{\frac{\partial #1}{\partial \theta}}
\newcommand{\ppphi}[1]{\frac{\partial #1}{\partial \phi}}
\newcommand{\pp}[2]{\frac{\partial #1}{\partial #2}}
\newcommand{\ppp}[2]{\frac{\partial^2 #1}{\partial^2 #2}}
\newcommand{\pppp}[3]{\frac{\partial^2 #1}{\partial #2 \partial #3}}
\newcommand{\Div}{\ensuremath{\nabla\cdot}}
\newcommand{\Curl}{\ensuremath{\nabla\times}}
\newcommand{\curl}[3]{\ensuremath{\begin{vmatrix} \hbi & \hbj & \hbk \\ \partial_x & \partial_y & \partial_z \\ #1 & #2 & #3 \end{vmatrix}}}
\newcommand{\cross}[6]{\ensuremath{\begin{vmatrix} \hbi & \hbj & \hbk \\ #1 & #2 & #3 \\ #4 & #5 & #6 \end{vmatrix}}}
\newcommand{\eps}{\epsilon}
\newcommand{\grad}{\nabla}
\newcommand{\image}{\operatorname{im}}
\newcommand{\integers}{\mathbb{Z}}
\newcommand{\ip}[2]{\ensuremath{\left<#1,#2\right>}}
\newcommand{\lam}{\lambda}
\newcommand{\lap}{\triangle}
\newcommand{\Matlab}{\textsc{Matlab}\xspace}
\newcommand{\exers}[1]{\bigskip\noindent\textbf{Exercises} #1}
\newcommand{\fexer}[2]{\bigskip\noindent\textbf{Lesson #1, \##2}\quad }
\newcommand{\prob}[1]{\bigskip\noindent\textbf{#1} }
\newcommand{\pts}[1]{(\emph{#1 pts}) }
\newcommand{\epart}[1]{\medskip\noindent\textbf{(#1)}\quad }
\newcommand{\ppart}[1]{\,\textbf{(#1)}\quad }
\newcommand{\note}[1]{[\scriptsize #1 \normalsize]}
\newcommand{\MatIN}[1]{\mtt{>> #1}}
\newcommand{\onull}{\operatorname{null}}
\newcommand{\rank}{\operatorname{rank}}
\newcommand{\range}{\operatorname{range}}
\renewcommand{\P}{\mathcal{P}}
\newcommand{\real}{\mathbb{R}}
\newcommand{\trace}{\operatorname{tr}}
\renewcommand{\Re}{\operatorname{Re}}
\renewcommand{\Im}{\operatorname{Im}}
\newcommand{\Arg}{\operatorname{Arg}}

\newcommand{\comm}[2]{\item \emph{#1}:\, #2}

\renewcommand{\ln}[2]{\comm{line #1}{#2}}
\newcommand{\lnpage}[3]{\comm{line #1 \underline{on page #2}}{#3}}
\newcommand{\lns}[2]{\comm{lines #1}{#2}}
\newcommand{\lnspage}[3]{\comm{lines #1 \underline{on page #2}}{#3}}
\newcommand{\fg}[2]{\comm{Figure #1}{#2}}
\newcommand{\eqn}[2]{\comm{equation #1}{#2}}

\newcommand{\reply}[2]{
\medskip\medskip
\item  \begin{quote}
\emph{#1}
\end{quote}

\medskip
\noindent #2}


\title[Author's replies to reviews of \emph{An exact solution \dots}]{Author's replies to reviews of \\ \emph{An exact solution for a steady, flow-line marine ice sheet}}

\author{Ed Bueler}

\date{\today}

\begin{document}
\maketitle

\thispagestyle{empty}



\subsection*{Editor's comments and instructions}  \begin{quote}
\emph{Scientific Editor Comments (S.H.~Faria):}

\emph{The Reviewers (K.~Hutter and R.~Gladstone) agree that this is a well-written and mathematically sound manuscript. They share however a concern about the novelty of the results. Owing to this, it is in my opinion very important to highlight more the new results of this work, and to explore further their consequences, in order to make this manuscript publishable in the Journal of Glaciology.  \dots}

\emph{Could you please examine the reviews and address, point by point, each of the issues raised by the reviewers? You should revise the manuscript in line with the reviewers' comments, or provide me with point-by-point well-argued reasons for not revising the manuscript.}
\end{quote}

\medskip
\noindent I appreciate the quick review process!  I have examined and addressed the reviewer's comments point-by-point.  I have spent most of my effort making revisions to address these points.

\medskip
\noindent I have responded to the best of my ability to both reviewers, and in the case of \#2 that has led, I hope, to several improvements to the paper.  But reviewer \#1's comments are not very helpful.  Other than tilting at windmills (i.e.~the ``term `Mass balance' \dots is incorrectly used'') and attempting to insult me (i.e.~``\dots worked out term paper in a first year graduate class \dots''), there is simply not much to respond to.

\medskip
\noindent Something that is not hidden in the paper---it is the first paragraph of the introduction and I return to it in the conclusion---drew comment from neither reviewer.  Namely, there is a historical purpose to this paper.  I show that \bod's plug flow solution both has been misunderstood\footnote{I can find no reference in the literature either to the fact that it solves a stress balance or to the fact that is an exact solution at all.} and is surprisingly-relevant to a topic of current interest, namely marine ice sheet modeling in the rapidly-sliding case.  I have added a short paragraph to the conclusion to describe the citation history of \citebod, and thus to emphasize this historical purpose.  This historical revival is part of the novelty of this paper.

\medskip
\noindent Based on other colleague's suggestions I have corrected the typography used for \bod's Icelandic name.  The revised acknowledgements clarify this too.


\subsection*{Reviewer \#1 (Hutter)}  \begin{itemize}
\reply{Professionally and technically this paper is well written, correct in its essential parts, but its global aims and scope by the author are not clear to me.}
{I appreciate this full sentiment, and I am not even surprised that the ``global aims and scope'' are unclear.  Indeed my theoretical focus on actually solving the PDEs as given is not the standard theoretical focus among mathematical glaciologists (e.g.~Hutter, Fowler, Schoof).  Textbooks like Hutter (1983), Fowler (1997), and even Greve and Blatter (2009) primarily focus on deriving PDEs and not solving them.  The reviewer and I do agree that the purpose of exact solutions is often in the building of reliable numerical models by applying rigorous understanding of the solutions (see below). \medskip \\
I hope that my aims have become clearer though the modifications I have made in response to specific comments by reviewer \#2.}

\reply{It is shown that Bodvardsson's parabolic ice profile and flow solution is also an exact solution for floating ice (van der Veen).}
{This reflects a total misunderstanding of the paper.  Reviewer \#2 is certainly not so confused.  Even the abstract says that \bod's solution is ``on a flat bed'' and then that a solution ``across the grounding line'' is built by ``connecting \bod's solution to the van der Veen (1983) solution for floating ice''. \medskip \\
I cannot imagine where the reviewer got the concept that \bod's solution is for floating ice.  Much of the paper is devoted to the different conditions on either side of the grounding line.  The formulas stating the exact solution in detail, namely (15) and (17), could not be clearer that different formulas apply to grounded ice (``$0 \le x \le x_g$'') and floating ice (``$x_g \le x \le x_c$'').}

\reply{\dots In a kind of inverse procedure---but not so named by the author---exact solutions to the ice sheet-shelf configuration are constructed by accordingly selecting the "ice stiffness" variation as a function of space. At least, this is what I believe the author does.}
{The reviewer has correctly seen that ``exact solutions \dots are constructed by \dots selecting the `ice stiffness' variation as a function of space.'' \medskip \\
\emph{Fortunately} I do not call this an ``inverse procedure!''  Such a name would be both inaccurate and counter to usage in the literature, wherein ``inverse'' procedures use model equations to determine quantities, which normally would be needed for forward solutions, by minimizing model output misfit with data.  In fact ice hardness has been computed by inversion in some literature.\footnote{For example, in E.~Larour et al.~(2005).~\emph{Rheology of the Ronne Ice Shelf, Antarctica, inferred from satellite radar interferometry data using an inverse control method}, Geophys.~Res.~Letters \textbf{32} (L05503).}   But the reviewer is suggesting that I forget that actual data is involved in geophysical inversions, and I certainly will not do that. \medskip \\
The closest name for the procedure, to what should be used, is probably ``manufactured.''  That usage has problems here because my manufacturing procedure is nonstandard.  Standard ``manufacturing'' in Bueler et al.~(2005), for example, and in general numerical fluid dynamics literature like Wesseling (2001), uses a chosen solution formula to generate a new additive term in the model equations.  Here we instead generate a \emph{coefficient} in the model equations, so as to balance them. \medskip \\
In any case the word ``contrived'' now appears more often in the revised paper.  It appears in the abstract and the text of the paper (near equation (14)).  ``Manufactured'' is used in one place also, where appropriate.  In my opinion these flags suffice to alert the reader that the variable ice hardness is important for construction of an exact solution, while not suggesting it comes from either observations or physical insight, by an inverse procedure or otherwise.}

\reply{\dots It, is however, not explicitly expressed what this somewhat artificial mathematical approach serves to the physically inclined glaciological modeler. To me, it seems clear: `Testing numerical schemes for their reliability and performance'. And the author takes this view in the second part of the paper, to test the performance of two numerical schemes, a shooting method-IBV approach and a TPBVP-approach. By doing this, he even finds a glaciological relevant result: Loss of numerical accuracy of both schemes when crossing the grounding line region because of the associated stiffness--`jump'.  This may give hints of understanding glaciologists' doubtful results in grounding line regions.}
{On this idea we agree!  Indeed 8 pages of this 22 page (review lengths) paper is devoted to using the newly-created exact solution in the ``testing numerical schemes for their reliability and performance'' role.  We can both hope that this ``may give hints'' about numerical results for grounding lines.}

\reply{There is one term ``Mass balance'', which is incorrectly used. The quantity $M$ is NOT a mass BALANCE, but an accumulation/ablation rate density, functioning as a climatological input quantity. I do not understand why glaciologists and other natural scientists (and here a mathematician) maintain this erroneous term for decades, which expresses simply the wrong thing.}
{Now the reviewer is tilting at windmills.  I stand by my usage of ``surface mass balance'' as utterly standard.  Many glaciologists, including the reviewer, express displeasure with language ``mass balance'', despite its widespread use.  I sympathize; we would all like it to be ``mass imbalance'' or something.  Too late. \medskip \\
I have added ``accumulation/ablation rate'' as an alternate name for the quantity $M$ when it first appears in an equation.}

\reply{I am not sure whether J. Glaciology is the adequate place for this paper, \dots}
{Because of the historical purpose mentioned above, and because some J.~Glaciol.~readers are interested in exact solutions of glacier dynamics equations for their own sake, but most especially because many readers\footnote{  Including the 29 authors of the MISMIP3d paper Pattyn et al.~(2013) which appeared in J.~Glaciol.} are interested in numerically-modelling marine grounding lines, this is a fully-appropriate J.~Glaciol.~paper when judged on content.}

\reply{\dots which to me looks like a worked out term paper in a first year graduate class of applied mathematics or mechanics.}
{I would prefer not to respond to this attempt at gratuitous insult. \medskip \\
A first year graduate student in applied mathematics should indeed be able to check that \bod found an exact solution to the plug flow, grounded glacier equations.  As a nonlinear ODE solution the result is not hard, though surprising (see Appendix A).  Yet dozens of researchers have spent hundreds of person-years and many millions of grant dollars trying to generate reliable numerical solutions to these same equations.  \bod's rigorous treatment in 1955 of a rapidly-sliding glacier dynamics model was lost early in the history of theoretical glaciology.  I am glad to have come across it.  I think it is worth reviving.}
\end{itemize}


\subsection*{Reviewer \#2 (R.~Gladstone)}  \begin{itemize}
\reply{The author provides an exact steady solution to a specific idealised marine ice sheet configuration by combining two previously established solutions. This combined solution is used to verify two numerical approaches. The paper is mostly clearly written and well laid out. The decisions about which material to move to appendices seem good to me, and the figures provide a concise and adequate visual representation of the ice sheet configuration and various solutions.}
{I appreciate this highly-accurate summary of the paper.  I believe it also shows that reviewer \#1 had adequate information so as to understand the major points of the paper.}

\reply{Regarding content, this marine ice sheet configuration has very contrived spatial patterns for ice hardness and for surface mass balance.}
{\bod's solution, remarkably, is an exact solution of equations that no one else seems to have ever solved exactly.  Though it is a physically-natural plug flow solution (i.e.~one with constant vertically-integrated longitudinal stress), in order to interpret it in the modern (i.e.~SSA) context, so as to make it relevant to a large readership, it requires a contrived ice hardness function.  This is true and I agree with the ``contrived'' assessment.  Based on specific suggestions by the reviewer (below), I have revised the text to clarify this construction as much as possible, and to clarify why this is no obstruction to verification use of the exact solution. \medskip \\
I simply disagree with the reviewer that it has a ``very contrived \dots surface mass balance''.  Indeed the formula $M=a(H-H_{ela})$ is the most widespread model for elevation-dependent surface mass balance, i.e.~with a lapse rate $a$ around an equilibrium line $H_{ela}$.  Part of the reason that this surface mass balance model is widespread is that people cite \citebod for this formula while skipping the harder ice dynamics material!}

\reply{\dots It provides an exact solution against which to compare numerical models, which could possibly replace the semi-analytic solution provided by Christian Schoof to a similar but simpler and less contrived problem for the purposes of numerical model verification.  Whether or not this solution gains wide use in model verification, this is an elegant exercise which goes some way towards shedding light on the nature of current problems with grounding line numerical modelling from a mathematical perspective.}
{I have nothing to add to this comment, which exactly describes my aspirations for the paper.  I suspect that the new exact solution is too weak to ``replace the semi-analytic solution \dots by \dots Schoof'', but at least it has a chance to be useful for numerical verification.}

\reply{The author would do well to add at least a brief paragraph discussing the validity of the basal drag law when extending the bodvardsson solution from a land terminating to a marine ice sheet.  Basal drag is given as a function of overburden pressure, whereas there are arguments in favour of using effective pressure rather than overburden pressure.}
{I should have seen this coming, and I have now added text to address it. \medskip \\
I understand that a linear relationship ``$\beta = c N$'' would be more acceptable, where $N$ is the effective pressure.  The basic reason that ``$\beta = k \rho g H$'' is also acceptable is that effective pressure itself also roughly scales with overburden pressure in the interior of ice sheets and under glaciers.  This idea is more commonly stated as ``subglacial water pressure scales with overburden pressure,'' in which case $P_o = \rho g H$ and $P=\lambda P_o$ and $N=P_o - P$ and $\beta = c N$ combine to give $\beta = c (1 - \lambda) \rho g H$, so $k=c(1-\lambda)$ gives $\beta = k \rho g H$.
\medskip \\
Of course this comment makes most sense away from marine ice sheet grounding lines.  Near grounding lines there may instead be an efficient connection to the ocean.  See the next point.}

\reply{\dots For a marine ice sheet a first approximation for effective pressure would be to assume a fully connected sub-glacial drainage system, i.e. effective pressure given by ice overburden pressure minus ocean pressure. Thus you could replace H with (h-z0) in eqn 5.}
{I am constrained by ``\bod's little theorem'' in Appendix A, that is, by exactly one known exact solution method for the grounded, steady SSA$+$mass continuity coupled equations.  And so, although I agree that having the effective pressure go to zero at the grounding line would be great, I don't see how to solve the resulting equations.  I'm stuck with a drop in basal shear stress at the grounding line.}

\reply{I've checked all the equations in the main text (except linearization about the exact solution) and all seem fine. I've read through the appendices and they seem clearly written, though I would question the choice of beta in appendix A for possible confusion with the sliding coefficient.}
{I am delighted to have such a careful review, though one might observe, ironically, that the reason I get to write this paper is that readers of \citebod have traditionally \emph{not} checked all the equations! \medskip \\
I have changed the names of abstract coefficients used in Appendix A, so as to avoid any confusion with the main text.}

\reply{I don't really have the right background to verify the section about linearization around the exact solution, in which the stiffness ratio is calculated, but I do have some questions about this section. Apologies if my questions seem naive, I am a little out of my depth here. You say that the poor convergence of previous studies in terms of near-grounding line solution can be explained by the high stiffness. Current thinking is that the step change in basal drag, and the associated rapid variations in stress and other properties near the grounding line, are the cause of the problem.}
{I think that both the reviewer and myself have played with numerical models of marine ice sheets, and I have to proceed with this discussion in such terms.  (He does have the right background for this!)  I claim furthermore, and essentially, that this ``current thinking'' is mostly wrong, though indeed jumps in basal drag confuse matters.\medskip \\
In particular, if you take an already steady-state numerical MISMIP model, with a well-resolved grounding line, and if you ``zero out'' the basal drag from the grounding line upstream a good distance (e.g.~by simply setting the sliding coefficient to zero, and having a smooth transition to nonzero values start substantially upstream), then all the standard difficulties and standard techniques (with which Gladstone is familiar) are needed at the then-evolving grounding line, even though there is \emph{no drop in basal drag there at all}.  That is, I claim that the transition from the grounded surface elevation formula ``$h=H+b$'' to the floating surface elevation formula ``$h=\omega H$'' is sufficient to cause all the familiar difficulties associated to marine grounding lines.  And that, in the context of the current paper, the stiffness contrast is a new way of seeing that there must be numerical consequences of the switch from grounded to floating.}

\reply{\dots Is the stiffness explanation really saying anything different? Given that stiffness is not a precisely defined mathematical property (although I accept that the stiffness ratio, which I have not come across before, is clearly defined), can it give us any new insight into the problem? In other words, is this useful in any practical way?}
{There is much literature by now, including by the reviewer, which attempts numerically to ``fix'' a ``difficulty at the grounding line''.  I suppose my question to the reviewer would be ``fix what feature in the continuum model and it continuum solutions?''  \medskip \\
That is, if the community of marine ice sheet modelers had not yet tried numerical solutions, but instead we had (god-like, and counter-factual) full understanding of the continuum model and all its continuum solutions, what feature of the continuum equations would alert us to try what numerical techniques to get a good numerical solution? \medskip \\
Of course the whole art of modeling marine ice sheets has not evolved that way.  Instead we have gone at the problem assuming that our existing (e.g.~from MacAyeal et al.~(1996)) ice-shelf-only numerical methods could do the job, discovered (e.g.~through MISMIP or otherwise) that these methods had trouble, and then tried various (e.g.~as in Gladstone et al.~(2010)) numerical techniques so as to make the numerical results match our intuitions and, especially, Schoof's semi-analytical (i.e.~by-hand approximate) solution. \medskip \\
So it was with surprise and then pleasure that, when I solved the steady marine ice sheet problem, in a case where the exact solution was known, by shooting with a mature numerical ODE solver (i.e.~LSODA) and asking for 12 digit accuracy, that I discovered that it did \emph{not} take tiny spatial steps near the grounding line.  What it did do was switch from a stiff method on the grounded side to a non-stiff method on the floating side. \medskip \\
Thus, in the most confident applied mathematical situation yet for marine ice sheets, with a known exact solution and a mature, grid-free numerical ODE solution method, this paper can say for the first time that the problem at the grounding line, from the point of view of the mathematical (not physical) difficulty of the equations, is \emph{stiffness} not smoothness.  I think that is progress.}

\reply{\dots How does stiffness differ from smoothness in this context, and what can modellers do about it?}
{The best known stiff-versus-not pair of otherwise comparable objects is the heat equation PDE (stiff) and the wave equation PDE (nonstiff).  In both cases if you start with smooth initial conditions then the solution is smooth for all time, whereas numerical methods make large errors, at least initially, for both equations if the initial condition is nonsmooth.  Nonetheless there is a \emph{huge} contrast in how these PDEs are addressed numerically.   Specifically, explicit methods for the heat equation have a much more restrictive time-stepping stability condition than explicit methods for the wave equation.  Said the other way, the heat and wave equation PDEs would be treated very similarly by numerical analysts if the only schemes we had were the implicit ones.  But that's not true; the heat equation wants implicit methods, and so much hard work follows, while the wave equation does not need them, and thus almost all numerics is explicit. \medskip \\
I must note before moving on that the geometry-evolution equation here is the \emph{coupled} pair of equations for mass continuity (1) and stress balance (2).  The ``right'' numerics for this pair is fundamentally unknown to numerical analysis.  For example, there is nothing approaching a convergence proof even for the most obvious numerical techniques.}

\reply{How easy would it be for you to create a plot of stiffness ratios for the same problem but with different sliding relations? In particular, if you replace [overburden pressure] with [overburden pressure minus ocean water pressure] does the step change in stiffness ratio go away? It might be interesting to see such a plot for the Schoof 2005 sliding law too (the Leguy 2014 sliding law is based on this).}
{I am deliberately sticking to numerical experiments in this paper for which the numerical error can be calculated by subtraction!  It is beyond the scope of this paper to try numerical experiments where only human intuition can tell if errors are being made. \medskip \\
That said, the reviewer's suggestion to try different numerical techniques to smooth out the stiffness jump makes sense.  And I as much as suggest it in the paper.  (``It is possible that [the success of Leguy's technique] can be explained as a reduction in stiffness contrast.'')  Before getting too optimistic about this, however, observe that Figure 8 shows a huge (one thousand times) stiffness contrast between upstream grounded ice and floating ice, so that just ``smoothing the jump'' near the grounding line does not mean one can ignor the consequences of stiffness. \medskip \\
There is another aspect to this which I also point out, by citing Higham and Trefethen (1993).  Namely that ``stiffness'' is an operational concept for numerical solving of differential equations, which is associated to a number of quantitative measures, of which the stiffness ratio is one, but which is well-quantified by no single measure.  Thus it is possible to overuse stiffness ratios.  I would not want the current paper to be ``find the numerical fiddle which most smooths the stiffness ratio'', for example.  I thought about that and did not go there.}

\reply{Line by line comments \smallskip \\
L18 ``perfectly'' should be just ``perfect''}
{Done.}

\reply{L32-35. This looks more like a conclusion than an introductory remark.}
{Agreed.  Most of the sentence claiming ``the most significant result'' is removed.}

\reply{Equation 10. Please clarify that this is essentially equation 2 following the assumptions made in the preceding paragraph. \smallskip \\
Equation 12. "combining these equations" $\to$ "substituting these equations into equation 1" or something like "combining these equations with the requirement for mass conservation"}
{Yes, this is a good point.  I have made the text between equations (10) and (12) much clearer on where things come from, even though this clarity is not present in the original (Bodvardsson, 1995).}

\reply{L105-112. This is basically what I figured out while trying to understand where eqns 10-12 came from. Perhaps it would make more sense to bring these arguments forward to where equations 10-12 are introduced?}
{Yes, agreed and done.  I have also kept a summary of the logic here in this part of the paper, as a way to transition into the interpretation of constant longitudinal stress and variable ice hardness.}

\reply{Paragraph following L112 (the line numbering seem to have been missed out in places). $T_x=0$ isnt just saying that ice hardness is variable, it is saying that ice hardness must vary in this particular way (as given by equation 14).  You say ``such variable hardness is physical.''  Well, yes, hardness is variable. But is there any physical justification for this particular pattern of hardness?}
{My explanation was poor, and I have tried again to write it properly.  It has been rewritten to emphasize that while non-constant ice hardness does arise from physical processes (esp.~temperature-dependence), and thus numerical models don't need added features so as to use this exact solution, the particular variability here is simply what is needed to have the solution solve the equations, specifically $T_x=0$.}

\reply{Top of page 9. Formulas $\to$ formulae}
{Done.}

\reply{Table 2. It would be helpful to indicate which of these quantities are prescribed in order to generate the exact solution, and which are a result of the solution, i.e. which are the user defined inputs (e.g. $H_0$, $L_0$) and which are outputs (e.g. $B(x_g)$).}
{Good point.  Changes have been made to Table 2 so as to clarify this.  Nonetheless a clean separation between inputs and outputs is impossible because, in determining the exact solution, rigid equations relate quantities which one might regard as ``inputs''.  For example, the maximum elevation $H$ and the equilibrium-line altitude $H_{ela}$ are related by the equation $H_0 = 1.5 H_{ela}$ which allows (13) to be a solution; see Appendix A.}

\reply{L137-140. Presumably these lines refer just to the numerical models? Presumably eqn 11 is used in generation of your exact solution? In which case it might be better to move this to the next section, ``Numerical Results''.}
{Another good point.  This text is moved to the numerical results section.}

\reply{L173 ``show'' $\to$ ``shown''}
{Done.}

\reply{References: I would recommend citing Schoof 2005 in addition to Leguy 2014 as this paper (along with a test implementation in Gagliardini 2007) shows the full derivation of the Leguy 2014 sliding relation. The Leguy 2014 sliding relation is just a small deviation from Schoof's work.}
{I have added a citation to Schoof (2005), which is appropriate.  But, as noted above and is well-known, it is not just an effective-pressure-dependent sliding law that is relevant to the abruptness of the basal shear stress transition at the grounding law (e.g.~a law like in Schoof (2005)), but also the way subglacial water pressure or effective pressure is modeled (e.g.~an equation like (14) in Leguy et al.~(2014)).  So I have rewritten this text to clarify that models of hydrology are relevant to smoother grounding line stresses.}

\end{itemize}


%\bigskip
%\small
%\bibliography{ice-bib}
%\bibliographystyle{siam}

\end{document}
