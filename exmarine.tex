\NeedsTeXFormat{LaTeX2e}
%\documentclass[twocolumn]{igs}
\documentclass[twocolumn,letterpaper]{igs}
%\documentclass[review]{igs}

\usepackage{igsnatbib}  % see igs2eguide.tex for example citation styles

% check if we are compiling under latex or pdflatex
\ifx\pdftexversion\undefined
  \usepackage[dvips]{graphicx}
\else
  \usepackage[pdftex]{graphicx}
\fi

\usepackage{amsmath,amssymb,xspace}
\newcommand{\onecol}[1]{\includegraphics[width=86mm]{#1}}
\newcommand{\twocol}[1]{\includegraphics[width=178mm]{#1}}

\newcommand{\url}{\texttt{}}

% macros
\newcommand{\bQ}{\mathbf{Q}}
\newcommand{\bq}{\hat{\mathbf{q}}}
\newcommand{\br}{\hat{\mathbf{r}}}
\newcommand{\bU}{\mathbf{U}}
\newcommand{\hatx}{\hat{\mathbf{x}}}
\newcommand{\bx}{\mathbf{x}}
\newcommand{\CC}{\mathbb{C}}
\newcommand{\Div}{\nabla\cdot}
\newcommand{\ddx}[1]{\frac{\partial #1}{\partial x}}
\newcommand{\ddy}[1]{\frac{\partial #1}{\partial y}}
\newcommand{\pp}[2]{\frac{\partial #1}{\partial #2}}
\newcommand{\ppt}[1]{\frac{\partial #1}{\partial t}}
\newcommand{\ppT}[1]{\frac{\partial #1}{\partial T}}
\newcommand{\ppx}[1]{\frac{\partial #1}{\partial x}}
\newcommand{\ppy}[1]{\frac{\partial #1}{\partial y}}
\newcommand{\ppz}[1]{\frac{\partial #1}{\partial z}}
\newcommand{\ppzz}[1]{\frac{\partial^2 #1}{\partial z^2}}
\newcommand{\eps}{\epsilon}
\newcommand{\grad}{\nabla}
\newcommand{\hh}{\hat h}
\newcommand{\ip}[2]{\left(#1,#2\right)}
\newcommand{\lam}{\lambda}
\newcommand{\lap}{\triangle}
\newcommand{\mtt}{\texttt}
\newcommand{\RR}{\mathbb{R}}
\newcommand{\vf}{\varphi}

\begin{document}

\title{Exact solutions for flowline marine ice sheets}

\abstract{FIXME}

\author{Ed Bueler}

\affiliation{Dept.~of Mathematics and Statistics, University of Alaska, Fairbanks AK 99775-6660, USA \\
E-mail: \emph{\texttt{elbueler\@@alaska.edu}}}


\maketitle

\subsection*{Introduction}  At its beginning, theoretical glaciological created two parabolic profiles as the shapes of steady flow-line ice sheets lying on flat beds, as in Figure \ref{fig:twoparabolas}.  One was the profile of an ice sheet with perfectly-plasticity \citep{Orowan,Nye52plastic} and the other the profile for a sliding (plug) flow investigated by \cite{Bodvardsson}.  These were the earliest precise ``global'' views of the free surface flows in glaciology.  They focussed on different aspects of the global problem, and they came to very different conclusions.  These authors found parabolas which exactly solved the nonlinear flow equations for geometry, but their solutions are infrequently used for modelling glacier and ice sheet flow in recent decades.\footnote{Draft date \emph{\today}.}

\begin{figure}[ht]
\onecol{twoparabolas}
\caption{The parabolas by Orowan and Nye (1949, 1952; dotted) and by Bodvardsson (1955; solid) for steady, flowline ice sheets on flat beds.  A dome thickness of $H_0=3000$ m and a total flowline distance of $L_0=500$ km (from dome to margin where $H=0$) are chosen for concreteness in this figure.} \label{fig:twoparabolas}
\end{figure}

The profiles, shown in Figure \ref{fig:twoparabolas}, are remarkably different curves in their descriptions of the dome (the highest point of the ice sheet) and margin shape (where the ice thickness goes to zero).   Up to scaling, one is of the form $x=1-y^2$ (Orowan-Nye perfect-plasticity)\nocite{Nye52plastic,Orowan} and the other is of the form $y=1-x^2$ (Bodvardsson plug flow).  The perfect-plasticity solution has a central peak at the highest point of the ice sheet, and a margin with unbounded surface gradient.  The plug flow solution has a smooth dome and a finite-slope, wedge-shaped margin.

This paper combines Bodvardsson's solution with the well-known exact solution for an ice shelf \citep{vanderVeen83,vanderVeen} to generate an exact solution for a flowline, steady marine ice sheet.  The result, shown in Figure \ref{fig:exactmarine}, is an exact solution of the steady equations for the rapidly-sliding marine ice sheet case addressed by the well-known paper \cite{SchoofMarine1} and by the MISMIP intercomparison \citep{MISMIP2012}.  This exact solution simultaneously solves the steady mass continuity equation and the so-called shallow shelf approximation (``SSA''; \citep{BBssasliding,MacAyeal,WeisGreveHutter}).  This sliding-ice stress balance was the model considered by most participants in MISMIP.

\begin{figure}[ht]
\onecol{exactmarine-geometry}
\caption{The geometry (solid) and velocity (dashed) of an exact solution of the simultaneous steady mass continuity and SSA stress balance equations on the domain shown, on both sides of the grounding line.} \label{fig:exactmarine}
\end{figure}


\subsection*{Continuum model}

The continuum model we consider is the steady-state form of the flowline, rapid-sliding model of \citep[equations (2.1)--(2.5)]{SchoofMarine1}.  This includes the mass continuity equation and the SSA stress balance equation.  We restrict to the linear sliding case, but with a reasonably general sliding coefficient.  We use the notation in Table \ref{tab:notation}.

The equations are then
\begin{gather}
(uH)_x - M = 0, \label{eq:steadymass} \\
T_x - \beta u - \rho g H h_x = 0, \label{eq:steadySSA} \\
T = 2 B H |u_x|^{\frac{1}{n}-1} u_x, \label{eq:Tstress} \\
h = \begin{cases} H+b, & \rho H \ge \rho_w (z_o - b) \\
                  \omega H + z_o, & \rho H < \rho_w (z_o - b) \end{cases}, \label{eq:surface}
\end{gather}
on an interval $x_a < x < x_c$.  Here $M = M(h)$ is the surface plus basal mass balance, which may depend on surface elevation, $B = B(x)$ is the ice hardness, which may on location, $\beta = \beta(x,H)$ is the sliding coefficient, which may on location and thickness.  Note $z=z_o$ is the elevation of the sea surface, and also $\omega = 1 - \rho/\rho_w$ is a frequently-appearing ``Archimedean factor'' relating ice surface elevation and thickness for floating ice.  

Equations \eqref{eq:steadymass} and \eqref{eq:steadySSA} are the mass-continuity and SSA stress balance equations, respectively, where \eqref{eq:Tstress} defines the vertically-integrated longitudinal stress $T$ \citep{SchoofStream}.  Equation \eqref{eq:surface} says that the ice surface $z=h$ is at elevation $H$ (the ice thickness) plus the bedrock elevation $z=b$ when the ice is grounded, and otherwise the ice surface $z=h$ is found from the Archimedean principle.

For the exact and numerical solutions in this paper, equations \eqref{eq:steadymass}--\eqref{eq:surface} are augmented by boundary conditions:
\begin{gather}
u(x_a) = u_a > 0, \quad H(x_a) = H_a > 0, \label{eq:leftbc} \\
T(x_c) = \frac{1}{2} \omega \rho g H(x_c)^2,  \label{eq:rightbc} \\
H, u, u_x \quad \text{ continuous at } x = x_g.  \label{eq:xgregularity}
\end{gather}
Thus $x_a$ is an upstream location where a Dirichlet boundary condition is applied to the (coupled) problem of determining velocity and thickness, while at the (non-moving) calving front $x_c$ we have the standard hydrostatic pressure ``imbalance'' condition \citep{SchoofMarine1}.  In between, at $x_g$, fact \eqref{eq:xgregularity} is a regularity requirement giving matching conditions between the grounded and floating parts.

Given data $M$, $B$, $\beta$, and of course physical constants $g,n,\rho,\rho_w$, we expect the problem consisting of equations \eqref{eq:steadymass}--\eqref{eq:xgregularity} to be well-posed. 

\subsection*{Bodvardsson's parabola and marine ice sheets}

FIXME

\subsubsection*{An exact solution to the SSA stress balance plus mass continuity}
The \cite{Bodvardsson} solution is also a parabola, but this time the margin has finite slope and the dome is smooth.  The \cite{Bodvardsson} solution was analyzed by \cite{Weertman61stability}, who decides that the physics chosen by Bodvardsson should be replaced by more SIA-like deformation, though sliding is allowed as well.

Bodvardsson doesn't actual solve the SIA but instead solves a plug-flow model with a basal velocity which is proportional to the glaciological driving stress.  The SIA with Glen (power) law rheology is stated in his equation (13).  The ice softness is parameterized in equation (1) by $S_0$, the ``pseudo yield stress'' for power law ice.  Thus his equation (13) has constants $n$ and $S_0$, as well as constant $k$ describing the basal sliding.  But he then says on page 4, regarding his equation (13): ``This equation is quite tedious and very difficult to handle especially because of the first term in the parentheses.  It is therefore fortunate that this term appears to be small compared to the second term.''  In the next paragraph he drops the ``first term in the parentheses,'' and thus drops the Glen law and any internal ice deformation at all, and writes down equation (14) with only the constant ``$k$'' describing the base.  He never goes back to SIA equation (13), nor are any solutions of equation (13) provided.

Regarding the ``$k$'' in equation, and noting that his actual results are all on a flat bed, he says between his equation (9) and (10): ``$s_b = k v w h$''.  In modern symbols we would write the shear stress applied to the base of the ice as
  $$\tau_b = - k u_b \rho g H$$
This a linear sliding relation $\tau_b = - \beta u_b$ where $\beta = k (\rho g H).$.  That is, $\beta$ scales with the overburden pressure.

But equations (14) and (16) and (17), for which he discusses stability, are not SIA.  The involve no ice deformation at all, nor are stresses in the ice balanced in any way.  His flows are \emph{plug flows} with a weak fluid which feels only linear basal resistance.

Bodvardsson's solution (18) of (17) is exact.

\emph{But} the quadratic solution (18) absolutely does not have a margin
shape which would come from deformation and Glen's law.  Rather his
margin is asymptotically a ``wedge'' with finite slope, because his ice
sheet profile is merely a parabola \emph{in the horizontal variable} $x$.
See the profile in figure 2 for the margin shape, the last 20km of
Bruarjokull.

\small
\begin{table}
\caption{Notation and SI units, with values for physical constants.}\label{tab:notation}

\medskip
\begin{tabular}{llll}
Symbol & Description & Units \\ \hline
$B$ & ice hardness; $=A^{-1/n}$ & $\text{Pa}\,\text{s}^{1/3}$  \\
$b$ & bedrock elevation & m \\
$\beta$ & sliding coefficient & $\text{Pa}\,\text{s}\,\text{m}^{-1}$ \\
$g$ & acceleration of gravity  & 9.81 $\text{m}\,\text{s}^{-2}$\\
$H$ & ice thickness & m \\
$h$ & ice surface elevation & m \\
$M$ & mass balance & $\text{m}\,\text{s}^{-1}$ \\
$n$ & Glen exponent in ice flow law & 3 \\
$\rho$ & density of ice & 910 $\text{kg}\,\text{m}^{-3}$ \\
$\rho_w$ & density of sea water & 1028 $\text{kg}\,\text{m}^{-3}$ \\
$T$ & vertically-integrated longitudinal stress & $\text{Pa}\,\text{m}$ \\
$\tau_{b}$ & basal shear stress applied to ice & Pa \\
$u$ & horizontal velocity & $\text{m}\,\text{s}^{-1}$ \\
$(x,z)$ & flow-line cartesian coordinates & m  \\
$z_o$ & ocean surface elevation & m \\
$\omega$ & Archimedean factor; $=1 - \rho/\rho_w$ & $0.115$
\end{tabular}
\end{table}

Bodvardsson's parabola must be the opposite end of some spectrum from the
purely-plastic, $n=\infty$ ice which gives a shape which is a parabola
on its side, that is, a square root in $x$!  In particular, Bodvardsson's equation (23) gives an exact solution of his equation (17), and this is shown in Figure \ref{fig:twoparabolas}.  FIXME: form with $L$ and $H_0$ needed for reference from Figure \ref{fig:twoparabolas}.

\citet[equation (17)]{Bodvardsson} writes 
\begin{equation}
a (H - H_{ela}) + (k^{-1} H H')' = 0  \label{bodsteady}
\end{equation}
for $k>0$ and $a>0$ and equilibrium-line altitude $H_{ela}$.  The offered solution \citep[equation (23)]{Bodvardsson} is
\begin{equation}
H(x) = H_0 (1 - (x/L_0)^2)  \label{bodsoln}
\end{equation}
where $H_0 = 1.5 H_{ela}$ and $a k L_0^2 = 9 H_{ela}$.  This is an exact solution to equation \eqref{bodsteady} with boundary condition $H(0)=H_0$, as the reader may verify.  Formula \eqref{bodsoln} defines the dashed parabola shown in Figure \ref{fig:twoparabolas}.

In the Bodvardsson theory the ice velocity is
\begin{equation}
u = - \frac{1}{k} H' \label{bodstressbalance}
\end{equation}
and the surface mass balance is
\begin{equation}
M = a (H - H_{ela}).  \label{bodmassbalance}
\end{equation}
This same surface balance appears in \cite{Weertman61stability,Jouvetetal2011}, among other places, and, though realisting in its broad outlines, it is generally understood to be a source of instability for the problem of determining the steady state of an ice sheet.

Bodvardsson explicitly considers solution \eqref{bodsoln} to solve a free boundary problem.  Specifically, the sense is that the single boundary condition $H(0)=H_0$, in the equation \eqref{bodsteady}, determines the quadratic solution \eqref{bodsoln}, which generates a value for $L_0$ from the hypothesis that the flux at $x=L_0$ is zero and the thickness at $x=L_0$ is also zero.  A DEEPER REASON WHY \eqref{bodsteady} SHOULD HAVE A QUADRATIC SOLUTION IS NOT OFFERED.

A MODERN RE-INTERPETATION  The solution \eqref{bodsoln} exactly solves a combination of the flow-line mass-continuity equation and the SSA stress balance \citep{MacAyeal,WeisGreveHutter}, but with somewhat-contrived (``manufactured'') sliding coefficient and ice softness.

We write flow-line mass-continuity equation and SSA stress balance equations together as a system describing a steady grounded flow with sliding.  The SSA is a balance of vertically-integrated longitudinal stresses, basal shear stress, and glaciological driving stress.  Specifically we assume flat bed and linear sliding $\tau_b = -\beta u$ and get:
\begin{align}
M - (u H)_x &= 0 \label{mcsteady} \\
\left(2 H B |u_x|^{(1/n)-1} u_x\right)_x - \beta u &= \rho g H H_x  \label{ssasteady}
\end{align}
Such a model will be familiar to many readers as the steady state form of the model used in much recent work on marine ice sheet flowline models \citep[e.g.][]{SchoofMarine1}.  In equations \eqref{mcsteady} and \eqref{ssasteady} we see a non-linearly viscous constitutive relation for the ice.  A balance is required in which only the along-flow deviatoric (``longitudinal'') stress
	$$\tau = 2 \nu u_x$$
appears, but not the vertical shear stress.  In this isothermal case, Glen's power law for ice flow implies
	$$\nu = B |u_x|^{(1/n)-1}$$
where $B$ is the ice hardness.  Following \cite{SchoofStream} we define the vertically-integrated longitudinal stress
\begin{equation}
T = \tau H = 2 H B |u_x|^{(1/n)-1} u_x.  \label{schoofsT}
\end{equation}

We now consider equations \eqref{mcsteady}, \eqref{ssasteady} in the case of specific mass balance \eqref{bodmassbalance}.
Furthermore let us suppose a sliding coefficient which scales with the overburden pressure,
\begin{equation}
\beta = \beta(x) = k \rho g H(x),\label{betadefine}
\end{equation}
where $k>0$ is a constant.  With these substitutions the system \eqref{ssasteady} appears as
\begin{align}
a(H-H_{ela}) - (u H)_x &= 0 \label{mcsteadysimpler} \\
\left(T\right)_x - k \rho g H u &= \rho g H H_x \label{ssasteadysimpler}
\end{align}

Now suppose that we consider only solutions with constant vertically-integrated longitudinal stress, $T_x \equiv 0$.  In that case a brief calculation reduces equations \eqref{ssasteadysimpler} to Bodvardsson's model, equation \eqref{bodsteady} above.  That is, if $T_x \equiv 0$ then we recover a parabolic profile for $H(x)$, namely equation \eqref{bodsoln}.

What does the condition ``$T_x\equiv 0$'' imply as a relation among the modeled quantities?  Given a thickness profile $H(x)$ and a strain rate profile $u_x(x)$, we may interpret ``$T_x\equiv 0$'' as a statement about \emph{variable ice hardness}.  Such variable hardness is physical and common to many models using the SSA \citep{BBssasliding}, so implementing such $x$-dependent hardness requires no new concepts or components in such models.  In particular, if $T=T_0$ is the constant value then equation \eqref{schoofsT} implies
\begin{equation}
B = B(x) = \frac{T_0}{2 H |u_x|^{(1/n)-1} u_x}. \label{hardnessdefine}
\end{equation}

We are now in the following extraordinary situation.  Given Bodvardsson's formula \eqref{bodsoln} for $H(x)$, and computing the horizontal velocity $u(x)$ by the simple rule from \cite{Bodvardsson},
	$$u = -\frac{1}{k} H',$$
we can construct a nontrivial solution to the steady SSA-plus-mass continuity system \eqref{ssasteady}.  Namely, we define the linear friction coefficient $\beta$ by equation \eqref{betadefine} and the ice hardness $B$ by equation \eqref{hardnessdefine}, and regard these as the fixed functions of $x$ shown in Figure \ref{fig:bodverifbetaB}.  The resulting profiles of elevation and horizontal velocity solve a nontrivial problem with a flat base and a surface mass balance function depending linearly on elevation, namely system \eqref{ssasteadysimpler}.  These profiles are shown in Figure \ref{fig:bodverifthickvel}.

The surface mass balance is, in this Test N case, a fixed function of $x$ also, in which we have incorporated solution \eqref{bodsoln},
	$$M = M(x) = a(H-H_{ela}) = a(H_0(1 - (x/L_0)^2 - H_{ela}).$$
By using this simple parabolic surface mass balance we remove one instability from the continuum problem, namely the accumulation-altitude feedback which was Bodavardsson's original purpose.  But we can put it back when needed.

\begin{figure}[ht]
\onecol{bodverifbetaB}
\caption{Sliding law parameter $\beta(x)$, equation \eqref{betadefine}, and ice hardness $B(x)$, equation \eqref{hardnessdefine}, for the grounded (Bodvarssson) part of the exact solution .} \label{fig:bodverifbetaB}
\end{figure}


\subsubsection*{Stability}
WHAT IS Bodvardsson's ACTUAL STABILITY RESULT?:  Bodvardsson treats the stability problem for equation (16) by showing that his free boundary solution (18) is linearly-UNstable.

FOR VERIFICATION PURPOSES WE CAN FIX $M(x)$ TO ITS BODVARDSSON VALUE, AND REMOVE ANY ISSUES WITH INSTABILITY.

FOR THE SOLUTION FROM THE PREVIOUS SUBSECTION WE HAVE BOTH THE ELEVATION-ALTITUDE AND THE MARINE ICE SHEET INSTABILITY, IF WE ADD TIME-DEPENDENCE.



\subsection*{Numerical results}

This re-interpreted Bodvardsson exact solution is a verification test (``Test N'') in the open source code of the Parallel Ice Sheet Model \citep[PISM; ][]{pism-user-manual}.  A special-purpose, stand-alone C code implementing a second-order finite difference scheme is described in the appendix.  It is also available as part of the PISM source code.  It has a parallel implementation using the PETSc \citep{petsc-user-ref} library and it uses a Newton scheme to achieve observed quadratic convergence of the iteration.

Figure \ref{fig:verifNresult} shows the result of using this numerical scheme to approximate the exact solution.  Both the maximum numerical thickness error and the maximum numerical velocity error are observed to converge at essentially an $O(\Delta x^2)$ error as the grid spacing $\Delta x$ refines to zero.  

\begin{figure}[ht]
% add symbolic link to
%    pism-dev/src/verif/tests/verifN/verifN.pdf
\onecol{verifN}
\caption{Verification result from Test N.  The ice thickness error decays at rate $O(\Delta x^{1.95})$ (dotted).} \label{fig:verifNresult}
\end{figure}


\subsection*{Conclusions} BLAH BLAH

\subsection*{Acknowledgements and erratum}  Thanks to Heinz Blatter and Helgi Bj\"ornsson for tracking down the original paper by Bodvardsson.  \cite{BLKCB} identify the constant accumulation SIA solution as ``Bodvarsson (1955)--Vialov (1958).''  This is incorrect as it is attributable only to Vialov.  Also, the spelling of Bodvardsson's name is ``Bodvarsson'' in many places, but the original paper in 1955 has ``Bodvardsson'' with a ``d''.  \cite{Weertman61stability}, an early response to Bodvardsson's work, drops the ``d''.


%         References
\bibliography{ice_bib}
\bibliographystyle{igs}

\appendix

\subsection{Appendix A: Bodvardsson's little theorem}  \cite{Bodvardsson} does not identify the source of the exact parabolic thickness solution to his plug flow equations, and it seems likely that he derived the solution from scratch.  We summarize his result as the theorem that if $A$ and $B$ are constant then there is a unique polynomial solution $y(x)$ to the nonlinear second-order differential equation
\begin{equation}
  (y y')' = Ay+B  \label{eq:abstractode}
\end{equation}
satisfying the single boundary condition $y(0) = y_0 > 0$ and additional technical conditions $y'(0) \le 0$ and $2A y_0 + 3 B \ge 0$.  Equation \eqref{eq:abstractode} is Bodvarssson's equation ``(17).''  The additional technical conditions are unstated other than the claim that his problem ``define[s] one and only one solution which is admittable from the physical point of view'' \citep{Bodvardsson}.  Here we justify the theorem and find the relations among parameters for the solution.

There is no claim this nonlinear problem is well-posed.  Rather, the existence of the unique polynomial solution allows an interpretation of the problem as a free boundary problem which determines the first positive zero $x_0>0$ of $y(x)$.  In Bodvardsson's context $x_0=L_0$ is the length of the glacier, equivalently the location of the ice sheet margin in the ``dry'' case where flotation does not occur.

It is easy to see by substitution into \eqref{eq:abstractode} that nontrivial solutions of degree $d$, i.e.~of the form $y(x) = \gamma x^d + (\text{lower degree})$ with $\gamma\ne 0$, exist only if $d=2$.  In that case we seek solutions which satisfy the boundary conditions, so
\begin{equation}
y(x) = y_0(1 - \alpha x + \beta x^2)  \label{eq:abstractsoln}
\end{equation}
for some $\alpha\ge 0$ and $\beta$ which are to be determined from $A,B$; this is Bodvarssson's solution ``(18).''  Substitution gives the two equations
\begin{equation}
3 y_0^2 \alpha^2 = 2 A y_0 + 3 B \quad \text{ and } \quad 6 y_0 \beta = A.  \label{eq:abstractrelations}
\end{equation}
These two relations determine $\alpha,\beta$ from $A,B$.  The first relation explains a technical condition as ensuring that $\alpha$ is real.

In the main text, Bodvardsson's problem relates the initial (upstream) ice thickness $y_0=H_0$, an ablation gradient $a>0$, the equilibrium-line altitude $H_{ela}$, and a sliding constant $k>0$ to the glacier thickness $y(x) = H(x)$.  He has $A=-ka$ and $B=k a H_{ela}$ in \eqref{eq:abstractode} so the technical condition $2A y_0 + 3 B \ge 0$ says $3 H_{ela} \ge 2 H_0$ after simplification.  This causes the equilibrium line altitude to be relatively high on the glacier.


\subsection{Appendix: Finite difference scheme for coupled Equations \eqref{mcsteady} and \eqref{ssasteady}}  The coupled, steady-state equations for mass continuity and the SSA stress balance form a continuum system that is solved exactly by the Bodvardsson solution (Test N), as explained in the main text.  Here we detail a numerical scheme for these equations.  The scheme uses centered, second-order finite differences for the SSA equation and a second-order upwind finite difference scheme for the mass continuity equation.  Figure \ref{fig:verifNresult} gives clear evidence that the scheme computes convergent ice thickness and ice velocity as the grid spacing decreases.

Let $N$ be the number of equal-length subintervals of the domain $[0,x_c]$, where $x_c$ is the horizontal location of the calving front.  Let $\Delta x = x_c/N$ and $x_i = i \Delta x$ for $i=0,1,\dots,N$.  % note for code:  N = Mx-1
Denote the numerical approximations $H_i\approx H(x_i)$ and $u_i \approx u(x_i)$.

The mass continuity equation \eqref{mcsteady} is approximated by second-order upwinding.  In our application the velocity is positive ($u_i\ge 0$), thus
\begin{equation}
M_i - \frac{3 u_i H_i - 4 u_{i-1} H_{i-1} + u_{i-2} H_{i-2}}{2 \Delta x} = 0 \label{mcsteadyFD}
\end{equation}
approximates \eqref{mcsteady} for $i=2,\dots,N$.

The Dirichlet boundary conditions in Test N are $H(0)=H_0$ and $u(0)=u_0=0$.  Thus the flux $uH$ is zero at $x=0$.  The mass continuity equation \eqref{mcsteady} additionally implies $(uH)'\big|_{x=0}=M(0)=M_0$.  To build a special second-order rule for the $i=1$ case (i.e.~at $x_1 = \Delta x$) from this information, we construct interpolating quadratic $Q(x) = c_0 + c_1 x + c_2 x^2$ which approximates the flux, $Q(x) \approx u(x)H(x)$, on $0 \le x \le x_1$.  Specifically, we require $Q(0)=0$, $Q'(0)=M_0$, and $Q(x_1) = u_1 H_1$.  Then $Q'(x_1) = M_0 - (2 u_1 H_1/  \Delta x)$ so
\begin{equation}
M_1 - M_0 + \frac{2 u_1 H_1}{\Delta x} = 0 \label{mcbcFD}
\end{equation}
is the second-order upwind condition at $x_1$, as equation \eqref{mcsteadyFD} has no $i=1$ case.

In order to solve the SSA stress balance equation \eqref{ssasteady} we avoid infinite viscosity by choosing a regularization constant $\eps=x_c^{-1}$ per year, a strain rate of 1 m$/$a over the full horizontal span $x_c$.  Let $q = (1-n)/n$, and define
\begin{equation}
F(u_l,u_r) = \left(\left(\frac{u_r-u_l}{\Delta x}\right)^2 + \eps^2\right)^{q/2} \frac{u_r-u_l}{\Delta x}.  \label{viscregFD}
\end{equation}

Also let $H_{i+1/2} = (H_i+H_{i+1})/2$.  With this notation the SSA stress balance equation in \eqref{ssasteady} is approximated by this centered-difference, second-order scheme
\begin{align}
&\frac{2 H_{i+1/2} B_{i+1/2} F(u_i,u_{i+1}) - 2 H_{i-1/2} B_{i-1/2} F(u_{i-1},u_i)}{\Delta x} \notag \\
&\qquad - \beta_i u_i - \rho g H_i \frac{H_{i+1}-H_{i-1}}{2\Delta x} = 0  \label{SSAFD}
\end{align}
for $i=1,\dots,N-1$.

At $x=x_c$ the calving-front stress balance condition is applied, so the vertically-integrated longitudinal stress $T$ (equation \eqref{schoofsT}) has a specific value
\begin{align}
T(x_c) &= \frac{1}{2} \omega \rho g H_N^2. \label{calvingstressFD}
\end{align}
The resulting boundary condition is handled as are many such Neumann boundary conditions in finite difference and finite volume schemes, namely by introducing and then removing a notional unknown $u_{N+1}$ at a notional node $x_{N+1}$ which is a distance $\Delta x$ beyond $x_N=x_c$ \citep{MortonMayers}.  Recalling equation \eqref{schoofsT}, we approximate $T(x_c)$ by the average
\begin{align}
T(x_c) &\approx H_{N+1/2} B_{N+1/2} F(u_N,u_{N+1}) \notag \\
  &\qquad+ H_{N-1/2} B_{N-1/2} F(u_{N-1},u_{N}),\label{notionalFD}
\end{align}
which also involves a notional thickness $H_{N+1}$.  Combining equations \eqref{calvingstressFD}, \eqref{notionalFD}, and the $i=N$ case of \eqref{SSAFD} allows us to eliminate all reference to the notional unknowns $u_{N+1}$ and $H_{N+1}$.  We derive this special replacement for \eqref{SSAFD} at $x_N$:
\begin{align*}
&\frac{\omega \rho g H_N^2 - 4 H_{N-1/2} B_{N-1/2} F(u_{N-1},u_N)}{\Delta x} \\
&\qquad - \beta_N u_N - \rho g H_N \frac{3 H_N - 4H_{N-1} + H_{N-2}}{2\Delta x} = 0.
\end{align*}
Note that for the driving stress we have used a second-order one-sided approximation of the surface slope $H'$.

In equations \eqref{mcsteadyFD} and \eqref{SSAFD} the $2N+2$ unknowns are the $N+1$ approximate thickness values $\{H_i\}_{i=0}^N$ and the $N+1$ approximate velocities $\{u_i\}_{i=0}^N$.  The additional parameters (knowns) are the surface mass balance values $M_i = M(x_i)$, the basal sliding coefficients $\beta_i = \beta(x_i)$, and the staggered-grid values of the ice hardness $B_{i+1/2} =B(x_{i+1/2})$ where $x_{i+1/2} = x_i + \Delta x/2$.  Equations \eqref{mcsteadyFD} and \eqref{SSAFD} form a presumably-well-posed system when augmented by the given boundary conditions.

Equations \eqref{mcsteadyFD}, \eqref{mcbcFD}, \eqref{SSAFD}, and \eqref{notionalFD}, along with Dirichlet boundary conditions $H_0=H_0$ and $u_0=0$ form a system of $2N+2$ equations $\mathbf{F}(\mathbf{v})=0$ in the same number of unknowns, which we denote as a vector $\mathbf{v}$.  These equations are solved by Newton's method \citep[e.g.][]{Pressetal}.  Following the paradigm sketched by \cite{KnollKeyes2004} we first implement a residual evaluation function which merely computes $\mathbf{F}(\mathbf{v}_0)$ given a guess $\mathbf{v}_0$.  The Jacobian-free Newton-Krylov method implemented in PETSc \citep{KnollKeyes2004,petsc-user-ref}, and/or the finite difference Jacobian also in PETSc, can solve the system up to sizes of about $N=1000$.  For scaling to higher resolution as shown in Figure \ref{fig:verifNresult} WE USED AN ANALYTICAL JACOBIAN.  FOR 2D CASES PERHAPS PRECONDITIONING BY AN APPROXIMATION TO THE JACOBIAN WOULD BE WISE.  FIXME: WHAT ABOUT INITIAL GUESS?

\end{document}
