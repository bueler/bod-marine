\NeedsTeXFormat{LaTeX2e}
%\documentclass[twocolumn]{igs}
\documentclass[twocolumn,letterpaper]{igs}
%\documentclass[review]{igs}

\usepackage{igsnatbib}  % see igs2eguide.tex for example citation styles

% check if we are compiling under latex or pdflatex
\ifx\pdftexversion\undefined
  \usepackage[dvips]{graphicx}
\else
  \usepackage[pdftex]{graphicx}
\fi

\usepackage{amsmath,amssymb,xspace}
\newcommand{\onecol}[1]{\includegraphics[width=86mm]{#1}}
\newcommand{\twocol}[1]{\includegraphics[width=178mm]{#1}}

\newcommand{\url}{\texttt{}}

% macros
\newcommand{\bQ}{\mathbf{Q}}
\newcommand{\bq}{\hat{\mathbf{q}}}
\newcommand{\br}{\hat{\mathbf{r}}}
\newcommand{\bU}{\mathbf{U}}
\newcommand{\hatx}{\hat{\mathbf{x}}}
\newcommand{\bx}{\mathbf{x}}
\newcommand{\CC}{\mathbb{C}}
\newcommand{\Div}{\nabla\cdot}
\newcommand{\ddx}[1]{\frac{\partial #1}{\partial x}}
\newcommand{\ddy}[1]{\frac{\partial #1}{\partial y}}
\newcommand{\pp}[2]{\frac{\partial #1}{\partial #2}}
\newcommand{\ppt}[1]{\frac{\partial #1}{\partial t}}
\newcommand{\ppT}[1]{\frac{\partial #1}{\partial T}}
\newcommand{\ppx}[1]{\frac{\partial #1}{\partial x}}
\newcommand{\ppy}[1]{\frac{\partial #1}{\partial y}}
\newcommand{\ppz}[1]{\frac{\partial #1}{\partial z}}
\newcommand{\ppzz}[1]{\frac{\partial^2 #1}{\partial z^2}}
\newcommand{\eps}{\epsilon}
\newcommand{\grad}{\nabla}
\newcommand{\hh}{\hat h}
\newcommand{\ip}[2]{\left(#1,#2\right)}
\newcommand{\lam}{\lambda}
\newcommand{\lap}{\triangle}
\newcommand{\mtt}{\texttt}
\newcommand{\RR}{\mathbb{R}}
\newcommand{\vf}{\varphi}

\begin{document}

\title{An exact solution for a steady, flow-line marine ice sheet}

\abstract{In 1955 G.~Bodvardsson published a flow-line exact solution which, though this seems not to have been recognized before, is an exact solution to the simultaneous stress balance and mass continuity equations widely-used in numerical models of marine ice sheets.  Bodvardsson's solution, which has parabolic ice thickness and constant vertically-integrated longitudinal stress, solves the steady shallow shelf approximation on a flat bed.  It has elevation-dependent mass balance and location-dependent ice hardness.  By connecting Bodvardsson's solution to the \cite{vanderVeen83} solution for floating ice, we construct a steady solution to the ``rapid-sliding'' case of the marine ice sheet problem, essentially the model used in the MISMIP intercomparison \citep{MISMIP2012}.  We exploit the exact solution to examine the accuracy of a grid-free numerical method and a fully-implicit finite difference method, focussing especially on the approximation of the geometry and velocity in the vicinity of the grounding line.}

\author{Ed Bueler}

\affiliation{Department of Mathematics and Statistics and Geophysical Institute, University of Alaska Fairbanks, USA \\
E-mail: \emph{\texttt{elbueler\@@alaska.edu}}}

\maketitle

\subsection*{Introduction}  Early theoretical glaciology created two parabolic profiles as the shapes of steady flow-line ice sheets lying on flat beds, as in Figure \ref{fig:twoparabolas}.  One was the profile of an ice sheet with perfectly-plasticity \citep{Orowan,Nye52plastic} and the other the profile for a sliding ``plug'' flow investigated by \cite{Bodvardsson}.  These global views of free surface flows in glaciology focus on different aspects of the global problem, and they come to very different conclusions.\footnote{Draft date \emph{\today}.}

These parabolas, which exactly solve the nonlinear flow equations for geometry, shown in Figure \ref{fig:twoparabolas}.  Up to scaling, one is of the form $x=1-y^2$ (Orowan-Nye) and the other is of the form $y=1-x^2$ (Bodvardsson).  The former perfect-plasticity solution has a central peak at the highest point of the ice sheet, and a margin with unbounded surface gradient.  The plug flow solution has a smooth dome and a finite-slope, wedge-shaped margin.

\begin{figure}[ht]
\onecol{twoparabolas}
\caption{The parabolas by Orowan and Nye (1949, 1952; dotted) and by Bodvardsson (1955; solid) for steady, flowline ice sheets on flat beds.  A dome thickness of $H_0=3000$ m and a length of $L_0=500$ km are chosen for concreteness.} \label{fig:twoparabolas}
\end{figure}

This paper combines Bodvardsson's solution with the well-known exact solution for an ice shelf \citep{vanderVeen83,vanderVeen} to generate an exact solution for a flowline, steady marine ice sheet.  The result, shown in Figure \ref{fig:exactmarine}, is an exact solution of the steady equations for the rapidly-sliding marine ice sheet case \citep{SchoofMarine1} which is addressed by the MISMIP intercomparison \citep{MISMIP2012}.  This exact solution simultaneously solves the steady mass continuity equation and the so-called shallow shelf approximation (``SSA''; Weis and others, 1999).\nocite{WeisGreveHutter}

\subsection*{Continuum model}  Our continuum model is the steady-state, flat bed form of the flowline, rapid-sliding model of Schoof\nocite{SchoofMarine1} (2007; equations (2.1)--(2.5)).  We restrict to the linear sliding case, but with a reasonably general sliding coefficient.  The primary unknowns in these equations are the ice thickness $H(x)$, velocity $u(x)$, and vertically-integrated longitudinal stress $T(x)$ \citep{SchoofStream}, where $x$ is the flowline distance.  Using other notation from Table \ref{tab:notation}, the equations are
\begin{gather}
(uH)_x - M = 0, \label{eq:steadymass} \\
T_x - \beta u - \rho g H h_x = 0, \label{eq:steadySSA} \\
T = 2 B H |u_x|^{\frac{1}{n}-1} u_x, \label{eq:Tstress}
\end{gather}
where the subscript $x$ denotes the derivative and where
\begin{gather}
h = \begin{cases} H+b,            & \rho H \ge \rho_w (z_o - b) \\
                  \omega H + z_o, & \rho H < \rho_w (z_o - b) \end{cases}, \label{eq:surface} \\
\beta = \begin{cases} k \rho g H,    & \rho H \ge \rho_w (z_o - b) \\
                      0,          & \rho H < \rho_w (z_o - b) \end{cases}. \label{eq:betafull}
\end{gather}
These equations apply on an interval $0 < x < x_c$ where $x_c$ is the floating calving-front.  The grounded ice rests on flat bedrock at elevation $b$.  Note $M(x)$ combines the surface and basal mass balance and $B(x)$ is the ice hardness, and these may depend on location.  In grounded ice the basal shear stress satisfies $\tau_b = - \beta u$ \citep{MacAyeal}, but here we scale the coefficient with the ice overburden pressure so that $\beta = k \rho g H$ \citep{Bodvardsson}.  The ``Archimedean factor'' $\omega = 1 - \rho/\rho_w$ relates surface elevation and thickness in floating ice and $z=z_o$ is the elevation of the ocean surface.

\begin{figure}[ht]
\onecol{exactmarine-geometry}
\caption{The geometry (solid) and velocity (dashed) of an exact solution of the simultaneous steady mass continuity and SSA stress balance equations for a marine ice sheet.  It is Bodvardsson's parabola when grounded and van der Veen's solution when floating.} \label{fig:exactmarine}
\end{figure}

Equations \eqref{eq:steadymass} and \eqref{eq:steadySSA} are the mass-continuity and SSA stress balance equations, respectively, while \eqref{eq:Tstress} defines $T$.  The grounding line $x=x_g$ is a location which must be solved-for, but there we have $\rho H = \rho_w (z_o - b)$.  Equation \eqref{eq:surface} says that the ice surface $z=h$ is at elevation $H+b$ when the ice is grounded, and otherwise the ice surface $z=h$ is found from the Archimedean principle.  Likewise equation \eqref{eq:betafull} gives the scaled form of the basal shear stress for grounded ice and otherwise zero.

For the exact and numerical solutions in this paper, equations \eqref{eq:steadymass}--\eqref{eq:betafull} are augmented by boundary conditions:
\begin{gather}
u(0) = u_a > 0, \quad H(0) = H_a > 0, \label{eq:leftbc} \\
T(x_c) = \frac{1}{2} \omega \rho g H(x_c)^2,  \label{eq:rightbc} \\
H, u, T \quad \text{ continuous at } x = x_g.  \label{eq:xgregularity}
\end{gather}
Here $x=0$ is an upstream location where a Dirichlet boundary condition is applied to the (coupled) problem of determining velocity and thickness, while at the (non-moving) calving front $x_c$ we have the standard hydrostatic pressure ``imbalance'' condition \citep{SchoofMarine1}.  Fact \eqref{eq:xgregularity} at $x_g$ is actually a regularity requirement, not strictly a boundary condition; see below.  By equation \eqref{eq:Tstress}, $T$ is continuous at $x_g$ if and only if $u_x$ is continuous at $x_g$ \citep{SchoofMarine1}.

Given data $B(x)$, $b$, $k>0$, $M(x)$, $x_c>0$, $z_o$, along with physical constants $g,n,\rho,\rho_w$, we expect the problem consisting of equations \eqref{eq:steadymass}--\eqref{eq:xgregularity} to be well-posed.  To our knowledge this has not been proved, nor do we attempt to prove it.  It is, however, worth considering the smoothness (regularity) of the solution, including what hypotheses would lead to satisfying \eqref{eq:xgregularity}, noting that the grounding line is an unknown point inside the domain.  Suppose that, for \emph{a priori} physical reasons, the mass balance $M(x)$ and ice hardness $B(x)$ are bounded and integrable, and further that $B(x)$ is bounded below by a positive constant.  From integrating $M$, equation \eqref{eq:steadymass} implies that the flux $q=uH$ is absolutely-continuous and thus bounded.  If there is an \emph{a priori} positive lower bound on thickness $H$, a function which we must assume is integrable, then we can conclude that the magnitude of $u$ is bounded because $u=q/H$.  If there is also an \emph{a priori} bound on the magnitude of the driving stress $-\rho g H h_x$ then equation \eqref{eq:steadySSA} implies $T$ is absolutely-continuous.  By equation \eqref{eq:Tstress} this implies $u$ has a bounded and integrable derivative, and thus that it is also absolutely-continuous.  From these facts we could then return to the flux and write $H=q/u$ which shows $H$ is absolutely-continuous away from locations where $u=0$ (e.g.~divides).  In summary, assuming \emph{(i)} that an integrable solution $(H,u,T)$ to \eqref{eq:steadymass}--\eqref{eq:Tstress} exists, \emph{(ii)} that the functions $M,B$ are bounded and integrable, \emph{(iii)} that $B$ is bounded below by a positive constant, \emph{(iv)} that a positive lower bound on thickness exists, and \emph{(v)} that an upper bound on the magnitude of the driving stress exists, then we can regard condition \eqref{eq:xgregularity}, giving the continuity at the grounding line, as properties of the solution instead of as part of the ``imposed'' problem statement.

\small
\begin{table}
\caption{Notation and SI units, including physical constants.}\label{tab:notation}

\medskip
\begin{tabular}{llll}
Symbol & Description & Units \\ \hline
$B$ & ice hardness; $=A^{-1/n}$ & $\text{Pa}\,\text{s}^{1/3}$  \\
$b$ & bedrock elevation & m \\
$\beta$ & linear sliding coefficient; $=0$ if floating & $\text{Pa}\,\text{s}\,\text{m}^{-1}$ \\
$g$ & acceleration of gravity  & 9.81 $\text{m}\,\text{s}^{-2}$\\
$H$ & ice thickness & m \\
$h$ & ice surface elevation & m \\
$k$ & overburden-pressure-scaled coefficient  & $\text{s}\,\text{m}^{-1}$ \\
$M$ & mass balance & $\text{m}\,\text{s}^{-1}$ \\
$n$ & Glen exponent in ice flow law & 3 \\
$\rho$ & density of ice & 910 $\text{kg}\,\text{m}^{-3}$ \\
$\rho_w$ & density of sea water & 1028 $\text{kg}\,\text{m}^{-3}$ \\
$T$ & vertically-integrated longitudinal stress & $\text{Pa}\,\text{m}$ \\
$\tau_{b}$ & basal shear stress applied to ice & Pa \\
$u$ & horizontal velocity & $\text{m}\,\text{s}^{-1}$ \\
$(x,z)$ & flow-line cartesian coordinates & m  \\
$x_g$ & grounding line & m  \\
$x_c$ & calving front & m  \\
$z_o$ & ocean surface elevation & m \\
$\omega$ & Archimedean factor; $=1 - \rho/\rho_w$ & $0.115$
\end{tabular}
\end{table}

\subsection*{Bodvardsson's parabola}

The \cite{Bodvardsson} solution for a grounded glacier was apparently first cited by \cite{Weertman61stability}, who decided that the physics chosen by Bodvardsson should be replaced by more shallow ice approximation (``SIA'')-like shear deformation, though Weertman allowed sliding as well, and this seems to have influenced readers from then on.  Bodvardsson does not solve the SIA, however, and instead solves a plug-flow model with a basal velocity which is proportional to the glaciological driving stress.  He first states an equation having both shear and longitudinal stress but he says about this equation that it ``is quite tedious and very difficult to handle especially because of the [shear] term in the parentheses.  It is therefore fortunate that [the shear] term appears to be small compared to the [basal sliding] term.''  Then he drops the shear term and writes an equation in which driving stress is balanced entirely by basal resistance.

Describing this plug flow in his terms, he chooses the shear stress applied to the base of the ice to have a coefficient which scales with the overburden pressure
\begin{equation}
\tau_b = - k \rho g H u  \label{eq:bodstresschoice}
\end{equation}
He then writes that the ice flux is $uH=-(H/k) H_x$, or equivalently that the ice velocity is
\begin{equation}
u = - \frac{1}{k} H_x. \label{eq:bodvelocity}
\end{equation}
As is best-known about Bodvardsson's work, he chooses the surface mass balance to be
\begin{equation}
M = a (H - H_{ela})  \label{eq:bodmassbalance}
\end{equation}
for a mass balance gradient $a>0$ and equilibrium-line altitude $H_{ela}$.  Combining these equations he reports \citep[equation (17)]{Bodvardsson} that the thickness solves 
\begin{equation}
a (H - H_{ela}) + (k^{-1} H H_x)_x = 0.  \label{eq:bodsteady}
\end{equation}
The offered solution \citep[\emph{equivalent} equations (18) and (23)]{Bodvardsson} is
\begin{equation}
H(x) = H_0 (1 - (x/L_0)^2)  \label{eq:bodsoln}
\end{equation}
where $H_0 = 1.5 H_{ela}$ and $a k L_0^2 = 9 H_{ela}$ \citep[equation (24)]{Bodvardsson}.

Equation \eqref{eq:bodsoln} is an exact solution to equation \eqref{eq:bodsteady} with boundary condition $H(0)=H_0$, as the reader may verify!  Bodvardsson does not offer a reason why there should be such a simple quadratic solution, and his solution seems to be a not-well-known result for certain nonlinear second-order ODEs; see Appendix A.  Formula \eqref{eq:bodsoln} defines the solid parabola shown in Figure \ref{fig:twoparabolas}.

The surface balance parameterization \eqref{eq:bodmassbalance} reappears in \cite{Weertman61stability}, among many other places.  It realistically parameterizes a potential climatic instability, which was Bodvardsson's, Weertman's, and most readers', major interest.  We are, however, interested in Bodvardsson's solution of the ice flow equations themselves.

Bodvardsson explicitly considers solution \eqref{eq:bodsoln} to solve a free boundary problem.  Specifically, the sense is that the single boundary condition $H(0)=H_0$, in the equation \eqref{eq:bodsteady}, determines the quadratic solution \eqref{eq:bodsoln}, which generates a value for $L_0$ from the hypothesis that the flux and the thickness at $x=L_0$ are both zero.

\subsection{A re-interpretation}  We now claim that the solution \eqref{eq:bodsoln} exactly solves a combination of the steady flow-line mass-continuity equation \eqref{eq:steadymass} and SSA stress balance equation \eqref{eq:steadySSA}, but with contrived (``manufactured'') ice softness.

In fact, suppose that we consider only solutions with constant vertically-integrated longitudinal stress, $T_x \equiv 0$.  In that case equations \eqref{eq:Tstress} and \eqref{eq:xgregularity} imply formula \eqref{eq:bodvelocity}.  Then \eqref{eq:steadymass} and \eqref{eq:bodmassbalance} give Bodvardsson's main equation \eqref{eq:bodsteady}.  That is, if \emph{(i)} $T_x \equiv 0$, \emph{(ii)} sliding resistance is linear and scales with overburden pressure, and \emph{(iii)} mass balance is proportional to elevation above the equilibrium line, then we recover a simple parabolic profile for $H(x)$, namely equation \eqref{eq:bodsoln}.

But what does the condition ``$T_x\equiv 0$'' imply as a relation among the modeled quantities?  Given a thickness profile $H(x)$ and a strain rate profile $u_x(x)$, we may interpret ``$T_x\equiv 0$'' as a statement about \emph{variable ice hardness}.  Such variable hardness is physical and common to many models using the SSA \citep{BBssasliding}, so implementing such $x$-dependent hardness requires no new concepts or components in such models.  In particular, if $T=T_0$ is the constant value then equation \eqref{eq:Tstress} implies
\begin{equation}
B = B(x) = \frac{T_0}{2 H |u_x|^{(1/n)-1} u_x}. \label{hardnessdefine}
\end{equation}
The value $T_0$ is set by the calving front stress. FIXME: EXPLAIN THAT THIS COULD BE GROUNDED OR AT FLOTATION OR WITH AN ATTACHED SHELF

\subsection*{Extending the exact solution to floating ice}

\begin{figure}[ht]
\onecol{exactmarine-geometry-detail}
\caption{Detail of Figure \ref{fig:exactmarine}, showing the floating ice shelf geometry and velocity.} \label{fig:exactmarine-detail}
\end{figure}

\begin{figure}[ht]
\onecol{exactmarine-M-B}
\caption{The mass balance $M(x)$ (solid) and ice hardness $B(x)$ (dashed) of the exact solution.} \label{fig:exactMB}
\end{figure}

\begin{figure}[ht]
\onecol{exactmarine-beta}
\caption{The sliding coefficient $\beta$ for the exact solution.  Note that floating ice always experiences zero basal stress, as in equation \eqref{eq:betafull}; the solid curve shows $\beta_0(x)$ while the dashed is the actual basal resistance experienced by the shelf, which drops to zero at the grounding line.} \label{fig:exactbeta}
\end{figure}


WHAT IS Bodvardsson's ACTUAL STABILITY RESULT?:  Bodvardsson treats the stability problem for equation (16) by showing that his free boundary solution (18) is linearly-UNstable.

FOR VERIFICATION PURPOSES WE CAN FIX $M(x)$ TO ITS BODVARDSSON VALUE, AND REMOVE ANY ISSUES WITH INSTABILITY.

FOR THE SOLUTION FROM THE PREVIOUS SUBSECTION WE HAVE BOTH THE ELEVATION-ALTITUDE AND THE MARINE ICE SHEET INSTABILITY, IF WE ADD TIME-DEPENDENCE.



\subsection*{Numerical results}

FIXME:  When one does nonlinear shooting using the exact correct $T$ value, one gets the best-case adaptive-grid error in Figure \ref{fig:shoot-good-error}.

\begin{figure}[ht]
\onecol{exactmarine-good-error}
\caption{Pointwise error from an adaptive time-stepping numerical ODE scheme in the ``good'' case where we use the exactly-correct initial value for $T$.} \label{fig:shoot-good-error}
\end{figure}

\begin{figure}[ht]
%\onecol{exactmarine-shooting-error}
\caption{Pointwise error from an adaptive time-stepping numerical ODE scheme in the realistic nonlinear shooting method case} \label{fig:shoot-error}
\end{figure}

A special-purpose, stand-alone C code implementing a second-order finite difference scheme is described in the appendix.  It is also available as part of the PISM source code.  It has a parallel implementation using the PETSc \citep{petsc-user-ref} library and it uses a Newton scheme to achieve observed quadratic convergence of the iteration.

Figure \ref{fig:verifNresult} shows the result of using this numerical scheme to approximate the exact solution.  Both the maximum numerical thickness error and the maximum numerical velocity error are observed to converge at essentially an $O(\Delta x^2)$ error as the grid spacing $\Delta x$ refines to zero.  

\begin{figure}[ht]
\onecol{verifN}
\caption{Verification result for finite difference code on grounded-only problem.  The ice thickness error decays at rate $O(\Delta x^{1.97})$ (dotted).} \label{fig:verifNresult}
\end{figure}


\subsection*{Conclusions} BLAH BLAH

\subsection*{Acknowledgements and erratum}  Thanks to Heinz Blatter and Helgi Bj\"ornsson for tracking down the original paper by Bodvardsson.  \cite{BLKCB} identify the constant accumulation SIA solution as ``Bodvarsson (1955)--Vialov (1958).''  This is incorrect as it is attributable only to Vialov.  Also, the spelling of Bodvardsson's name is ``Bodvarsson'' in many places, but the original paper in 1955 has ``Bodvardsson'' with a ``d''.  \cite{Weertman61stability}, an early response to Bodvardsson's work, drops the ``d''.


%         References
\bibliography{ice-bib}
\bibliographystyle{igs}

\appendix

\subsection{Appendix A: Bodvardsson's little theorem}  \cite{Bodvardsson} does not identify the source of the exact parabolic thickness solution to his plug flow equations, and it seems likely that he derived the solution from scratch.  We summarize his result as the theorem that if $A$ and $B$ are constant then there is a unique polynomial solution $y(x)$ to the nonlinear second-order differential equation
\begin{equation}
  (y y')' = Ay+B  \label{eq:abstractode}
\end{equation}
satisfying the single boundary condition $y(0) = y_0 > 0$ and additional technical conditions $y'(0) \le 0$ and $2A y_0 + 3 B \ge 0$.  Equation \eqref{eq:abstractode} is Bodvarssson's equation ``(17).''  The additional technical conditions are unstated other than the claim that his problem ``define[s] one and only one solution which is admittable from the physical point of view'' \citep{Bodvardsson}.  In this appendix we justify this little theorem and, following Bodvardsson, we derive relations among parameters which yield the solution.  The existence of the unique polynomial solution to this problem allows its interpretation as a free boundary problem which determines the first positive zero $x_0>0$ of $y(x)$.  In Bodvardsson's context $x_0=L_0$ is the length of the glacier, equivalently the location of the ice sheet margin in the ``dry'' case where flotation does not occur.

It is easy to see by substitution into \eqref{eq:abstractode} that nontrivial solutions of degree $d$, i.e.~of the form $y(x) = \gamma x^d + (\text{lower degree})$ with $\gamma\ne 0$, exist only if $d=2$.  In that case we seek solutions which satisfy the boundary conditions, so
\begin{equation}
y(x) = y_0(1 - \alpha x + \beta x^2)  \label{eq:abstractsoln}
\end{equation}
for some $\alpha\ge 0$ and $\beta$ which are to be determined from $A,B$; this is Bodvarssson's solution ``(18).''  Substitution gives the two equations
\begin{equation}
3 y_0^2 \alpha^2 = 2 A y_0 + 3 B \quad \text{ and } \quad 6 y_0 \beta = A.  \label{eq:abstractrelations}
\end{equation}
These two relations determine $\alpha,\beta$ from $A,B$.  The first relation explains a technical condition as ensuring that $\alpha$ is real.

In the main text, Bodvardsson's problem relates the initial (upstream) ice thickness $y_0=H_0$, an ablation gradient $a>0$, the equilibrium-line altitude $H_{ela}$, and a sliding constant $k>0$ to the glacier thickness $y(x) = H(x)$.  He has $A=-ka$ and $B=k a H_{ela}$ in \eqref{eq:abstractode} so the technical condition $2A y_0 + 3 B \ge 0$ says $3 H_{ela} \ge 2 H_0$ after simplification.  This causes the equilibrium line altitude to be relatively high on the glacier.


\subsection{Appendix: a finite difference scheme}  The steady-state equations for mass continuity \eqref{eq:steadymass} and stress balance \eqref{eq:steadySSA} form a coupled system that can be approximately solved by the numerical scheme described here.  It uses centered, second-order finite differences for both equations.

We define an equally-spaced grid on the domain $[x_a,x_c]$.  Because the boundary condition at the calving front evaluates the stress $T$, the right endpoint $x_c$ is at a ``staggered'' location halfway in-between grid points.  Let $L=x_c-x_a$.  If $N$ is an integer then we define $\Delta x = L / (N+1/2)$ and $x_j = x_a + j\Delta x$ for $j=0,\dots,N+1$.  Denote the numerical approximations $H_i\approx H(x_i)$ and $u_i \approx u(x_i)$.  Let $x_j^* = x_j + \Delta x/2$ be the staggered location, for $j=0,\dots,N$, and note $x_c = x_N^* < x_{N+1}$.  Denote $B_j^*=B(x_j^*)$ and $M_j^*=M(x_j^*)$.

The mass continuity equation \eqref{eq:steadymass} is approximated by a second-order method centered at the staggered location.  For $j=0,\dots,N$,
\begin{equation}
\frac{u_{j+1} H_{j+1} - u_j H_j}{\Delta x} - M_j^* = 0 \label{eq:steadymassFD}
\end{equation}

In equation \eqref{eq:steadySSA} we avoid infinite viscosity by regularization \citep{SchoofStream}.  Let $\eps=1/L$ per year, i.e.~a strain rate corresponding to 1 m$/$a velocity change over the whole domain, $q = (1-n)/n$, and define
\begin{equation}
F(u_l,u_r) = \left(\left(\frac{u_r-u_l}{\Delta x}\right)^2 + \eps^2\right)^{q/2} \frac{u_r-u_l}{\Delta x}. \label{eq:viscregFD}
\end{equation}
Then we approximate the stress $T$ at staggered points,
\begin{equation}
T_j^* = 2 B_j^* \left(\frac{H_j + H_{j+1}}{2}\right) F(u_j,u_{j+1}), \label{eq:TFD}
\end{equation}
for $j=0,\dots,N$, and equation \eqref{eq:steadySSA} is approximated by
\begin{equation}
\frac{T_j^* - T_{j-1}^*}{\Delta x} - \beta_j u_j - \rho g H_j \frac{h_{j+1} - h_{j-1}}{2 \Delta x} = 0 \label{eq:steadySSAFD}
\end{equation}
where $\beta_j = k \rho g H_j$ if the ice is grounded at $x_j$ (i.e.~if $\rho H_j \ge \rho_w (z_o - b)$) and $\beta_j=0$ if the ice is floating, and where $h_j = H_j + b$ if the ice is grounded and $h_j = \omega H_j + z_o$ if the ice is floating.  Equation \eqref{eq:steadySSAFD} applies as stated both for grounded and floating ice, and it applies for all interior regular points, thus for $j=1,\dots,N$.

At this point we have $2N+4$ unknowns, namely $u_j,H_j$ for $j=0,\dots,N+1$.  There are $2N+1$ nonlinear equations in \eqref{eq:steadymassFD} and \eqref{eq:steadySSAFD} above.  The two upstream Dirichlet equations \eqref{eq:leftbc}, namely $u_0=u_a$ and $H_0=H_a$, and an approximation of the calving front condition \eqref{eq:rightbc} complete the system.  For the last one we write
\begin{equation}
\frac{1}{2} \omega \rho g \left(\frac{H_N + H_{N+1}}{2}\right)^2 = T_N^* \label{eq:rightbcFD}
\end{equation}
where $T_N^*$ is the approximation given in \eqref{eq:TFD}.

We now have a system of $2N+4$ equations in the same number of unknowns, which we can write abstractly as $\mathbf{F}(\mathbf{v})=0$.  These equations are solved by Newton's method \citep{Kelley} as implemented in PETSc \citep{petsc-user-ref}.  We first write a residual evaluation function which merely computes $\mathbf{F}(\mathbf{v})$ given $\mathbf{v}$, and a finite-difference Jacobian can be computed by PETSc, so this allows us to solve systems up to size about $N=1000$.  (It is critical in this finite-difference Jacobian case that the equations $\mathbf{F}(\mathbf{v})=0$ are scaled so that both the unknowns $\mathbf{v}$ and the ``residuals'', namely the output values of $\mathbf{F}$, have values which are $O(1)$.)  For higher resolution WE NEED AN ANALYTICAL JACOBIAN (FIXME).  FOR 2D CASES PERHAPS PRECONDITIONING BY AN APPROXIMATION TO THE JACOBIAN WOULD BE WISE.  

FIXME: WHAT ABOUT INITIAL GUESS?

\end{document}
