\NeedsTeXFormat{LaTeX2e}
%\documentclass[twocolumn]{igs}
%\documentclass[twocolumn,letterpaper]{igs}
\documentclass[review,letterpaper]{igs}

\usepackage{igsnatbib}  % see igs2eguide.tex for example citation styles

% check if we are compiling under latex or pdflatex
\ifx\pdftexversion\undefined
  \usepackage[dvips]{graphicx}
\else
  \usepackage[pdftex]{graphicx}
\fi

\usepackage{amsmath,xspace}

% for "review" mode, fix lineno problem
\newcommand*\patchAmsMathEnvironmentForLineno[1]{%
  \expandafter\let\csname old#1\expandafter\endcsname\csname #1\endcsname
  \expandafter\let\csname oldend#1\expandafter\endcsname\csname end#1\endcsname
  \renewenvironment{#1}%
     {\linenomath\csname old#1\endcsname}%
     {\csname oldend#1\endcsname\endlinenomath}}% 
\newcommand*\patchBothAmsMathEnvironmentsForLineno[1]{%
  \patchAmsMathEnvironmentForLineno{#1}%
  \patchAmsMathEnvironmentForLineno{#1*}}%
\AtBeginDocument{%
\patchBothAmsMathEnvironmentsForLineno{equation}%
\patchBothAmsMathEnvironmentsForLineno{align}%
\patchBothAmsMathEnvironmentsForLineno{flalign}%
\patchBothAmsMathEnvironmentsForLineno{alignat}%
\patchBothAmsMathEnvironmentsForLineno{gather}%
\patchBothAmsMathEnvironmentsForLineno{multline}%
}

\newcommand{\onecol}[1]{\includegraphics[width=86mm]{#1}}
\newcommand{\twocol}[1]{\includegraphics[width=178mm]{#1}}

\newcommand{\url}{\texttt{}}

% macros
\newcommand{\bQ}{\mathbf{Q}}
\newcommand{\bq}{\hat{\mathbf{q}}}
\newcommand{\br}{\hat{\mathbf{r}}}
\newcommand{\bU}{\mathbf{U}}
\newcommand{\hatx}{\hat{\mathbf{x}}}
\newcommand{\bx}{\mathbf{x}}
\newcommand{\CC}{\mathbb{C}}
\newcommand{\Div}{\nabla\cdot}
\newcommand{\ddx}[1]{\frac{\partial #1}{\partial x}}
\newcommand{\ddy}[1]{\frac{\partial #1}{\partial y}}
\newcommand{\pp}[2]{\frac{\partial #1}{\partial #2}}
\newcommand{\ppt}[1]{\frac{\partial #1}{\partial t}}
\newcommand{\ppT}[1]{\frac{\partial #1}{\partial T}}
\newcommand{\ppx}[1]{\frac{\partial #1}{\partial x}}
\newcommand{\ppy}[1]{\frac{\partial #1}{\partial y}}
\newcommand{\ppz}[1]{\frac{\partial #1}{\partial z}}
\newcommand{\ppzz}[1]{\frac{\partial^2 #1}{\partial z^2}}
\newcommand{\eps}{\epsilon}
\newcommand{\grad}{\nabla}
\newcommand{\hh}{\hat h}
\newcommand{\ip}[2]{\left(#1,#2\right)}
\newcommand{\lam}{\lambda}
\newcommand{\lap}{\triangle}
\newcommand{\mtt}{\texttt}
\newcommand{\RR}{\mathbb{R}}
\newcommand{\vf}{\varphi}

\begin{document}

\title{An exact solution for a steady, flow-line marine ice sheet}

\abstract{In 1955 G.~Bodvardsson published a flow-line exact solution which, though this seems not to have been recognized before, is an exact solution to the steady form of the simultaneous stress balance and mass continuity equations widely-used in numerical models of marine ice sheets.  Bodvardsson's solution, which has parabolic ice thickness and constant vertically-integrated longitudinal stress, solves the steady shallow shelf approximation on a flat bed.  It has elevation-dependent mass balance and, in the interpretation given here, location-dependent ice hardness.  By connecting Bodvardsson's solution to the \cite{vanderVeen83} solution for floating ice, we construct an exact solution to the ``rapid-sliding'' case of the marine ice sheet problem.  We exploit this exact solution to examine the accuracy of a grid-free numerical method and also a fully-implicit, fixed-grid finite difference method, focussing especially on the approximation of the geometry and velocity in the vicinity of the grounding line.}

\author{Ed Bueler}

\affiliation{Department of Mathematics and Statistics and Geophysical Institute, University of Alaska Fairbanks, USA \\
E-mail: \emph{\texttt{elbueler\@@alaska.edu}}}

\maketitle

\sectionsize

\subsection*{Introduction}  Early theoretical glaciology created two parabolic profiles as the shapes of steady flow-line ice sheets lying on flat beds, as in Figure \ref{fig:twoparabolas}.  One was the profile of an ice sheet with perfectly-plasticity \citep{Orowan,Nye52plastic} and the other the profile for a sliding ``plug'' flow investigated by \cite{Bodvardsson}.  These are global views of free surface flows in glaciology.  They focus on different aspects of the global problem and they come to rather different conclusions.\footnote{Draft date \emph{\today}.}

These parabolas exactly solve their respective nonlinear equations which determine ice thickness.  Up to scaling, one is of the form $x=1-y^2$ (Orowan-Nye) and the other is of the form $y=1-x^2$ (Bodvardsson).  The former perfect-plasticity solution has a central peak at the highest point of the ice sheet, and a margin with unbounded surface gradient.  The plug flow solution has a smooth dome and a finite-slope, wedge-shaped margin.

\begin{figure}[ht]
\onecol{twoparabolas}
\caption{The parabolas by Orowan and Nye (1949, 1952; dotted) and by Bodvardsson (1955; solid) for steady, flowline ice sheets on flat beds.  A dome thickness of $H_0=3000$ m and a length of $L_0=500$ km are chosen for concreteness.} \label{fig:twoparabolas}
\end{figure}

This paper combines Bodvardsson's solution with the better-known exact solution for an ice shelf \citep{vanderVeen83,vanderVeen} to generate an exact solution for a flowline, steady marine ice sheet.  The result, shown in Figure \ref{fig:exactmarine}, is an exact solution of the steady equations for the rapidly-sliding marine ice sheet case \citep{SchoofMarine1}.  This is the model addressed by the MISMIP intercomparison \citep{MISMIP2012}.  This exact solution simultaneously solves the steady mass continuity equation and the so-called shallow shelf approximation (``SSA''; Weis and others, 1999)\nocite{WeisGreveHutter} stress balance.

\subsection*{Continuum model}  Our model equations describe the steady-state, flat bed form of the flowline, rapid-sliding model of Schoof\nocite{SchoofMarine1} (2007; equations (2.1)--(2.5)).  We restrict to the linear sliding case, but with a nonconstant sliding coefficient.  The primary unknowns in these equations are the ice thickness $H(x)$, velocity $u(x)$, and vertically-integrated longitudinal stress $T(x)$ \citep{SchoofStream}, where $x$ is the flowline distance.  Using other notation from Table \ref{tab:notation}, the equations are
\begin{gather}
(uH)_x - M = 0, \label{eq:steadymass} \\
T_x - \beta u - \rho g H h_x = 0, \label{eq:steadySSA} \\
T = 2 B H |u_x|^{\frac{1}{n}-1} u_x. \label{eq:Tstress}
\end{gather}
Here the subscript $x$ denotes the derivative and
\begin{gather}
h = \begin{cases} H+b,            & \rho H \ge \rho_w (z_o - b) \\
                  \omega H + z_o, & \rho H < \rho_w (z_o - b) \end{cases}, \label{eq:surface} \\
\beta = \begin{cases} k \rho g H,    & \rho H \ge \rho_w (z_o - b) \\
                      0,          & \rho H < \rho_w (z_o - b) \end{cases}. \label{eq:betafull}
\end{gather}
Equations \eqref{eq:steadymass}--\eqref{eq:betafull} apply on an interval $0 < x < x_c$ where $x_c$ is the floating calving-front.  The grounded ice rests on flat bedrock at elevation $b$, a constant in this context.  Note $M(x)$ combines the surface and basal mass balance and $B(x)$ is the ice hardness, and these may depend on location.  In grounded ice the basal shear stress satisfies $\tau_b = - \beta u$ \citep{MacAyeal}, but we will scale the coefficient with the ice overburden pressure so that $\beta = k \rho g H$ \citep{Bodvardsson}.  The ``Archimedean factor'' $\omega = 1 - \rho/\rho_w$ relates surface elevation and thickness in floating ice.  Note $z=z_o$ is the elevation of the ocean surface.

\begin{figure}[ht]
\onecol{exactmarine-geometry}
\caption{The geometry (solid) and velocity (dashed) of an exact solution of the simultaneous steady mass continuity and SSA stress balance equations for a marine ice sheet.  It is Bodvardsson's parabola when grounded and van der Veen's solution when floating.} \label{fig:exactmarine}
\end{figure}

Equations \eqref{eq:steadymass} and \eqref{eq:steadySSA} are the mass-continuity and SSA stress balance equations, respectively, while \eqref{eq:Tstress} defines $T$.  Equation \eqref{eq:surface} says that the ice surface $z=h$ is at elevation $H+b$ when the ice is grounded, and otherwise the ice surface $z=h$ is found from the Archimedean principle.  Likewise equation \eqref{eq:betafull} gives the scaled form of the basal shear stress for grounded ice, while that stress is zero for floating ice.  The grounding line $x=x_g$ is a location which must be solved-for, at which we know $\rho H(x_g) = \rho_w (z_o - b)$.

For the exact and numerical solutions in this paper, equations \eqref{eq:steadymass}--\eqref{eq:betafull} are augmented by boundary conditions:
\begin{gather}
u(0) = u_a > 0, \quad H(0) = H_a > 0, \label{eq:leftbc} \\
T(x_c) = \frac{1}{2} \omega \rho g H(x_c)^2,  \label{eq:rightbc} \\
H, u, T \quad \text{ continuous at } x = x_g.  \label{eq:xgregularity}
\end{gather}
Here $x=0$ is an upstream location where Dirichlet boundary conditions are applied (equation \eqref{eq:leftbc}), while at the (non-moving) calving front $x_c$ we have the standard hydrostatic pressure ``imbalance'' condition \eqref{eq:rightbc} \citep{SchoofMarine1}.  Fact \eqref{eq:xgregularity} at $x_g$ may be regarded as a regularity requirement, and not strictly a boundary condition; see below.  By equation \eqref{eq:Tstress}, $T$ is continuous at $x_g$ if and only if $u_x$ is continuous at $x_g$.

\subsection*{On well-posedness and the grounding line}  Given data $B(x)$, $b$, $k>0$, $M(x)$, $x_c>0$, $z_o$, along with physical constants $g,n,\rho,\rho_w$, we expect the problem consisting of equations \eqref{eq:steadymass}--\eqref{eq:xgregularity} to be well-posed.  To our knowledge this has not been proved, nor do we attempt to prove it.  It is, however, worth considering the smoothness (regularity) of the solution to \eqref{eq:steadymass}--\eqref{eq:xgregularity}, including what hypotheses would lead to satisfying \eqref{eq:xgregularity} at the free (unknown) location $x_g$ in the interior of the domain.

Indeed, suppose that, for physical reasons, the mass balance $M(x)$ and ice hardness $B(x)$ are bounded, and further that $B(x)$ is bounded below by a positive constant.  From integrating $M$, equation \eqref{eq:steadymass} implies that the flux $q=uH$ is absolutely-continuous and thus bounded.  If there is a positive lower bound on thickness $H$, then we can conclude that the magnitude of $u$ is bounded because $u=q/H$.  If the magnitude of the driving stress $-\rho g H h_x$ is bounded then equation \eqref{eq:steadySSA} implies $T$ is absolutely-continuous.  By equation \eqref{eq:Tstress} this implies $u$ has a bounded and integrable derivative, and thus that $u$ is also absolutely-continuous.  From these facts we could then return to the flux and write $H=q/u$ which shows $H$ is absolutely-continuous away from locations where $u=0$ (e.g.~divides).  In summary, assuming \emph{(i)} that an integrable solution $(H,u,T)$ to \eqref{eq:steadymass}--\eqref{eq:xgregularity} exists, \emph{(ii)} that the functions $M,B$ are bounded and integrable, \emph{(iii)} that $B$ is bounded below by a positive constant, \emph{(iv)} that a positive lower bound on thickness exists, and \emph{(v)} that an upper bound on the magnitude of the driving stress exists, then we can regard condition \eqref{eq:xgregularity}, giving the continuity at the grounding line, as properties of the solution instead of as part of the ``imposed'' problem statement.

%\small
\begin{table}
\caption{Notation and SI units, including physical constants.}\label{tab:notation}

\medskip
\begin{tabular}{llll}
Symbol & Description & Units \\ \hline
$B$ & ice hardness; $=A^{-1/n}$ & $\text{Pa}\,\text{s}^{1/3}$  \\
$b$ & bedrock elevation & m \\
$\beta$ & sliding coefficient & $\text{Pa}\,\text{s}\,\text{m}^{-1}$ \\
$g$ & acceleration of gravity  & 9.81 $\text{m}\,\text{s}^{-2}$\\
$H$ & ice thickness & m \\
$h$ & ice surface elevation & m \\
$k$ & pressure-scaled sliding coefficient  & $\text{s}\,\text{m}^{-1}$ \\
$M$ & mass balance & $\text{m}\,\text{s}^{-1}$ \\
$n$ & Glen exponent in ice flow law & 3 \\
$\rho$ & density of ice & 910 $\text{kg}\,\text{m}^{-3}$ \\
$\rho_w$ & density of sea water & 1028 $\text{kg}\,\text{m}^{-3}$ \\
$T$ & $z$-integrated longitudinal stress & $\text{Pa}\,\text{m}$ \\
$\tau_{b}$ & basal shear stress applied to ice & Pa \\
$u$ & horizontal velocity & $\text{m}\,\text{s}^{-1}$ \\
$(x,z)$ & flow-line cartesian coordinates & m  \\
$x_g$ & grounding line & m  \\
$x_c$ & calving front & m  \\
$z_o$ & ocean surface elevation & m \\
$\omega$ & Archimedean factor; $=1 - \rho/\rho_w$ & $0.115$
\end{tabular}
\end{table}
%\normalsize

\subsection*{Bodvardsson's parabola}  \cite{Bodvardsson} built, from minimal existing literature of course, to a rigorous initial theory of the flow of glaciers and ice sheets.  His test case was Br\'uarj\"okull, a glacier on the northern margin of Vatnaj\"okull in Iceland.  It flows over a smooth bed for 20 km, from a location where its thickness is 600 m, to a zero thickness margin.  This glacier is entirely grounded.  He shows that a good fit to measured surface elevations can be made using his model.

He initially states an equation for the ice sheet surface elevation which has both vertical-plane shear and longitudinal stress within the ice.  However, he says this equation ``is quite tedious and very difficult to handle especially because of the [shear] term in the parentheses.  It is therefore fortunate that [the shear] term appears to be small compared to the [basal sliding] term.''  Then he drops the shear term and writes an equation in which driving stress is balanced entirely by basal resistance.  He solves and analyzes this plug flow model, which we now recall.

For basal resistance he chooses a coefficient which scales with the overburden pressure, so that the basal shear stress is
\begin{equation}
\tau_b = - k \rho g H u  \label{eq:bodstresschoice}
\end{equation}
He then writes the ice flux as $uH=-(H/k) H_x$, or equivalently the ice velocity as
\begin{equation}
u = - \frac{1}{k} H_x. \label{eq:bodvelocity}
\end{equation}
As is perhaps best-known about Bodvardsson's work, he chooses the surface mass balance to be
\begin{equation}
M = a (H - H_{ela})  \label{eq:bodmassbalance}
\end{equation}
for a mass balance gradient $a>0$ and equilibrium-line altitude $H_{ela}$.  Combining these equations yields \citep[equation (17)]{Bodvardsson}
\begin{equation}
a (H - H_{ela}) + (k^{-1} H H_x)_x = 0  \label{eq:bodsteady}
\end{equation}
for the thickness.  His solution to this equation is \citep[equivalent equations (18) and (23)]{Bodvardsson}
\begin{equation}
H(x) = H_0 (1 - (x/L_0)^2)  \label{eq:bodsoln}
\end{equation}
where $H_0 = 1.5 H_{ela}$ and $a k L_0^2 = 9 H_{ela}$ \citep[equation (24)]{Bodvardsson}.

Despite its simplicity, equation \eqref{eq:bodsoln} is an exact solution to equation \eqref{eq:bodsteady}, with boundary condition $H(0)=H_0$, as the reader may verify.  Bodvardsson does not offer a reason why there should be such a simple quadratic solution, and his solution seems to be a not-well-known result for a narrow class of nonlinear second-order ODEs; see Appendix A.  Formula \eqref{eq:bodsoln} defines the solid parabola shown in Figure \ref{fig:twoparabolas}.

Bodvardsson explicitly considers solution \eqref{eq:bodsoln} as solving a free boundary problem, in the sense that the single boundary condition $H(0)=H_0$ applied to the equation \eqref{eq:bodsteady} determines the quadratic solution \eqref{eq:bodsoln}.  Solving \eqref{eq:bodsteady} with one boundary condition generates a value $L_0$ from the hypothesis that the flux and the thickness at $x=L_0$ are both zero.

The \cite{Bodvardsson} solution for a grounded glacier was apparently first cited by \cite{Weertman61stability}.  Weertman who decided that the physics chosen by Bodvardsson should be replaced by a shear deformation model more like the shallow ice approximation, though he allowed sliding as well.  This replacement seems to have influenced readers from then on.

The surface balance parameterization \eqref{eq:bodmassbalance} reappears in \cite{Weertman61stability}, among many other places.  It realistically parameterizes a potential climatic instability, which was Bodvardsson's, Weertman's, and most readers', major interest.  We are, however, interested now in Bodvardsson's solution to the ice flow equations themselves.


\subsection{A re-interpretation}  We now observe that \eqref{eq:bodsoln} exactly solves a combination of the steady flow-line mass-continuity equation \eqref{eq:steadymass} and the SSA stress balance equation \eqref{eq:steadySSA}, but with contrived (``manufactured'') ice softness.

In fact, suppose we consider solutions of \eqref{eq:steadymass} and \eqref{eq:steadySSA} with constant vertically-integrated longitudinal stress, $T_x \equiv 0$.  In that case equation \eqref{eq:Tstress} implies Bodvardsson's formula for the velocity, namely equation \eqref{eq:bodvelocity}.  Furthermore, equations \eqref{eq:steadymass} and \eqref{eq:bodmassbalance} then give Bodvardsson's main equation \eqref{eq:bodsteady}.  That is, if \emph{(i)} $T_x \equiv 0$, if \emph{(ii)} sliding resistance is linear and scales with overburden pressure, and if \emph{(iii)} mass balance is proportional to elevation above the equilibrium line, then we recover a simple parabolic profile for $H(x)$, namely equation \eqref{eq:bodsoln}.

But what does the condition ``$T_x\equiv 0$'' imply as a relation among the modeled quantities?  Given a thickness profile $H(x)$ and a strain rate profile $u_x(x)$, we may interpret ``$T_x\equiv 0$'' as a statement of \emph{variable ice hardness}.  Such variable hardness is physical and common to many numerical models using the SSA \citep[for example]{BBssasliding}, so implementing such $x$-dependent hardness requires no new programming.  In particular, if $T=T_0$ is the constant value then equation \eqref{eq:Tstress} implies
\begin{equation}
B(x) = \frac{T_0}{2 H |u_x|^{(1/n)-1} u_x}. \label{eq:hardnessdefine}
\end{equation}


\subsection*{Extending the exact solution to floating ice}  The value $T_0$ in \eqref{eq:hardnessdefine} can be set by a downstream stress condition, just as it is in many models for flowline ice shelves \citep[e.g.][]{MISMIP2012,SchoofMarine1}.  Two well-known observations are relevant:  \emph{(i)}  For floating ice with $\beta=0$, equations \eqref{eq:steadymass}--\eqref{eq:rightbc} also have a known exact solution, specifically in the case where the mass balance $M$ and the ice hardness $B$ are constant \citep{vanderVeen83,vanderVeen}.  \emph{(ii)}  The vertically-integrated longitudinal stress $T$ in a flowline ice shelf (i.e.~one without lateral stresses) satisfies equation \eqref{eq:rightbc} at each location $x$ in the shelf, that is $T(x) = \frac{1}{2} \omega \rho g H(x)^2$, because this is a first integral of equation \eqref{eq:steadySSA} if $\beta=0$.

Based on these observations we can construct a marine ice sheet exact solution by extending Bodvardsson's grounded solution above to the floating ice.  First taking $b=0$ as the flat bed elevation, we suppose the ocean has surface elevation $z_o>0$, thus determining the grounding-line thickness by $H(x_g) = (\rho_w/\rho) (z_o - b)$.  Then from Bodvardsson's thickness solution \eqref{eq:bodsoln} we can determine $x_g$.  At $x_g$, from \eqref{eq:bodvelocity} and \eqref{eq:bodsoln}, we know $u(x_g)$ as well.  For $x_g \le x \le x_c$, the floating ice shelf, we then set $M(x) = M(x_g)$ as constant from the formula \eqref{eq:bodmassbalance}, thus making $M(x)$ continuous across the grounding line.  The equation $T(x) = \frac{1}{2} \omega \rho g H(x)^2$ determines $T_o=T(x_g)$ for use in equation \eqref{eq:hardnessdefine}, which determines $B(x_g)$ in particular, and so then we set $B(x)=B(x_g)$ for $x_g \le x \le x_c$.

The result of the above construction is the following set of formulas for an exact marine ice sheet satisfying our steady model equations \eqref{eq:steadymass}--\eqref{eq:xgregularity}.  The velocity comes from combining the Bodvardsson (1955) and van der Veen (1983) results,
\begin{equation}
u(x) = \begin{cases} \frac{2 H_0}{k L_0^2}\,x, & 0 \le x \le x_g, \\
                     u_s(x), & x_g \le x \le x_c.
       \end{cases} \label{eq:marinevel}
\end{equation}
where $u_s(x)$ is defined by
\begin{equation}
u_s(x)^{n+1} = u(x_g)^{n+1} + \frac{C_s}{M(x_g)} \left(\left[Q_g + M(x_g) (x-x_g)\right]^{n+1} - Q_g^{n+1}\right) \label{eq:vanderveenvel}
\end{equation}
and $C_s = \left(\rho g \omega/(4 B(x_g))\right)^n$ and $Q_g = u(x_g) H(x_g)$.  Similarly the thickness is:
\begin{equation}
H(x) = \begin{cases} H_0 \left(1 - (\frac{x}{L_0})^2\right), & 0 \le x \le x_g, \\
                     \frac{Q_g + M(x_g) (x-x_g)}{u_s(x)}, & x_g \le x \le x_c.
       \end{cases} \label{eq:marinethickness}
\end{equation}
Formulas \eqref{eq:marinevel} and \eqref{eq:marinethickness} define the continuous functions which are shown in Figure \ref{fig:exactmarine}, using the specific values in Table \ref{tab:exactsoln}.

%\small
\begin{table}
\caption{Specific values of the exact solution shown in Figures \ref{fig:exactmarine}--\ref{fig:exactmarine-detail}; ``g.l.'' = grounding line and ``c.f.'' = calving front.}\label{tab:exactsoln}

\medskip
\begin{tabular}{llll}
Symbol & Description & Units \\ \hline
$H(0)$ & thickness at $x=0$ & 2880 m  \\
$u(0)$ & ice velocity at $x=0$ & 100 $\text{m}\,\text{a}^{-1}$  \\ \hline
$x_g$ & location of g.l. & 350 km  \\
$B(x_g)$ & ice hardness at g.l. & $4.614 \times 10^{8}$ $\text{Pa}\,\text{s}^{1/3}$  \\
$H(x_g)$ & thickness at g.l. & 570 m  \\
$M(x_g)$ & mass balance at g.l. & -4.290 $\text{m}\,\text{a}^{-1}$  \\
$T(x_g)$ & stress at g.l. & $1.665 \times 10^{8}$ $\text{Pa}\,\text{m}$  \\
$u(x_g)$ & ice velocity at g.l. & 450 $\text{m}\,\text{a}^{-1}$  \\ \hline
$x_c$ & location of c.f. & 390 km  \\
$H(x_c)$ & thickness at c.f. & 182.938 m  \\
$T(x_c)$ & stress at c.f. & $0.171 \times 10^{8}$ $\text{Pa}\,\text{m}$  \\
$u(x_c)$ & ice velocity at c.f. & 464.092 $\text{m}\,\text{a}^{-1}$  \\ \hline
$b$ & bedrock elevation & 0 m \\
$z_o$ & ocean surface elevation & 504.572 m \\
\end{tabular}
\end{table}
%\normalsize

From the thickness $H(x)$ and the velocity $u(x)$ we can find continuous functions $M(x)$ and $B(x)$ for the full marine flowline by using equations \eqref{eq:bodmassbalance} and \eqref{eq:hardnessdefine}.  These functions are shown in Figure \ref{fig:exactMB}.  Then we can use equation \eqref{eq:Tstress} to find $T(x)$; this is shown in Figure \ref{fig:exactbetaT}.  In Figure \ref{fig:exactbetaT} we also show the sliding coefficient $\beta(x)$, which drops to zero discontinuously at $x_g$.  Finally Figure \ref{fig:exactmarine-detail} shows a detail of the grounding line and floating ice in the exact solution.

\begin{figure}[ht]
\onecol{exactmarine-M-B}
\caption{The mass balance $M(x)$ (solid) and ice hardness $B(x)$ (dashed) of the exact solution.} \label{fig:exactMB}
\end{figure}

\begin{figure}[ht]
\onecol{exactmarine-beta-T}
\caption{The sliding coefficient $\beta(x)$ (solid) and the vertically-integrated longitudinal stress $T(x)$ (dashed) for the exact solution.  The solid curve shows $\beta = k \rho g H$ on both sides of the grounding line.  The actual basal resistance experienced by the shelf drops to zero at the grounding line (dotted).} \label{fig:exactbetaT}
\end{figure}

Note that, because this paper is focussed on ice flow dynamics and grounding lines, we treat $M(x)$ as a predetermined field (i.e.~the one shown in Figure \ref{fig:exactMB}).  This removes the climatically-interesting elevation--accumulation feedback, and the associated instability, which was considered by Bodvardsson, Weertman, and others.  This feedback can be added back by restoring the elevation-dependence of the mass balance, that is, by using equation \eqref{eq:bodmassbalance} to determine $M$ from $H$.

\begin{figure}[ht]
\onecol{exactmarine-geometry-detail}
\caption{Detail of Figure \ref{fig:exactmarine}, showing the floating ice shelf geometry and velocity.} \label{fig:exactmarine-detail}
\end{figure}

The floating ice shelf is a relatively short 40 km; see Figures \ref{fig:exactmarine} and \ref{fig:exactmarine-detail}.  To explain, note that the equilibrium line $H_{ela}$ in Bodvardsson's (1955) solution is high on the ice sheet because of its relation to the upstream ice thickness in the construction of the exact solution (i.e.~$H_{ela} = (2/3) H_0$).  This in turn implies $M(x_g)$ is quite negative (equation \eqref{eq:bodmassbalance}; see Figure \ref{fig:exactMB}).  Because van der Veen's (1983) solution uses constant mass balance, and because we want continuity for $M(x)$, we therefore have an ice shelf experiencing rapid melting.  The location of the calving front $x_c$ must be put before the location where the ice has melted away.  As a result of these same factors we can also see a rapid decline in the stress $T(x)$ from its constant grounded value to its small value at $x_c$ (Figure \ref{fig:exactbetaT} and Table \ref{tab:exactsoln}).  Though the ice shelf shown here is very wedge-like, the thickness $H(x)$ for floating ice comes from formula \eqref{eq:marinethickness} and thus it is not a simple linear function.


\subsection*{Numerical results}

FIXME:  When one does nonlinear shooting using the exact correct $T$ value, one gets the best-case adaptive-grid error in Figure \ref{fig:shoot-error}.

\begin{figure}[ht]
\onecol{exactmarine-error}
\caption{Pointwise error in thickness (upper panel) and in velocity (lower panel) from an adaptive numerical ODE scheme.  Both the ``cheating'' case (solid), where we use the exactly-correct initial value for $T$, and the ``realistic'' case (dashed), where the shooting method converges on the correct initial value for $T$ by the bisection method, are shown.} \label{fig:shoot-error}
\end{figure}

FIXME: FIGURE SHOWING GRID(s)

A special-purpose, stand-alone C code implementing a second-order finite difference scheme is described in the appendix.  It is also available as part of the PISM source code.  It has a parallel implementation using the PETSc \citep{petsc-user-ref} library and it uses a Newton scheme to achieve observed quadratic convergence of the iteration.

Figure \ref{fig:verifNresult} shows the result of using this numerical scheme to approximate the exact solution.  Both the maximum numerical thickness error and the maximum numerical velocity error are observed to converge at essentially an $O(\Delta x^2)$ error as the grid spacing $\Delta x$ refines to zero.  

\begin{figure}[ht]
\onecol{verifN}
\caption{FIXME: REGENERATE USING marine.c.  Verification result for finite difference code on grounded-only problem.  The ice thickness error decays at rate $O(\Delta x^{1.97})$ (dotted).} \label{fig:verifNresult}
\end{figure}


\subsection*{Conclusions} BLAH BLAH

\subsection*{Acknowledgements and erratum}  Thanks to Heinz Blatter and Helgi Bj\"ornsson for tracking down the original paper by Bodvardsson.  \cite{BLKCB} identify the constant accumulation SIA solution as ``Bodvarsson (1955)--Vialov (1958).''  This is incorrect as it is attributable only to Vialov.  Also, the spelling of Bodvardsson's name is ``Bodvarsson'' in many places, but the original paper in 1955 has ``Bodvardsson'' with a ``d''.  \cite{Weertman61stability}, an early response to Bodvardsson's work, drops the ``d''.


%         References
\bibliography{ice-bib}
\bibliographystyle{igs}

\appendix

\subsection{Appendix A: Bodvardsson's little theorem}  \cite{Bodvardsson} does not identify a source for the exact parabolic thickness solution to his plug flow equations, and it seems likely that he derived it from scratch.  We summarize his result as the theorem that if $A$ and $B$ are constant then there is a unique polynomial solution $y(x)$ to the nonlinear second-order differential equation
\begin{equation}
  (y y')' = Ay+B  \label{eq:abstractode}
\end{equation}
satisfying the single boundary condition $y(0) = y_0 > 0$, subject to both an initial downslope assumption ($y'(0) \le 0$) and the technical inequality
\begin{equation}
2A y_0 + 3 B \ge 0.  \label{eq:abstractinequality}
\end{equation}
Equation \eqref{eq:abstractode} is equation (17) in \cite{Bodvardsson}.  The additional assumptions (i.e.~initial downslope plus \eqref{eq:abstractinequality}) are unstated, though he comments that there is ``one and only one solution which is admittable from the physical point of view.''

Here we justify this little theorem and, generally following \cite{Bodvardsson}, derive relations among parameters which yield the solution.  The unique polynomial solution to this problem may be interpreted as solving a free boundary problem for the first positive zero $x_0>0$ of $y(x)$.  In Bodvardsson's context $x_0=L_0$ is the length of the glacier, the location of the margin (in the ``dry'' case).

It is easy to see by substitution into \eqref{eq:abstractode} that nontrivial solutions of degree $d$, i.e.~of the form $y(x) = \gamma x^d + (\text{lower degree})$ with $\gamma\ne 0$, exist only if $d=2$.  In that case we seek solutions which satisfy the boundary conditions, so
\begin{equation}
y(x) = y_0(1 - \alpha x + \beta x^2)  \label{eq:abstractsoln}
\end{equation}
for some $\alpha\ge 0$ and $\beta$ which are to be determined from $A,B$; this is equation (18) in \citep{Bodvardsson}.  Substitution gives the two equations
\begin{equation}
3 y_0^2 \alpha^2 = 2 A y_0 + 3 B \quad \text{ and } \quad 6 y_0 \beta = A.  \label{eq:abstractrelations}
\end{equation}
These two relations determine $\alpha,\beta$ from $A,B$.  The first relation explains \eqref{eq:abstractinequality}, noting $y_0$ and $\alpha$ are real.

In the main text, Bodvardsson's problem relates four numbers: the initial (upstream) ice thickness $y_0=H_0$, an ablation gradient $a>0$, the equilibrium-line altitude $H_{ela}$, and a sliding constant $k>0$ to the glacier thickness $y(x) = H(x)$.  He has $A=-ka$ and $B=k a H_{ela}$ in \eqref{eq:abstractode} so the technical condition \eqref{eq:abstractinequality} says $3 H_{ela} \ge 2 H_0$ after simplification.  This causes the equilibrium line altitude to be relatively high on the glacier.


\subsection{Appendix B: A finite difference scheme}  The steady-state equations for mass continuity \eqref{eq:steadymass} and stress balance \eqref{eq:steadySSA} form a coupled system that can be approximately solved by the numerical scheme described here.  It uses centered, second-order finite differences for both equations.

We define an equally-spaced grid on the domain $[x_a,x_c]$.  Because the boundary condition at the calving front evaluates the stress $T$, the right endpoint $x_c$ is at a ``staggered'' location halfway in-between grid points.  Let $L=x_c-x_a$.  If $N$ is an integer then we define $\Delta x = L / (N+1/2)$ and $x_j = x_a + j\Delta x$ for $j=0,\dots,N+1$.  Denote the numerical approximations $H_i\approx H(x_i)$ and $u_i \approx u(x_i)$.  Let $x_j^* = x_j + \Delta x/2$ be the staggered location, for $j=0,\dots,N$, and note $x_c = x_N^* < x_{N+1}$.  Denote $B_j^*=B(x_j^*)$ and $M_j^*=M(x_j^*)$.

The mass continuity equation \eqref{eq:steadymass} is approximated by a second-order method centered at the staggered location.  For $j=0,\dots,N$,
\begin{equation}
\frac{u_{j+1} H_{j+1} - u_j H_j}{\Delta x} - M_j^* = 0 \label{eq:steadymassFD}
\end{equation}

In equation \eqref{eq:steadySSA} we avoid infinite viscosity by regularization \citep{SchoofStream}.  Let $\eps=1/L$ per year, i.e.~a strain rate corresponding to 1 m$/$a velocity change over the whole domain, $q = (1-n)/n$, and define
\begin{equation}
F(u_l,u_r) = \left(\left(\frac{u_r-u_l}{\Delta x}\right)^2 + \eps^2\right)^{q/2} \frac{u_r-u_l}{\Delta x}. \label{eq:viscregFD}
\end{equation}
Then we approximate the stress $T$ at staggered points,
\begin{equation}
T_j^* = B_j^* \left(H_j + H_{j+1}\right) F(u_j,u_{j+1}), \label{eq:TFD}
\end{equation}
for $j=0,\dots,N$, and equation \eqref{eq:steadySSA} is approximated by
\begin{equation}
\frac{T_j^* - T_{j-1}^*}{\Delta x} - \beta_j u_j - \rho g H_j \frac{h_{j+1} - h_{j-1}}{2 \Delta x} = 0 \label{eq:steadySSAFD}
\end{equation}
where $\beta_j = k \rho g H_j$ if the ice is grounded at $x_j$ (i.e.~if $\rho H_j \ge \rho_w (z_o - b)$) and $\beta_j=0$ if the ice is floating, and where $h_j = H_j + b$ if the ice is grounded and $h_j = \omega H_j + z_o$ if the ice is floating.  Equation \eqref{eq:steadySSAFD} applies as stated both for grounded and floating ice, and it applies for all interior regular points, thus for $j=1,\dots,N$.

At this point we have $2N+4$ unknowns, namely $u_j,H_j$ for $j=0,\dots,N+1$.  There are $2N+1$ nonlinear equations in \eqref{eq:steadymassFD} and \eqref{eq:steadySSAFD} above.  The two upstream Dirichlet equations \eqref{eq:leftbc}, namely $u_0=u_a$ and $H_0=H_a$, and an approximation of the calving front condition \eqref{eq:rightbc} complete the system.  For the last one we write
\begin{equation}
\frac{1}{2} \omega \rho g \left(\frac{H_N + H_{N+1}}{2}\right)^2 = T_N^* \label{eq:rightbcFD}
\end{equation}
where $T_N^*$ is the approximation given in \eqref{eq:TFD}.

We now have a system of $2N+4$ equations in the same number of unknowns, which we can write abstractly as $\mathbf{F}(\mathbf{v})=0$.  These equations are solved by Newton's method \citep{Kelley} as implemented in PETSc \citep{petsc-user-ref}.  We first write a residual evaluation function which merely computes $\mathbf{F}(\mathbf{v})$ given $\mathbf{v}$, and a finite-difference Jacobian can be computed by PETSc, so this allows us to solve systems up to size about $N=1000$.  (It is critical in this finite-difference Jacobian case that the equations $\mathbf{F}(\mathbf{v})=0$ are scaled so that both the unknowns $\mathbf{v}$ and the ``residuals'', namely the output values of $\mathbf{F}$, have values which are $O(1)$.)  For higher resolution WE NEED AN ANALYTICAL JACOBIAN (FIXME).  FOR 2D CASES PERHAPS PRECONDITIONING BY AN APPROXIMATION TO THE JACOBIAN WOULD BE WISE.  

FIXME: WHAT ABOUT INITIAL GUESS?

\end{document}
