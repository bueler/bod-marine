\NeedsTeXFormat{LaTeX2e}
\documentclass[twocolumn]{igs}
%\documentclass[twocolumn,letterpaper]{igs}
%\documentclass[review,letterpaper]{igs}

\usepackage{igsnatbib}  % see igs2eguide.tex for example citation styles
\usepackage[pdftex]{graphicx}
\usepackage{amsmath}
\usepackage{xspace}
\usepackage[T1]{fontenc}
\renewcommand{\dh}{\fontencoding{T1}\selectfont{\symbol{240}}}

\newcommand{\onecol}[1]{\includegraphics[width=86mm]{#1}}
\newcommand{\twocol}[1]{\includegraphics[width=178mm]{#1}}

\newcommand{\bod}{B\"o\dh varsson\xspace}
\newcommand{\bods}{B\"o\dh varsson's}
\newcommand{\citebod}{B\"o\dh varsson (1955)\nocite{Bodvardsson}\xspace}
\newcommand{\citepbod}{(B\"o\dh varsson, 1955)\nocite{Bodvardsson}\xspace}

\newcommand{\bw}{\mathbf{w}}
\newcommand{\hu}{\hat u}
\newcommand{\hH}{\hat H}
\newcommand{\hT}{\hat T}
\newcommand{\tu}{\tilde u}
\newcommand{\tH}{\tilde H}
\newcommand{\tT}{\tilde T}
\newcommand{\eps}{\epsilon}
\newcommand{\Hela}{H_{\text{ela}}}


\begin{document}

\title{An exact solution for a steady, flow-line marine ice sheet}

\abstract{G.~\bod's 1955 plug flow solution for an Icelandic glacier problem is shown to be an exact solution to the steady form of the simultaneous stress balance and mass continuity equations widely-used in numerical models of marine ice sheets.  The solution, which has parabolic ice thickness and constant vertically-integrated longitudinal stress, solves the steady shallow shelf approximation with linear sliding on a flat bed.  It has elevation-dependent surface mass balance rate and, in the interpretation given here, a contrived location-dependent ice hardness distribution.  By connecting \bod's solution to the \cite{vanderVeen83} solution for floating ice, we construct an exact solution to the ``rapid-sliding'' marine ice sheet problem, continuously across the grounding line.  We exploit this exact solution to examine the accuracy of two numerical methods, one grid-free and the other based on a fixed, equally-spaced grid.}

\author{Ed Bueler}

\affiliation{Department of Mathematics and Statistics and Geophysical Institute, University of Alaska Fairbanks, USA \\
E-mail: \emph{\texttt{elbueler\@@alaska.edu}}}

\maketitle

\sectionsize

\section*{Introduction}

Early theoretical glaciology created two fundamentally-different parabolic profiles as the shapes of steady flow-line ice sheets lying on flat beds, as in Figure \ref{fig:twoparabolas}.  One was the profile of an ice sheet with perfect-plasticity \citep{Orowan,Nye52plastic} and the other the profile for a sliding ``plug'' flow investigated by \citebod.  These global views of free surface flows in glaciology focus on different aspects of the problem and they come to rather different conclusions.  Up to scaling, one is of the form $x=1-y^2$ (Orowan-Nye) and the other is of the form $y=1-x^2$ (\bod).  The former perfect-plasticity solution has a central peak at the highest point of the ice sheet, and a margin with unbounded surface gradient.  The latter plug flow solution has a smooth dome and a finite-slope, wedge-shaped margin.

\begin{figure}[ht]
\onecol{twoparabolas}
\caption{The parabolas by Orowan and Nye (1949, 1952; dotted) and by \bod (1955; solid) for steady, flowline ice sheets on flat beds.  A dome thickness of $H_0=3000$ m and a length of $L_0=500$ km are chosen for concreteness.} \label{fig:twoparabolas}
\end{figure}

This paper shows how to combine \bod's solution with the well-known exact solution for an ice shelf \citep{vanderVeen83,vanderVeen} to generate an exact solution for a flowline, steady marine ice sheet.  It is shown in Figure \ref{fig:exactmarine}.  This exact solution simultaneously solves the steady mass continuity equation and the so-called shallow shelf approximation (``SSA''; Weis and others, 1999)\nocite{WeisGreveHutter} stress balance.  It is an exact solution in the rapidly-sliding marine ice sheet case \citep{SchoofMarine1}, the model addressed by the MISMIP intercomparison \citep{MISMIP2012}.  After presenting the model equations and constructing the exact solution in the next two sections, we examine errors made by two different numerical methods.  We observe, through linearization of the equations around the exact solution, that the grounding line generates a strong ``stiffness'' constrast in the sense of numerical analysis.


\section{Continuum model}

\subsection*{Model equations}  Our model equations describe the steady-state, flat bed case of the rapid-sliding model of Schoof\nocite{SchoofMarine1} (2007; equations (2.1)--(2.5)), but we restrict to the linear sliding case.  The primary unknowns in these equations are the ice thickness $H(x)$, velocity $u(x)$, and vertically-integrated longitudinal stress $T(x)$ \citep{SchoofStream}, where $x$ is the flowline distance.  Using notation from Table \ref{tab:notation}, the equations are
\begin{gather}
(uH)_x - M = 0, \label{eq:steadymass} \\
T_x - \beta u - \rho g H h_x = 0, \label{eq:steadySSA} \\
T = 2 B H |u_x|^{\frac{1}{n}-1} u_x. \label{eq:Tstress}
\end{gather}
Here the subscript $x$ denotes the derivative.  Equations \eqref{eq:steadymass} and \eqref{eq:steadySSA} are the mass-continuity and SSA stress balance equations, respectively, while \eqref{eq:Tstress} defines $T$.  In \eqref{eq:steadymass} the climatic-basal mass balance rate $M(x)$ \citep{massbalanceglossary} combines the surface mass balance rate and the rate of basal melt/refreeze; it is the accumulation/ablation rate function for a fluid layer.  In grounded ice the basal shear stress is linear so $\tau_b = - \beta u$ \citep{MacAyeal} appears in \eqref{eq:steadySSA}, with $\beta=0$ in floating ice so the same linear form holds throughout.  The coefficient $B(x)$ in \eqref{eq:Tstress} is called the ice ``hardness.''

Let $\omega = 1 - \rho/\rho_w$ be the ``Archimedean factor'' which relates ice surface elevation to thickness in floating ice and let $z_o$ be the elevation of the ocean surface.  The grounded ice rests on bedrock at elevation $b$, which we assume is a constant in this paper.  By the flotation criterion,
\begin{equation}
h = \begin{cases} H+b,            & \rho H \ge \rho_w (z_o - b) \\
                  \omega H + z_o, & \rho H < \rho_w (z_o - b) \end{cases}, \label{eq:surface}
\end{equation}
is the ice surface elevation in grounded and floating cases, respectively.  Using the same flotation criterion we can clarify the basal shear stress,
\begin{equation}
\beta = \begin{cases} k \rho g H,    & \rho H \ge \rho_w (z_o - b) \\
                      0,          & \rho H < \rho_w (z_o - b) \end{cases}, \label{eq:betafull}
\end{equation}
where we have scaled $\beta = k \rho g H$ with the ice overburden pressure $\rho g H$ following \citebod.

Equations \eqref{eq:steadymass}--\eqref{eq:betafull} apply on an interval $0 < x < x_c$ where $x_c$ is the floating calving-front.  We must solve for the location $x=x_g$ of the grounding line, at which we know $\rho H(x_g) = \rho_w (z_o - b)$.

\begin{figure}[ht]
\onecol{em-geometry}
\caption{The geometry (solid) and velocity (dashed) of an exact solution of the simultaneous steady mass continuity and SSA stress balance equations for a marine ice sheet.  The solution is \bod's when grounded and van der Veen's when floating.} \label{fig:exactmarine}
\end{figure}

For the exact and numerical solutions in this paper, equations \eqref{eq:steadymass}--\eqref{eq:betafull} are augmented by boundary conditions:
\begin{gather}
u(0) = u_a > 0, \quad H(0) = H_a > 0, \label{eq:leftbc} \\
T(x_c) = \frac{1}{2} \omega \rho g H(x_c)^2,  \label{eq:rightbc} \\
H, u, T \quad \text{ continuous at } x = x_g.  \label{eq:xgregularity}
\end{gather}
Here $x=0$ is an upstream location where Dirichlet boundary conditions are applied (equations \eqref{eq:leftbc}), while at the calving front $x=x_c$ we have the standard hydrostatic pressure imbalance condition \eqref{eq:rightbc} \citep{SchoofMarine1}.

\subsection*{On well-posedness and the grounding line}  Given data $B(x)$, $b$, $k>0$, $M(x)$, $x_c>0$, $z_o$, along with physical constants $g,n,\rho,\rho_w$, we expect the problem consisting of equations \eqref{eq:steadymass}--\eqref{eq:xgregularity} to be well-posed.  To our knowledge this has not been proved, nor do we attempt to prove it.  It is, however, worth considering the smoothness of the solution to \eqref{eq:steadymass}--\eqref{eq:rightbc}, including what hypotheses would lead to satisfying \eqref{eq:xgregularity} at the free (i.e.~unknown) location $x_g$ in the interior of the domain.

Indeed, suppose that, for physical reasons, the climatic-basal mass balance rate $M(x)$ and ice hardness $B(x)$ are bounded, and further that $B(x)$ is bounded below by a positive constant.  From integrating $M$, equation \eqref{eq:steadymass} implies that the flux $q=uH$ is absolutely-continuous and thus bounded.  If there is a positive lower bound on thickness $H$, then we can conclude that the magnitude of $u$ is bounded because $u=q/H$.  If the magnitude of the driving stress $-\rho g H h_x$ is bounded then equation \eqref{eq:steadySSA} implies $T$ is absolutely-continuous.  By equation \eqref{eq:Tstress} this implies $u$ has a bounded and integrable derivative, and thus that $u$ is also absolutely-continuous.  From these facts we could then return to the flux and write $H=q/u$ which shows $H$ is absolutely-continuous away from locations where $u=0$ (e.g.~divides).  In summary, assuming \emph{(i)} that an integrable solution $(H,u,T)$ to \eqref{eq:steadymass}--\eqref{eq:rightbc} exists, \emph{(ii)} that the functions $M,B$ are bounded and integrable, \emph{(iii)} that $B$ is bounded below by a positive constant, \emph{(iv)} that a positive lower bound on thickness exists, and \emph{(v)} that an upper bound on the magnitude of the driving stress exists, then we can regard conditions \eqref{eq:xgregularity}, giving continuity at the grounding line, as properties of the solution instead of as part of the ``imposed'' problem statement.

\begin{table}
\caption{Notation and SI units.  Values of physical constants.}\label{tab:notation}

\medskip
\begin{tabular}{llll}
Symbol & Description & Units \\ \hline
$B$ & ice hardness; $=A^{-1/n}$ & $\text{Pa}\,\text{s}^{1/3}$  \\
$b$ & bedrock elevation & m \\
$\beta$ & sliding coefficient & $\text{Pa}\,\text{s}\,\text{m}^{-1}$ \\
$g$ & acceleration of gravity  & 9.81 $\text{m}\,\text{s}^{-2}$\\
$H$ & ice thickness & m \\
$h$ & ice surface elevation & m \\
$k$ & pressure-scaled sliding coefficient  & $\text{s}\,\text{m}^{-1}$ \\
$M$ & climatic-basal mass balance rate & $\text{m}\,\text{s}^{-1}$ \\
$n$ & Glen exponent in ice flow law & 3 \\
$\rho$ & density of ice & 910 $\text{kg}\,\text{m}^{-3}$ \\
$\rho_w$ & density of sea water & 1028 $\text{kg}\,\text{m}^{-3}$ \\
$T$ & $z$-integrated longitudinal stress & $\text{Pa}\,\text{m}$ \\
$\tau_{b}$ & basal shear stress applied to ice & Pa \\
$u$ & horizontal velocity & $\text{m}\,\text{s}^{-1}$ \\
$(x,z)$ & flow-line cartesian coordinates & m  \\
$x_g$ & grounding line & m  \\
$x_c$ & calving front & m  \\
$z_o$ & ocean surface elevation & m \\
$\omega$ & Archimedean factor; $=1 - \rho/\rho_w$ & $0.115$
\end{tabular}
\end{table}


\section{Exact solution}

\subsection*{\bod's parabola}  \citebod built, based on minimal existing literature, a rigorous theory of the flow of glaciers and ice sheets.  His test case was Br\'uarj\"okull, a glacier on the northern margin of Vatnaj\"okull in Iceland.  It flows over a smooth bed for 20 km, from a location where its thickness is 600 m, to a zero thickness margin.  This glacier is entirely grounded.  He shows that a good fit to measured surface elevations can be made using his model.

He initially states an equation for the ice sheet surface elevation which has both vertical-plane shear and longitudinal stress within the ice.  However, he says this equation ``is quite tedious and very difficult to handle especially because of the [shear term].  It is therefore fortunate that [the shear] term appears to be small compared to the [basal sliding] term.''  Then he drops the shear term and writes an equation in which driving stress is balanced entirely by sliding resistance.  He solves and analyzes this plug flow model, for which we state his equations in detail.

As is perhaps best-known about \bod's work, he chooses the surface mass balance rate to be
\begin{equation}
M = a (H - \Hela)  \label{eq:bodmassbalance}
\end{equation}
for a surface mass balance gradient $a>0$ and equilibrium-line altitude $\Hela$.  His basal resistance formula scales with the overburden pressure, so that the basal shear stress is
\begin{equation}
\tau_b = - \beta u = - k \rho g H u.  \label{eq:bodstresschoice}
\end{equation}
He then writes the ice flux as $uH=-(H/k) H_x$, or equivalently the ice velocity as
\begin{equation}
u = - \frac{1}{k} H_x. \label{eq:bodvelocity}
\end{equation}
(We return to equation \eqref{eq:bodvelocity} below, explaining it in more modern language.)  Combining equations \eqref{eq:bodmassbalance} and \eqref{eq:bodvelocity} with mass continuity \eqref{eq:steadymass} yields equation (17) in \citepbod, namely
\begin{equation}
a (H - \Hela) + (k^{-1} H H_x)_x = 0.  \label{eq:bodsteady}
\end{equation}
His thickness solution to this equation is \citep[equations (18) and (23)]{Bodvardsson}
\begin{equation}
H(x) = H_0 (1 - (x/L_0)^2)  \label{eq:bodsoln}
\end{equation}
where $H_0 = 1.5 \Hela$ and $a k L_0^2 = 9 \Hela$ \citep[equation (24)]{Bodvardsson}.

Despite its simplicity, equation \eqref{eq:bodsoln} is an exact solution to equation \eqref{eq:bodsteady}, with boundary condition $H(0)=H_0$, as the reader may verify.  Formula \eqref{eq:bodsoln} defines the solid parabola shown in Figure \ref{fig:twoparabolas}.  \bod does not offer a reason why there should be such a simple quadratic solution to \eqref{eq:bodsteady}.  In fact, his solution seems to be a new result for a narrow class of nonlinear second-order ODEs; see Appendix A.

Seeking a polynomial solution of \eqref{eq:bodsteady} with the single boundary condition $H(0)=H_0$ generates a location $x=L_0$ where both the flux and the thickness are zero; see Appendix A.  \bod explicitly considers solution \eqref{eq:bodsoln} as solving a free boundary problem in this sense.

\bod had a particular ``sliding law'' in mind, namely equation \eqref{eq:bodstresschoice}, in which the sliding coefficient scales with overburden pressure.  The modern reader may protest that $\beta$ should instead be a function of, e.g.~proportional to, effective pressure $N=\rho g H - P$ where $P$ is the subglacial water pressure.  However, it is not unreasonable to suppose that $P$ also scales with---is roughly a fixed fraction $\lambda$ of---overburden pressure.  In that case we have equations $\beta = c N$, $N=\rho g H - P$, and $P = \lambda \rho g H$ which combine to give $\beta = c (1-\lambda) \rho g H$.  Thus, if the reader so wishes, in \bod's model we can write $\beta = k \rho g H$ with $k=c (1-\lambda)$.

\citebod was apparently first cited by \cite{Weertman61stability}, who decided that his sliding, plug-flow physics should be replaced by a shear deformation model more like the shallow ice approximation.  This replacement seems to have influenced readers from then on.  The surface balance parameterization \eqref{eq:bodmassbalance} also reappears in \cite{Weertman61stability}, among other places.  It parameterizes a potential climatic instability, which was \bod's, Weertman's, and most readers', major interest.  We are, however, interested now in \bod's solution to the ice flow equations themselves.


\subsection*{An SSA re-interpretation}  We have claimed that \eqref{eq:bodsoln} exactly solves a combination of the steady flow-line mass-continuity equation \eqref{eq:steadymass} and the SSA stress balance equation \eqref{eq:steadySSA}, but equation \eqref{eq:bodvelocity} may be mysterious.  However, suppose we look for solutions of \eqref{eq:steadymass} and \eqref{eq:steadySSA} with constant vertically-integrated longitudinal stress, $T_x = 0$.  In that case equation \eqref{eq:steadySSA} and the scaling \eqref{eq:bodstresschoice} implies \bod's formula for the velocity, namely equation \eqref{eq:bodvelocity}.  Furthermore, equations \eqref{eq:steadymass} and \eqref{eq:bodmassbalance} then give \bod's main equation \eqref{eq:bodsteady}.  That is, if \emph{(i)} $T_x = 0$, \emph{(ii)} sliding resistance is linear and scales with overburden pressure, and \emph{(iii)} climatic-basal mass balance rate is proportional to elevation above the equilibrium line, then we can recover a simple parabolic profile for $H(x)$, namely equation \eqref{eq:bodsoln} which is a solution to \eqref{eq:bodsteady}.

But what does the condition ``$T_x = 0$'' imply as a relation among the modeled quantities?  Given a thickness profile $H(x)$ and a strain rate profile $u_x(x)$, we may interpret $T_x = 0$ as a statement of \emph{variable ice hardness} $B(x)$.  Generally, variation in hardness can be physically-explained, for example in numerical models which use the SSA in a temperature-dependent way \citep{BBssasliding}, so assuming that hardness is $x$-dependent requires no conceptual extensions in the marine ice sheet modeling context.  Numerical models can use the exact solution we are building for verification without requiring special modifications to incorporate variable hardness.

Thus the variable hardness $B(x)$ used here allows us to ``manufacture'' an exact solution \citep{BLKCB}.  Specifically, the assumption $T=T_0$ in equation \eqref{eq:Tstress} yields this formula for ice hardness in grounded ice,
\begin{equation}
B(x) = \frac{T_0}{2 H |u_x|^{(1/n)-1} u_x}. \label{eq:hardnessdefine}
\end{equation}

\subsection*{Extending the exact solution to floating ice}  The value $T_0$ in \eqref{eq:hardnessdefine} can be set by a downstream stress condition, just as it is in many models for flowline ice shelves \citep[e.g.][]{MISMIP2012,SchoofMarine1}.  Two well-known observations are relevant:  \emph{(i)}  For floating ice with $\beta=0$, equations \eqref{eq:steadymass}--\eqref{eq:rightbc} also have a known exact solution, specifically in the case where the climatic-basal mass balance rate $M$ and the ice hardness $B$ are constant \citep{vanderVeen83,vanderVeen}.  \emph{(ii)}  The vertically-integrated longitudinal stress $T$ in a flowline ice shelf (i.e.~one without lateral stresses) satisfies equation \eqref{eq:rightbc} at each location $x$ in the shelf, that is, $T(x) = \frac{1}{2} \omega \rho g H(x)^2$, because this is a first integral of equation \eqref{eq:steadySSA} if $\beta=0$.

Based on these observations we can construct a marine ice sheet exact solution by extending \bod's grounded solution to the floating ice.  Note the grounding-line thickness is $H(x_g) = (\rho_w/\rho) z_o$ if we take $b=0$.  Then from \bod's thickness solution \eqref{eq:bodsoln} we can determine $x_g$.  At $x_g$, from \eqref{eq:bodvelocity} and \eqref{eq:bodsoln}, we know $u(x_g)$ as well.  For $x_g \le x \le x_c$, the floating ice shelf, we then set $M(x) = M(x_g)$ as constant from the formula \eqref{eq:bodmassbalance}, thus making $M(x)$ continuous across the grounding line, while at the same time allowing us to use van der Veen's construction which is based on a constant climatic-basal mass balance rate.  The equation $T(x) = \frac{1}{2} \omega \rho g H(x)^2$ determines $T_o=T(x_g)$ for use in equation \eqref{eq:hardnessdefine}, which determines $B(x_g)$ in particular, and so then we set $B(x)=B(x_g)$ for $x_g \le x \le x_c$, a constant also needed in van der Veen's construction.

The results of the above choices are the following formulae for an exact marine ice sheet satisfying our steady model equations \eqref{eq:steadymass}--\eqref{eq:xgregularity}.  The velocity comes from combining the \bod (1955) and van der Veen (1983) results,
\begin{equation}
u(x) = \begin{cases} \frac{2 H_0}{k L_0^2}\,(x + x_a), & 0 \le x \le x_g, \\
                     u_s(x), & x_g \le x \le x_c.
       \end{cases} \label{eq:marinevel}
\end{equation}
where $u_s(x)$ is defined by
\begin{equation}
u_s(x)^{n+1} = u(x_g)^{n+1} + \frac{C_s}{M(x_g)} \left(Q_s(x)^{n+1} - Q_g^{n+1}\right), \label{eq:vanderveenvel}
\end{equation}
$C_s = \left(\rho g \omega/(4 B(x_g))\right)^n$, $Q_g = u(x_g) H(x_g)$, and $Q_s(x) = Q_g + M(x_g) (x-x_g)$.  Similarly the thickness is:
\begin{equation}
H(x) = \begin{cases} H_0 \left(1 - (\frac{x+x_a}{L_0})^2\right), & 0 \le x \le x_g, \\
                     \frac{Q_s(x)}{u_s(x)}, & x_g \le x \le x_c.
       \end{cases} \label{eq:marinethickness}
\end{equation}
Formulas \eqref{eq:marinevel} and \eqref{eq:marinethickness} define the continuous functions which are shown in Figure \ref{fig:exactmarine}, using the specific values in Table \ref{tab:exactsoln}.

\begin{table}
\caption{Specific values used in (above line) or determined by (below line) the equations which define the exact solution shown in Figures \ref{fig:exactmarine}--\ref{fig:em-detail}.  Note ``g.l.'' = grounding line and ``c.f.'' = calving front.}\label{tab:exactsoln}

\medskip
\begin{tabular}{llll}
Symbol & Description & Units \\ \hline
$a$ & surface mass balance gradient & 0.003 $\text{a}^{-1}$ \\
$b$ & bedrock elevation & 0 m \\
$H_0$ & thickness used in \eqref{eq:bodsoln} & 3000 m  \\
$\Hela$ & equilibrium line altitude & 2000 m  \\
$k$ & scaled sliding coefficient & 757.366 $\text{s}\,\text{m}^{-1}$ \\
$L_0$ & length used in \eqref{eq:bodsoln} & 500 km  \\
$x_a$ & offset & 100 km  \\
$z_o$ & ocean surface elevation & 504.572 m \\
$H(0)$ & thickness at $x=0$ & 2880 m  \\
$u(0)$ & ice velocity at $x=0$ & 100 $\text{m}\,\text{a}^{-1}$  \\ \hline
$x_g$ & location of g.l. & 350 km  \\
$B(x_g)$ & ice hardness at g.l. & $4.614 \times 10^{8}$ $\text{Pa}\,\text{s}^{1/3}$  \\
$H(x_g)$ & thickness at g.l. & 570 m  \\
$M(x_g)$ & climatic-basal mass balance rate at g.l. & -4.290 $\text{m}\,\text{a}^{-1}$  \\
$T(x_g)$ & stress at g.l. & $1.665 \times 10^{8}$ $\text{Pa}\,\text{m}$  \\
$u(x_g)$ & ice velocity at g.l. & 450 $\text{m}\,\text{a}^{-1}$  \\
$x_c$ & location of c.f. & 390 km  \\
$H(x_c)$ & thickness at c.f. & 182.938 m  \\
$T(x_c)$ & stress at c.f. & $0.171 \times 10^{8}$ $\text{Pa}\,\text{m}$  \\
$u(x_c)$ & ice velocity at c.f. & 464.092 $\text{m}\,\text{a}^{-1}$  \\
\end{tabular}
\end{table}

From the thickness $H(x)$ and the velocity $u(x)$ we can find continuous functions $M(x)$ and $B(x)$ for the full  flowline by using equations \eqref{eq:bodmassbalance} and \eqref{eq:hardnessdefine}.  These functions are shown in Figure \ref{fig:exactMB}.  Then we can use equation \eqref{eq:Tstress} to find $T(x)$; this is shown in Figure \ref{fig:exactbetaT}.  In Figure \ref{fig:exactbetaT} we also show the sliding coefficient $\beta(x)$, which drops to zero discontinuously at $x_g$.  Finally Figure \ref{fig:em-detail} shows a detail of the grounding line and floating ice in the exact solution.

\begin{figure}[ht]
\onecol{em-M-B}
\caption{The climatic-basal mass balance rate $M(x)$ (solid) and ice hardness $B(x)$ (dashed) of the exact solution.} \label{fig:exactMB}
\end{figure}

\begin{figure}[ht]
\onecol{em-beta-T}
\caption{The sliding coefficient $\beta(x)$ (solid) and the vertically-integrated longitudinal stress $T(x)$ (dashed) for the exact solution.  The solid curve shows $\beta = k \rho g H$ on both sides of the grounding line.  The actual basal resistance experienced by the shelf drops to zero at the grounding line (dotted).} \label{fig:exactbetaT}
\end{figure}

\begin{figure}[ht]
\onecol{em-geometry-detail}
\caption{Detail of Figure \ref{fig:exactmarine}, showing the floating ice shelf geometry and velocity.} \label{fig:em-detail}
\end{figure}

The floating ice shelf is a relatively short 40 km; see Figures \ref{fig:exactmarine} and \ref{fig:em-detail}.  To explain, note that the equilibrium line $\Hela$ in \bod's (1955) solution is high on the ice sheet because of its relation to the upstream ice thickness in the construction of the exact solution (i.e.~$\Hela = (2/3) H_0$).  This in turn implies $M(x_g)$ is quite negative (equation \eqref{eq:bodmassbalance}; see Figure \ref{fig:exactMB}).  Because van der Veen's (1983) solution uses constant climatic-basal mass balance rate, and because we want continuity for $M(x)$, we therefore have an ice shelf experiencing rapid melting.  The location of the calving front $x_c$ must, of course, be put upstream of the location where the ice has melted away.  As a result of these same factors we also see a rapid decline in the stress $T(x)$ from its constant grounded value to its small value at $x_c$ (Figure \ref{fig:exactbetaT} and Table \ref{tab:exactsoln}).  Though the ice shelf shown here is very wedge-like, the thickness $H(x)$ for floating ice comes from formula \eqref{eq:marinethickness}, and it is not a linear function because $u_s(x)$ is not constant on the shelf.


\section{Numerical Results}

\subsection*{Verification of a grid-free ``shooting'' numerical method}  In our steady flowline case the model equations form a two-point boundary value problem (BVP) for ordinary differential equations (ODEs).  Specifically, the three first-order ODEs \eqref{eq:steadymass}--\eqref{eq:Tstress} are subject to two boundary conditions \eqref{eq:leftbc} at $x=0$ and one at $x=x_c$, the calving-front stress condition \eqref{eq:rightbc}.  

A nonlinear ``shooting'' method \citep[section 17.1]{Pressetal} applies to this problem.  We use the correct values for $u(0)$ and $H(0)$ from \eqref{eq:leftbc} and guess an additional value $T_0$ for $T(0)$.  Then we use a numerical ODE initial value problem (IVP) solver to compute a solution $(\tilde u(x),\tilde H(x),\tilde T(x))$ from $x=0$ to $x=x_c$.  The failure of the ODE IVP solution to satisfy boundary condition \eqref{eq:rightbc} is a measure of the wrongness of $T_0$.  In fact, based on \eqref{eq:rightbc} we define the function
\begin{equation}
F(T_0) = \tilde T(x_c) - \frac{1}{2} \omega \rho g \tilde H(x_c)^2  \label{eq:Fbisection}
\end{equation}
and then we can apply a numerical method to find the solution (root) $\hat T_0$ to the problem $F(T_0)=0$.  This root gives us complete initial conditions so that the ODE IVP solution also solves the two-point BVP \eqref{eq:steadymass}--\eqref{eq:rightbc}.

A robust root-finding method is bisection \citep[section 9.1]{Pressetal}.  It is guaranteed to converge if $F$ is continuous and if an initial bracket is given, which is easy to find in this case.  Regarding faster root-finding methods than bisection, such as Newton's method, we observe that $F'$ may not exist because of the low regularity of the solution at the interior point $x=x_g$. However, by using our exact solution we will see clear evidence that the bisection iteration succeeds in finding the root $\hat T_0$ to many digits despite the uncertain smoothness of $F$.

\begin{figure}[ht]
\onecol{em-error}
\caption{Pointwise error in thickness (upper panel) and in velocity (lower panel) from an adaptive numerical ODE scheme.  Both the ``cheating'' case (solid), where we use the exactly-correct initial value for $T$, and the ``realistic'' case (dashed), where the shooting method converges on the correct initial value for $T$ by the bisection method, are shown.} \label{fig:shoot-error}
\end{figure}

This ``shooting'' method has the advantage that the advanced stepsize control mechanism of an ODE IVP solver determines the spatial grid points, so as to solve the ODEs to a desired tolerance.  Thereby we avoid \emph{a priori} choice of the grid, and in this sense the method is grid-free.  In this case we use LSODA from the ODEPACK collection \citep{Hindmarsh1983ODEPACK} because it both automatically adjusts stepsize to achieve desired tolerance and because it automatically switches method when stiffness \citep[section 16.6]{Pressetal} is detected.

We now apply this grid-free procedure to the same problem for which we have the exact solution given in equations \eqref{eq:marinevel}--\eqref{eq:marinethickness}.  Using relative tolerance $10^{-12}$ and absolute tolerance $10^{-14}$ for LSODA we get the results in Figure \ref{fig:shoot-error}.  We have shown the error in two runs, one in which we have used the exactly-correct initial value $T_0$ (``cheating'') and one in which we start with a large initial bracket on $T_0$ and converge on the correct calving-front boundary condition through shooting and bisection (``realistic'').  In the ``cheating'' runs we see that the numerical error just from solving the ODE, i.e.~independent of errors in boundary conditions, is quite small, perhaps the 10th or 11th digit for $H$ and $u$.  The much larger error seen in the ``realistic'' case suggests, however, that $F$ in \eqref{eq:Fbisection} is significantly irregular.  Apparently matching the calving-front boundary condition by numerical shooting from upstream causes the loss of 4 or 5 digits of accuracy.  Nonetheless, even in this ``realistic'' case our numerical method achieves 6 or 7 digit accuracy over the whole domain, including in the immediate vicinity of the grounding line.  Note that though the peak inaccuracy is near the grounding line, that error is only modestly larger than errors elsewhere.

The ODE solver also detects the grounding line as a point of transition to shorter (spatial) steps, as seen in Figure \ref{fig:shoot-dt-adaptive}.  More significantly, however, the grounded ice requires a stiff method while the floating ice allows a nonstiff one, according to the automatic switch mechanism in LSODA.  Note that high accuracy (e.g. 6 or 7 digits) is achieved in the ``realistic'' case despite rather large grid spacing in the grounded ice, with large portions at 5--10 km spacing.  The spacing drops to a minimum of 100 m just downstream of the grounding line at $x_g=350$ km.

\begin{figure}[ht]
\onecol{em-dt-adaptive}
\caption{The adaptive numerical ODE scheme in the ``realistic'' case makes steps of 1 km to 10 km in grounded ice, but at the grounding line $x_g=350$ km the step size is reduced to a few hundred meters.  The adaptive mechanism automatically switches from a stiff method where grounded (circles) to a non-stiff method where floating (stars).} \label{fig:shoot-dt-adaptive}
\end{figure}

In all numerical tests we have treated $M(x)$ as a predetermined field (i.e.~the one shown in Figure \ref{fig:exactMB}).  This removes the climatically-important elevation--accumulation feedback, and the associated instability, of interest to \citebod and others.  This feedback can, however, be restored by using equation \eqref{eq:bodmassbalance} to determine $M$ from $H$.


\subsection*{Linearization around the exact solution}  The above numerical evidence shows that a distinct change in stiffness occurs at the grounding line.  To analyze this we linearize the model equations around the exact solution.  Denote the exact solution $(u_0,H_0,T_0)$ and consider a small perturbation $u = \hu + \eps \tu$, $H = \hH + \eps \tH$, and $T = \hT + \eps \tT$.  Denote the column vector of perturbations by $\bw = [\tu, \tH, \tT]^T$.  Assuming $u_x > 0$, equations \eqref{eq:steadymass}--\eqref{eq:Tstress} imply that, to first order in $\eps$, the perturbation solves this linear ODE system in grounded ice,
\begin{align}
&\begin{bmatrix}
\frac{2}{n} B \hH (\hu_x)^q & 0 & 0 \\
\hH & \hu & 0 \\
0 & -\rho g \hH & 1
\end{bmatrix}
\bw_x \notag \\
&\qquad =
\begin{bmatrix}
0 & - 2 B (\hu_x)^{1/n} & 1 \\
- \hH_x & - \hu_x & 0 \\
-k \rho g \hH & - k \rho g \hu + \rho g \hH_x & 0
\end{bmatrix}
\bw  \label{eq:earlylinearization}
\end{align}
where $q = \frac{1}{n} - 1$.  In floating ice only the last rows differ from \eqref{eq:earlylinearization}:
\begin{equation}
\begin{bmatrix}
 & \dots & \\
0 & - \omega \rho g \hH & 1
\end{bmatrix}
\bw_x
=
\begin{bmatrix}
 & \dots & \\
0 & \omega \rho g \hH_x & 0
\end{bmatrix}
\bw
\label{eq:floatinglinearization}
\end{equation}

If we define $L(x)$ and $R(x)$ to be the left- and right-side matrices in \eqref{eq:earlylinearization} and \eqref{eq:floatinglinearization} then $\bw$ solves
\begin{equation}
\bw_x = A(x) \bw \label{eq:linearization}
\end{equation}
along its whole length, where $A(x) = L(x)^{-1} R(x)$.  Note that we have chosen $L(x)$ to be lower triangular.  The inverse of $L(x)$ exists, and is easy to compute, because its diagonal entries are nonzero.

The linear ODE system \eqref{eq:linearization} is stiff if there is a large ratio of magnitudes in the eigenvalues of $A(x)$ \citep{Pressetal}.  Because the entries and eigenvalues of $A(x)$ are exactly computable using the exact solution values $(\hu(x),\hH(x),\hT(x))$, we can plot in Figure \ref{fig:stiffness} the $x$-dependent ``stiffness ratio'' for $A(x)$, namely the ratio of absolute values of the real parts of the largest and smallest eigenvalues of $A(x)$, along the whole length of the flowline.  Note this ratio is small in the nonstiff floating ice.  This ratio, which is independent of the direction of integration (i.e.~upstream versus downstream), is by no means the last word on quantifying stiffness, which turns out to be difficult generally \citep[e.g.][]{HighamTrefethen1993}.

\begin{figure}[ht]
\onecol{em-stiffness-ratio}
\caption{Stiffness ratio $|\operatorname{Re}(\lambda_1)|/|\operatorname{Re}(\lambda_3)|$ for the linearized problem \eqref{eq:linearization}, where $\lambda_i$ are the eigenvalues of $A(x)$ in \eqref{eq:linearization}.} \label{fig:stiffness}
\end{figure}

We believe that the strong stiffness contrast at the grounding line is significant in explaining large near-grounding-line errors made by gridded numerical methods \citep{Gladstoneetal2010,MISMIP2012}.  The stiffness ratio drops by a factor of almost ten at the grounding line, though it is largest in the interior part of the grounded ice.  It is possible that models of modified effective pressure near the grounding line \citep{Leguyetal2014TCD}, in sliding laws that are parameterized by effective pressure \citep{Schoof2005cavitation}, can reduce this stiffness contrast.

\subsection*{Verification of a fixed-grid finite difference numerical method}  We also implemented an equally-spaced, second-order, finite difference scheme using Newton iteration, described in Appendix B.  Our exact solution allows us to measure, for the first time in a rapidly-sliding marine ice sheet context, the errors from such a numerical scheme of the common type implemented in practical marine ice sheet models \citep[e.g.][]{PollardDeConto2009WAIS,Winkelmannetal2011}.

Figure \ref{fig:convmarine} shows that the maximum numerical thickness and velocity errors are observed to converge at much less than the optimal $O(\Delta x^2)$ rate under grid refinement \citep{MortonMayers}.  This is essentially because of the loss of smoothness in the exact solution at the grounding line.  By contrast, use of the grounded-only \citebod exact solution, without a grounding line, confirms that the same finite difference method gives optimal $O(\Delta x^2)$ convergence; not shown.

\begin{figure}[ht]
\onecol{convmarine}
\caption{Maximum errors in ice thickness (upper panel) and velocity (lower panel) on grids with spacing from $20$ km down to $5$ m.  When initialized with the exact solution, the numerical scheme converges at a rate $\Delta x^{1.08}$ for both thickness and velocity (large dots plus dotted line).  For a more realistic initial iterate the convergence rate is initially good, but at resolutions below 1 km the Newton iteration fails to converge (stars; see text).} \label{fig:convmarine}
\end{figure}

It is important to distinguish the errors attributable to the finite difference discretization itself from errors attributable to imperfect convergence of the nonlinear iterative solver which is applied to solve the discretized equations.  To evaluate the former type of errors we initialized the nonlinear solver with the exact solution values.  These are not the exact solutions of the discretized equations but they are (obviously) close.  We see converged solutions to the discretized equations down to 5 m grids.  The resulting thickness and velocity errors, at the millimeter and millimeter-per-year level, respectively (Figure \ref{fig:convmarine}), are larger than from the adaptive (grid-free) higher-order ODE method above; compare Figure \ref{fig:shoot-error}.  Nonetheless errors at this level are certainly acceptable, if they were to represent the realistic case.

However, if we use a simple ``wedge'' initial iterate, which has a linear thickness profile from the upstream initial condition $H(0)$ down to 300 m at the calving front, and a similar linear velocity profile increasing from $u(0)$ to 300 $\text{m}\,\text{a}^{-1}$ at the calving front, then we see more realistic results which reflect the experience of ice sheet modelers addressing these equations.  Here the initial iterate is relatively far from the exact solution.  Difficulties arise in the global convergence behavior of the Newton solver, even though standard line search techniques are used in these computations \citep{Kelley}.  For grids finer than 1 km the iteration for this scheme does not converge, apparently because the Jacobian matrix is not providing useful directional information as to the location of the solution of the discretized equations.

Our results suggest a lack of nonlinear solver robustness that we attribute to the nonsmooth, stiffness-constrasting properties of the problem near the grounding line.  Grounding line parameterizations \citep[e.g.][]{Gladstoneetal2010,Feldmannetal2014} may act like homotopy continuation methods \citep{Kelley} to improve global solver behavior in this case, but such considerations go beyond our scope.  No regularization of grounding line discontinuities and slope discontinuities in the formulas for $\beta(x)$ and $h(x)$ (respectively) were applied in the current paper.


\section{Conclusion}  As noted by \cite{BLKCB}, \cite{Wesseling}, and many other sources, verification of numerical methods is a valuable first step in effective numerical modeling of realistic flows.  This is especially so in geophysical flows where validation by comparison to controlled laboratory experiments is difficult.  Thus the rediscovery of an exact solution to a marine ice sheet problem is a welcome development.  Even though this solution is for a steady-state and flat bed case, it provides a partial alternative to hard-to-interpret intercomparison results \citep{MISMIP2012}.

Because this solution is found in some of the first work in theoretical glaciology \citepbod, we have ``rescued'' an early approach to sliding dynamics.  The ``rapid-sliding'' case turns out to be one of the first dynamical situations examined, even though most early efforts at global views of ice dynamics tended toward the plastic ice \citep{Orowan,Nye52plastic}, frozen bed \citep{Vialov}, and vertical-shear dominated \citep{Weertman61stability} models, and these came to dominate the field until recent decades.

Citations of \citebod usually relate to its theory of climatic instability for glaciers and ice sheets in which surface mass balance rate depends on elevation.  The only exception to this pattern known to the current author is that \cite{Fowler1992} refers to the basal-shear-dominated dynamics of \citebod as ``approximate results.''  We hope that the current work revives interest in \bod's ice dynamical solution and corrects this (understandable) misreading of his results as approximate.

Application of the new exact solution also reveals one feature of the marine ice sheet problem that we feel has been overlooked.  Namely that there is a strong stiffness contrast, in the sense of differential equations, in the flowline case at the grounding line.  This is, conceptually, in addition to the loss of smoothness seen at the grounding line.  Both smoothness and stiffness must be addressed by numerical methods.  Modelers should have more than grid refinement in mind as they attempt to model grounding lines correctly.


\subsection*{Acknowledgements (and erratum)}  Comments from referees R.~Gladstone and K.~Hutter have focussed and improved the paper.  Thanks to G.~A\dh algeirsdottir, H.~Blatter, and H.~Bj\"ornsson for tracking down a copy of \citebod and a bibliography of the works of Gunnar \bod.  \cite{BLKCB} incorrectly identify the constant accumulation SIA solution as ``Bodvarsson (1955)--\cite{Vialov}''.  In fact it is attributable only to Vialov.  Finally, note that spellings of \bod include ``Bodvardsson'' (e.g.~in \citebod) and ``Bodvarsson'' in the literature.


%         References
%\bibliography{ice-bib}
%\bibliographystyle{igs}

\begin{thebibliography}{29}
\expandafter\ifx\csname natexlab\endcsname\relax\def\natexlab#1{#1}\fi
\expandafter\ifx\csname selectlanguage\endcsname\relax
  \def\selectlanguage#1{\relax}\fi

\bibitem[Balay and others, 2011]{petsc-user-ref}
Balay, S. and others, 2014. {PETS}c {U}sers {M}anual, {\em Tech. Rep. ANL-95/11
  - Revision 3.4\/}, Argonne National Laboratory.

\bibitem[Bodvardsson, 1955]{Bodvardsson}
Bodvardsson, G., 1955. On the flow of ice-sheets and glaciers, {\em
  J{\"o}kull\/}, {\bf 5}, 1--8.

\bibitem[Bueler and Brown, 2009]{BBssasliding}
Bueler, E. and J.~Brown, 2009. Shallow shelf approximation as a ``sliding law''
  in a thermodynamically coupled ice sheet model, {\em J. Geophys. Res.\/},
  {\bf 114}, f03008, doi:10.1029/2008JF001179.

\bibitem[Bueler and others, 2005]{BLKCB}
Bueler, E., C.~S. Lingle, J.~A. Kallen-Brown, D.~N. Covey and L.~N. Bowman,
  2005. Exact solutions and numerical verification for isothermal ice sheets,
  {\em J. Glaciol.\/}, {\bf 51}(173), 291--306.

\bibitem[Cogley and others, 2011]{massbalanceglossary}
Cogley, G. and others, 2011. Glossary of {M}ass-{B}alance and {R}elated
  {T}erms, {IHP-VII Technical Documents in Hydrology No. 86, IACS
  Contribution No. 2, UNESCO-IHP, Paris}.

\bibitem[Feldmann and others, 2014]{Feldmannetal2014}
Feldmann, J., T.~Albrecht, C.~Khroulev, F.~Pattyn and A.~Levermann, 2014.
  Resolution-dependent performance of grounding line motion in a shallow model
  compared to a full-{S}tokes model according to the {MISMIP3d}
  intercomparison, {\em J. Glaciol.\/}, {\bf 60}(220), 353--360.

\bibitem[Fowler, 1992]{Fowler1992}
Fowler, A.~C., 1992. Modelling ice sheet dynamics, {\em Geophysical \&
  Astrophysical Fluid Dynamics\/}, {\bf 63}, 29--65.

\bibitem[Gladstone and others, 2010]{Gladstoneetal2010}
Gladstone, R.~M., A.~J. Payne and S.~L. Cornford, 2010. Parameterising the
  grounding line in flow-line ice sheet models, {\em The Cryosphere\/}, {\bf
  4}, 605--619.

\bibitem[Higham and Trefethen, 1993]{HighamTrefethen1993}
Higham, D. and L.~N. Trefethen, 1993. Stiffness of {ODE}s, {\em BIT\/}, {\bf
  33}, 285--303.

\bibitem[Hindmarsh, 1983]{Hindmarsh1983ODEPACK}
Hindmarsh, A.~C, 1983. {ODEPACK, A Systematized Collection of ODE Solvers},
  {\em IMACS Transactions on Scientific Computation\/}, {\bf 1}, 55--64, edited
  by Stepleman et al.

\bibitem[Kelley, 1987]{Kelley}
Kelley, C.~T., 1987. Solving Nonlinear Equations with Newton's Method,
  Fundamentals of Algorithms, SIAM Press.

\bibitem[Leguy and others, 2014]{Leguyetal2014TCD}
Leguy, G.~R., X.~S. Asay-Davis and W.~H. Lipscomb, 2014. Parameterization of
  basal hydrology near grounding lines in a one-dimensional ice sheet model,
  {\em The Cryosphere Discussions\/}, {\bf 8}(1), 363--419.

\bibitem[MacAyeal, 1989]{MacAyeal}
MacAyeal, D.~R., 1989. Large-scale ice flow over a viscous basal sediment:
  theory and application to ice stream {B}, {A}ntarctica, {\em J. Geophys.
  Res.\/}, {\bf 94}(B4), 4071--4087.

\bibitem[Morton and Mayers, 2005]{MortonMayers}
Morton, K.~W. and D.~F. Mayers, 2005. Numerical {S}olutions of {P}artial
  {D}ifferential {E}quations: {A}n {I}ntroduction, Cambridge University Press,
  2nd ed.

\bibitem[Nye, 1952]{Nye52plastic}
Nye, J.~F., 1952. A method of calculating the thicknesses of the ice-sheets,
  {\em Nature\/}, {\bf 169}(4300), 529--530.

\bibitem[Orowan, 1949]{Orowan}
Orowan, E., 1949. Discussion, {\em J. Glaciol.\/}, {\bf 1}(5), 231--236.

\bibitem[Pattyn and others, 2012]{MISMIP2012}
Pattyn, F., C.~Schoof, L.~Perichon and others, 2012. Results of the {M}arine
  {I}ce {S}heet {M}odel {I}ntercomparison {P}roject, {MISMIP}, {\em The
  Cryosphere\/}, {\bf 6}, 573--588.

\bibitem[Pollard and DeConto, 2009]{PollardDeConto2009WAIS}
Pollard, David and Robert~M. DeConto, 2009. Modelling {W}est {A}ntarctic ice
  sheet growth and collapse through the past five million years, {\em
  Nature\/}, {\bf 458}, 329--333.

\bibitem[Press and others, 1992]{Pressetal}
Press, W.~H., S.~A. Teukolsky, W.~T. Vetterling and B.~P. Flannery, 1992.
  Numerical {R}ecipes in {C}: {T}he {A}rt of {S}cientific {C}omputing,
  Cambridge University Press, 2nd ed.

\bibitem[Schoof, 2005]{Schoof2005cavitation}
Schoof, C., 2005. The effect of cavitation on glacier sliding, {\em Proc. R.
  Soc. A\/}, {\bf 461}, 609--627.

\bibitem[Schoof, 2006]{SchoofStream}
Schoof, C., 2006. A variational approach to ice stream flow, {\em J. Fluid
  Mech.\/}, {\bf 556}, 227--251.

\bibitem[Schoof, 2007]{SchoofMarine1}
Schoof, C., 2007. Marine ice-sheet dynamics. {P}art 1. {T}he case of rapid
  sliding, {\em J. Fluid Mech.\/}, {\bf 573}, 27--55.

\bibitem[van~der Veen, 1983]{vanderVeen83}
van~der Veen, C.~J., 1983. A note on the equilibrium profile of a free floating
  ice shelf, {IMAU} Report V83-15. State University Utrecht, Utrecht.

\bibitem[van~der Veen, 2013]{vanderVeen}
van~der Veen, C.~J., 2013. Fundamentals of {G}lacier {D}ynamics, CRC Press, 2nd
  ed.

\bibitem[Vialov, 1958]{Vialov}
Vialov, S.~S., 1958. Regularities of glacial shields movement and the theory of
  plastic viscous flow, International Association of Scientific Hydrology
  Publication 47 (Symposium at Chamonix 1958---Physics of the movement of ice),
  266--275.

\bibitem[Weertman, 1961]{Weertman61stability}
Weertman, J., 1961. Stability of ice-age ice sheets, {\em J. Geophys. Res.\/},
  {\bf 66}, 3783--3792.

\bibitem[Weis and others, 1999]{WeisGreveHutter}
Weis, M., R.~Greve and K.~Hutter, 1999. Theory of shallow ice shelves, {\em
  Continuum Mech. Thermodyn.\/}, {\bf 11}(1), 15--50.

\bibitem[Wesseling, 2001]{Wesseling}
Wesseling, Pieter, 2001. Principles of {C}omputational {F}luid {D}ynamics,
  Springer-Verlag.

\bibitem[Winkelmann and others, 2011]{Winkelmannetal2011}
Winkelmann, R., M.~A. Martin, M.~Haseloff, T.~Albrecht, E.~Bueler, C.~Khroulev
  and A.~Levermann, 2011. The {P}otsdam {P}arallel {I}ce {S}heet {M}odel
  ({PISM-PIK}) {P}art 1: {M}odel description, {\em The Cryosphere\/}, {\bf 5},
  715--726.
\end{thebibliography}

\appendix

\subsection{Appendix A: \bod's little theorem}  \citebod does not identify a source for the exact parabolic thickness solution to his plug flow equations, and we have not been able to attribute it to earlier work.  We summarize his result as the theorem that there is a unique polynomial solution $y(x)$ to the nonlinear second-order differential equation
\begin{equation}
  (y y')' = c_1 y + c_0  \label{eq:abstractode}
\end{equation}
satisfying the single boundary condition $y(0) = y_0 > 0$, subject to both an initial downslope assumption ($y'(0) \le 0$) and to the technical inequality $2 c_1 y_0 + 3 c_0 \ge 0$.

Equation \eqref{eq:abstractode} is equation (17) in \citebod.  The additional assumptions (i.e.~initial downslope plus the technical inequality) are unstated, though he comments that there is ``one and only one solution which is admittable from the physical point of view.''

Here we justify this little theorem and, generally following \citebod, derive relations among parameters which allow a solution.  The unique polynomial solution to this problem may be interpreted as solving a free boundary problem for the first positive zero $x_0>0$ of $y(x)$.  In \bod's context $x_0=L_0$ is the length of the glacier, the location of the margin in the everywhere-grounded case.

It is easy to see by substitution into \eqref{eq:abstractode} that nontrivial solutions of degree $d$, i.e.~of the form $y(x) = C x^d + (\text{lower degree})$ with $C \ne 0$, exist only if $d=2$.  In that case we seek solutions which satisfy the boundary conditions $y(0)=y_0$ and $y'(0) \le 0$, so
\begin{equation}
y(x) = y_0(1 - \alpha x + \gamma x^2)  \label{eq:abstractsoln}
\end{equation}
for some $\alpha\ge 0$ and $\gamma$ which are to be determined; this is equation (18) in \citepbod.  Substitution of \eqref{eq:abstractsoln} into \eqref{eq:abstractode} gives the two equations
\begin{equation}
3 y_0^2 \alpha^2 = 2 c_1 y_0 + 3 c_0 \quad \text{ and } \quad 6 y_0 \gamma = c_1.  \label{eq:abstractrelations}
\end{equation}
These two relations determine $\alpha$ and $\gamma$ from $c_0$ and $c_1$.  The first relation explains the technical inequality, noting $3 y_0^2 \alpha^2 \ge 0$ of course.

In the main text, \bod's problem relates four numbers to the unknown glacier thickness $y(x)=H(x)$: the initial (upstream) ice thickness $y_0=H_0$, an ablation gradient $a>0$, the equilibrium-line altitude $\Hela$, and a scaled sliding coefficient $k>0$.  He has $c_0=k a \Hela$ and $c_1=-ka$ in \eqref{eq:abstractode} so the technical inequality says $3 \Hela \ge 2 H_0$ after simplification.  This causes the equilibrium line altitude to be relatively high on the glacier.


\subsection{Appendix B: A finite difference scheme}  The steady-state equations for mass continuity \eqref{eq:steadymass} and stress balance \eqref{eq:steadySSA} form a coupled system that can be approximately solved by the centered, second-order finite difference scheme described here.  There is no claim that this scheme is optimal, but merely that it is a reasonable fixed-grid method for initial evaluation.  Because we use it to solve a steady-state problem, it generalizes to the time-dependent case as an implicit method.

We define an equally-spaced grid on the domain $[0,x_c]$.  Because the boundary condition at the calving front evaluates the stress $T$, we put the right endpoint $x_c$ at a ``staggered'' location halfway in-between grid points.  Thus if $N$ is the number of spaces then we define $\Delta x = x_c / (N+1/2)$ and $x_j = j\Delta x$ for $j=0,\dots,N+1$.  Denote the numerical approximations $H_i\approx H(x_i)$ and $u_i \approx u(x_i)$.  Let $x_j^* = x_j + \Delta x/2$ be the staggered location, for $j=0,\dots,N$.  Note $x_c = x_N^* < x_{N+1}$.  Denote $B_j^*=B(x_j^*)$ and $M_j^*=M(x_j^*)$.

The mass continuity equation \eqref{eq:steadymass} is approximated by a second-order difference centered at the staggered location.  For $j=0,\dots,N$,
\begin{equation}
\frac{u_{j+1} H_{j+1} - u_j H_j}{\Delta x} - M_j^* = 0 \label{eq:steadymassFD}
\end{equation}

In equation \eqref{eq:steadySSA} we avoid infinite viscosity by using a regularization \citep{SchoofStream}.  Let $\eps=1/x_c$ per year, i.e.~a strain rate corresponding to 1 m$/$a velocity change over the whole domain.  Also let $q = (1-n)/n$, and define
\begin{equation}
F(u_l,u_r) = \left(\left(\frac{u_r-u_l}{\Delta x}\right)^2 + \eps^2\right)^{q/2} \frac{u_r-u_l}{\Delta x}. \label{eq:viscregFD}
\end{equation}
Then we approximate the stress $T$ at staggered points,
\begin{equation}
T_{j}^* = B_j^* \left(H_j + H_{j+1}\right) F(u_j,u_{j+1}), \label{eq:TFD}
\end{equation}
for $j=0,\dots,N$.  Equation \eqref{eq:steadySSA} is approximated by
\begin{equation}
\frac{T_{j}^* - T_{j-1}^*}{\Delta x} - \beta_j u_j - \rho g H_j \frac{h_{j+1} - h_{j-1}}{2 \Delta x} = 0 \label{eq:steadySSAFD}
\end{equation}
for $j=1,\dots,N$, where $\beta_j = k \rho g H_j$ if the ice is grounded at $x_j$ (i.e.~if $\rho H_j \ge \rho_w (z_o - b)$) and $\beta_j=0$ if the ice is floating, and where $h_j = H_j + b$ if the ice is grounded and $h_j = \omega H_j + z_o$ if the ice is floating.  Thus equation \eqref{eq:steadySSAFD} applies as stated both for grounded and floating ice.

At this point we have $2N+4$ scalar unknowns, namely $u_j$ and $H_j$ for $j=0,\dots,N+1$.  There are $2N+1$ nonlinear equations in \eqref{eq:steadymassFD} and \eqref{eq:steadySSAFD} above.  The two upstream Dirichlet equations \eqref{eq:leftbc}, namely $u_0=u(0)$ and $H_0=H(0)$, brings the number of equations to $2N+3$.  The following approximation of the calving front condition \eqref{eq:rightbc}, completes the system:
\begin{equation}
\frac{1}{2} \omega \rho g \left(\frac{H_N + H_{N+1}}{2}\right)^2 = T_N^*, \label{eq:rightbcFD}
\end{equation}
where $T_N^*$ is the approximation given in \eqref{eq:TFD}.

Thus we have a system of $2N+4$ nonlinear equations in the same number of unknowns.  One can write this system abstractly as $\mathbf{F}(\mathbf{v})=0$.  These equations are solved by Newton's method \citep{Kelley}, as implemented in the PETSc library \citep{petsc-user-ref}.  A residual evaluation function computes $\mathbf{F}(\mathbf{v})$ given $\mathbf{v}$.  A finite-difference Jacobian matrix $J=\mathbf{F}'$ can then be computed by PETSc, and this allows us to solve systems up to size about $N=10^3$.  We also implemented an exact Jacobian using by-hand differentiation of the above formulas.  For initial guesses sufficiently near the exact solution, this exact Jacobian permits solutions of the system for up to $N=10^5$.  A full analysis of the robustness and convergence rate of this Newton solver is beyond the scope of the current paper.

\end{document}
